\chapter{Overview}
\pagenumbering{arabic}
\label{Overview}
\index{overview}

This chapter \footnote{Last change:
$ $Id$ $}
provides an overview of the \aipspp\ system.

% ----------------------------------------------------------------------------

\section{The AIPS++ directory hierarchy}
\label{Directories}
\index{directory!hierarchy}

In the following discussion of the \aipspp\ directory hierarchy we will
assume that \aipspp\ has been installed in directory \file{/aips++}.  This
is the preferred location, but in practice the \aipspp\ ``root'' directory
can reside anywhere.  The root directory is generally referred to as
\file{\$AIPSROOT}.  Many other \aipspp\ directories have standard variable
names as listed in \sref{variables}.

The major subdirectories of \file{/aips++} are \file{code} (\file{\$AIPSCODE}),
\file{docs} (\file{\$AIPSDOCS}), and one or more architecture-specific
subdirectories with names such as \file{sun4sol\_ntv} and \file{alpha\_gnu}
which contain the \aipspp\ system - that is, everything needed to run \aipspp,
including executables and sharable objects.  These architecture-specific
subdirectories are referred to collectively as \file{\$AIPSARCH}.

% ----------------------------------------------------------------------------

\subsection{Code directories}
\label{Code directories}
\index{directory!code}

The directory hierarchy beneath \file{/aips++/code}, or \file{\$AIPSCODE},
consists of a collection of ``packages'' which are contained in separate
subdirectories.  The principle package is \file{/aips++/code/aips} which
contains the source code for a \cplusplus\ class library required by all other
packages.  It also contains applications common to all areas of astronomical
data processing.

The \file{dish}, \file{synthesis}, and \file{vlbi} packages contain standard
classes and applications common to single dish, aperture synthesis, and VLBI
data processing tasks from all radio telescopes.  Apart from a dependence of
the \file{vlbi} package on the \file{synthesis} package, inclusion of these
standard packages in an end-user installation is optional.

A \file{contrib} package is the place for source code contributed from the
\aipspp\ user community for redistribution with \aipspp.  If found to be
generally useful, code from the \file{contrib} package may eventually be
merged into one of the standard packages, but otherwise it is unsupported.

A \file{trial} package contains source code written by members of the
\aipspp\ consortium but not yet accepted for inclusion in the standard
distribution.

Each \aipspp\ consortium member is also entitled to maintain a package for
data processing applications specific to its telescope(s).  The sources for
these classes and applications reside in the consortium-specific packages:
\file{atnf}, \file{bima}, \file{drao}, \file{nfra}, \file{nral}, \file{nrao},
and \file{tifr}.  These may or may not use the standard \file{dish},
\file{synthesis}, and \file{vlbi} packages, and their installation is also
optional.

The standard \aipspp\ packages (\file{aips}, \file{dish}, \file{synthesis},
and \file{vlbi}) contain an \file{implement} subdirectory which contains class
header and implementation files, a \file{fortran} subdirectory which contains
\textsc{fortran} subroutines, an \file{apps} directory containing applications,
and \file{scripts} and \file{data} subdirectories for package-related
procedure files and system data such as standard colour-maps, calibrator lists
and source catalogues.

The \file{implement} and \file{fortran} subdirectories may contain module
subdirectories which serve to collect software ``modules'' in one place.  For
example, the \file{Tables} module contains all class header and implementation
files pertaining to the \file{Table} class, and classes derived from, and
related to it.  The \file{implement} directory and all module subdirectories
may also contain a \file{test} subdirectory which contains one or more
self-contained test programs specifically for the module.

All files associated with an \aipspp\ application reside in a subdirectory of
the \file{apps} directory of the same name as the application.  Each
application must reside in its own subdirectory.

The substructure of the consortium-specific packages is left entirely to
\aipspp\ consortium members to determine.

There are a number of other subdirectories of \file{/aips++/code} which are
unrelated to packages.  The \file{install} subdirectory contains all of the
utilities required to install and maintain \aipspp\ as discussed in this
document.  \file{doc} contains \aipspp\ documentation sources, including the
\aipspp\ ``specs'', ``memos'', and ``notes'' series, and reference and design
documentation in the corresponding subdirectories.

Also below \file{/aips++/code} is an \file{include} subdirectory which
contains symbolic links to the \file{implement} subdirectories for each
package.  The purpose of these symlinks is to allow \aipspp\ includes to be
specified as ``\code{\#include <package/Header.h>}'' by adding
\mbox{\file{-I/aips++/code/include}} to the include path.

On \aipspp\ consortium installations an additional \file{admin} subdirectory
of \file{/aips++/code} contains files relating to the administration of the
\aipspp\ project.

The \file{/aips++/code} directory hierarchy appears as follows:

\begin{verbatim}
         :                                         +--- App1 ----
         :                           +--- apps ----+--- App2 ----
         |                           |             +--- .... ----
         |                           |
         |                           |             +-- Module1 --+--- test ----
         |                           +- implement -+-- Module2 --+--- test ----
         |                           |             +--- .... ----
         |                           |             +--- test ----
         |                           |
         |             +--- aips ----+-- fortran --+...
         |             |             +-- scripts --
         |             |             +--- data ----
         |             |
         |             +--- dish ----+...
         |             +- synthesis -+...
         |             +--- vlbi ----+...
         |             +-- contrib --+...
         |             +--- trial ---+...
         |             |
         |             +--- atnf ----+...
         |             +--- bima ----+...
         |             +--- drao ----+...
         |             +--- nfra ----+...
         |             +--- nral ----+...
         |             +--- nrao ----+...
         |             +--- tifr ----+...
/aips++ -+--- code ----+
         |             |             +- codedevl --
         |             |             +- codemgmt --
         |             +-- install --+- docutils --
         |             |             +- printer ---
         |             |             +-- <arch> ---
         |             |
         |             |             +-- design ---+...
         |             |             +--- html ----
         |             |             +--- memos ---
         |             |             +--- notes ---
         |             +---- doc ----+-- papers ---
         |             |             +-- project --
         |             |             +- reference -
         |             |             +--- specs ---
         |             |             +--- .... ----
         |             |
         |             +-- include --
         |             |
         |             |             +- personnel -
         :             +--- admin ---+- projects --
         :                           +-- system ---
\end{verbatim}

% ----------------------------------------------------------------------------

\subsection{Documentation directories}
\label{Documentation directories}
\index{directory!documentation}

The \file{/aips++/docs} subdirectory, or \file{\$AIPSDOCS} (\sref{variables})
contains \textsc{ascii}, \textsc{html} and \textsc{PostScript} documents
compiled from the sources in \file{/aips++/code/doc} whose directory structure
it shadows:

\begin{verbatim}
         :             +--- aips ----+...
         :             +--- dish ----+...
         |             +--- .... ----+...
         |             |
         |             +-- design ---+...
         |             +--- html ----
         |             +--- memos ---
/aips++ -+--- docs ----+--- notes ---
         |             +-- papers ---
         |             +-- project --
         |             +- reference -
         :             +--- specs ---
         :             +--- .... ----
\end{verbatim}

\noindent
\file{html} files compiled from inline comments in the \cplusplus\ source code
are deposited in the various package-specific subdirectories.

% ----------------------------------------------------------------------------

\subsection{System directories}
\label{System directories}
\index{directory!system}

The \aipspp\ system directory hierarchy is created in the first instance when
\aipspp\ is installed (see \sref{Installation}), and maintained thereafter by
the \code{sysdirs} target which is invoked by \code{allsys} in the top level
\filreff{makefile}{makefiles}.

Except for the \filref{aipsinit} files in \file{\$AIPSROOT}, the \aipspp\ 
system is completely self-contained within the architecture-specific
subdirectory, referred to as \file{\$AIPSARCH} (\sref{variables}).  In this
context ``architecture'' should be interpreted to include variants in the
operating system version and compiler.

In practical terms, the fact that the \aipspp\ system does not rely on
anything in the \file{\$AIPSCODE} directories allows the source code to be
deleted after the \aipspp\ installation is complete in a production-line
system.

At an \aipspp\ development site with machines of several architectures where
the source code must be retained, the strict separation of code from system
provides for the \file{\$AIPSARCH} tree to reside on a machine of the
corresponding architecture without duplication of the \file{code} directories.
With thoughtful unix filesystem management it also allows that the \aipspp\ 
system for one architecture may remain available even if the server for any
other architecture has crashed.

The system directories have the following structure:

\begin{verbatim}
         :             +---- lib ----+...
         :             +---- bin ----
         |             |
         |             +-- libdbg ---+...
         |             +-- bindbg ---
         |             |
         |             +-- bintest --
         |             |
         |             +-- libexec --
         |             |
         |             +---- aux ----
         |             +---- tmp ----+...
         |             |
         |             |             +--- info ----
/aips++ -+-- (arch1) --+---- doc ----+--- man1 ----
         |             |             +--- cat1 ----
         |             |             +---  :
         |             |
         |             |             +-- (host1) --
         |             +-- (site1) --+-- (host2) --
         |             |             +--   :
         |             |
         :             +-- (site2) --+...
         :             +---   :
\end{verbatim}

\noindent
The \file{lib} directory contains optimized static object libraries and
possibly sharable objects.  It sometimes also contains a subdirectory which
serves as a \cplusplus\ template repository.  The \file{bin} directory
contains \aipspp\ system scripts and optimized applications.  It is added to
the \code{PATH} environment variable by the \exeref{aipsinit} scripts.

The \file{libdbg} and \file{bindbg} directories contain debug versions of the
libraries and executables.  The \file{bindbg} directory is not usually
populated but serves as the temporary residence for executables which are in
the process of being debugged.

The \file{bintest} directory is used temporarily to store test executables and
test results, and \file{libexec} contains scripts of various kinds which are
not meant to be executed directly but are instead included by other scripts.

Files which are produced as intermediaries of system generation are cached in
the \file{aux} directory.  In particular, it includes dependency lists
generated by the makefiles.  Temporary storage is provided during a rebuild
beneath the \file{tmp} directory.  The structure and usage of the \file{tmp}
directory hierarchy is soley the concern of the \aipspp\ makefiles.  It
contains subdirectories specific to each \aipspp\ package (see
\sref{Code directories}).

Online documentation is contained in the \file{doc} subdirectory.  This
includes unix manual pages and help files.  The \exeref{aipsinit} scripts add
this directory to the \code{MANPATH} environment variable if it is defined at
the time that \aipsexe{aipsinit} is invoked.

Finally, the \file{\$AIPSARCH} directory contains site subdirectories which
contain site-specific \file{aipsrc} and \file{makedefs} files (see
\sref{Configuration files}), and possibly host-specific subdirectories which
in turn contain host-specific \file{aipsrc} files.  Multiple site-, and
host-specific directories were provided to make it easier for a central site
to administer \aipspp\ for a collection of remote sites.  If properly
configured, it should allow a verbatim copy of the \aipspp\ system at the
central site to be downloaded at the remote site with only a minimum of
reconfiguration required.

% ----------------------------------------------------------------------------

\subsection{RCS directories}
\label{RCS directories}
\index{directory!rcs@\rcs}
\index{nfs@\textsc{nfs}}
\index{rcs@\rcs!directories|see{directory, \rcs}}
\index{archive \rcs\ repository|see{directory, \rcs}}
\index{master \rcs\ repository|see{directory, \rcs}}
\index{slave \rcs\ repository|see{directory, \rcs}}
\index{repository!rcs@\rcs|see{directory, \rcs}}

The \aipspp\ \rcs\ directories are maintained only at \aipspp\ code
development sites.  They are not provided with the end-user \aipspp\ 
distribution.

The one true \aipspp\ master \rcs\ repository resides on host
\host{aips2.nrao.edu} at NRAO's Array Operations Center in Socorro, New
Mexico, USA.  This is the working copy for \aipspp\ code management,
consortium sites access this one way or another for checkin and checkout (see
\sref{Code management}).

The \file{master} \rcs\ directory hierarchy is the same as the \file{code}
hierarchy described above except that the master has an \file{etc}
subdirectory which is used for central administrative purposes (see
\sref{Central services}).  In order to reduce the amount of network traffic
involved in a checkin/out the master only contains revisions for the current
and preceding major \aipspp\ version numbers (see \exeref{avers}).  A separate
\file{archive} repository contains all revisions other than those for the
current major version number.

\begin{verbatim}
         :             +--- etc -----
         :             |
         |             |             +--- apps ----+...
         |             |             +- implement -+...
         |             +--- aips ----+-- fortran --+...
         |             |             +-- scripts --
         |             |             +--- data ----
         +-- master ---+
         |             +--- dish ----+...
         |             +- synthesis -+...
         |             +--- vlbi ----+...
         |             +-- contrib --+...
         |             :
---------+
         |             +--- etc -----
         |             |
         |             |             +--- apps ----+...
         |             |             +- implement -+...
         |             +--- aips ----+-- fortran --+...
         |             |             +-- scripts --
         |             |             +--- data ----
         +-- archive --+
         |             +--- dish ----+...
         |             +- synthesis -+...
         |             +--- vlbi ----+...
         :             +-- contrib --+...
         :             :
\end{verbatim}

Most \aipspp\ consortium sites have the master \rcs\ repository remotely
\textsc{nfs} mounted beneath \file{\$AIPSROOT}, and some also maintain a copy
of the archive repository.

All \aipspp\ consortium sites, including that at Socorro, maintain a slave
\rcs\ repository which is regularly updated by the \aipspp\ code distribution
system (\sref{Code distribution}) so that it keeps track of the master (aside
from the \file{master/etc} directory).  This slave \rcs\ repository is used to
update the plain-text \file{code} directories (\sref{Code directories}).

The \aipspp\ \filref{makefiles} only ever refer to the \rcs\ repositories
indirectly, through a symbolic link called \file{rcs} which usually points to
\file{slave} but could, in principle, be switched to point to \file{master} or
\file{archive}.

\begin{verbatim}
         :
         :
         +- (master) --+...
         +- (archive) -+...
         |                           +--- apps ----+...
         |                           +- implement -+...
         |             +--- aips ----+-- fortran --+...
         |             |             +-- scripts --
         +---- rcs     |             +--- data ----
         |      =      |
/aips++ -+--- slave ---+
         |             +--- dish ----+...
         |             +- synthesis -+...
         |             +--- vlbi ----+...
         :             +-- contrib --+...
         :             :
\end{verbatim}

\noindent
\exeref{inhale} creates a \file{tmp} directory beneath the \file{slave}
directory for the duration of its operation.

% ----------------------------------------------------------------------------

\subsection{\unixexe{ftp} directories}
\label{ftp directories}
\index{directory!ftp@\unixexe{ftp}}
\index{master host}
\index{ftp@\unixexe{ftp}!directory|see{directory, \unixexe{ftp}}}

The \aipspp\ \unixexe{ftp} directories contain compressed \unixexe{tar} files
for distribution of \aipspp\ sources.  These reside directly under the
\aipspp\ master's root directory which on the \aipspp\ master host
\host{aips2.nrao.edu} is also the home directory of the anonymous
\unixexe{ftp} account, \verb+~+\file{ftp/pub}.  This allows the \unixexe{ftp}
directories, the master \rcs\ repository, and the archive \rcs\ repository to
be accessible via anonymous \unixexe{ftp}.

\begin{verbatim}
         +-- master ---+...
         +-- archive --+...
~ftp ----+
         |             +-- master ---
         +---- pub ----+--- code ----
                       +-- import ---
\end{verbatim}

There are three main \unixexe{ftp} directories.  The plain-text \aipspp\ 
sources required for an end-user installation are stored in \file{pub/code}.
Separate \unixexe{gzip}'d \unixexe{tar} files are provided for each \aipspp\ 
package (see \sref{Code directories}) in keeping with the optional nature of
package installations.

Third-party public-domain software required by \aipspp\ is redistributed in
\file{pub/import}.  If necessary, utilities essential for \aipspp\ are
automatically fetched from here by \exeref{configure} when \aipspp\ is first
installed.

The \file{pub/master} directory serves as a repository for the update files
produced by the code distribution system (see \exeref{exhale}).
\aipsexe{exhale} also creates a \file{tmp} subdirectory within it for the
duration of its operation.

It is intended that \unixexe{ftp} directories will also be provided for
patches and distribution of binary installations once \aipspp\ gains a user
base.  At that time consortium sites will also act as distributors of
\aipspp\ for their local geographic region, but until then the \unixexe{ftp}
distribution files will only be available from \host{aips2.nrao.edu}.

% ----------------------------------------------------------------------------
 
\section{AIPS++ variable names}
\label{variables}
\index{variables}
\index{variables!environment}
\index{variables!makefile}
\index{environment variables}
\index{makefile!variables}
\index{AIPSPATH@\code{AIPSPATH}|see{variables}}
\index{AIPSROOT@\code{AIPSROOT}|see{variables}}
\index{AIPSARCH@\code{AIPSARCH}|see{variables}}
\index{AIPSSITE@\code{AIPSSITE}|see{variables}}
\index{AIPSHOST@\code{AIPSHOST}|see{variables}}
\index{AIPSMSTR@\code{AIPSMSTR}|see{variables}}
\index{AIPSLAVE@\code{AIPSLAVE}|see{variables}}
\index{AIPSRCS@\code{AIPSRCS}|see{variables}}
\index{AIPSCODE@\code{AIPSCODE}|see{variables}}
\index{AIPSDOCS@\code{AIPSDOCS}|see{variables}}
\index{MSTRETCD@\code{MSTRETCD}|see{variables}}
\index{CODEINSD@\code{CODEINSD}|see{variables}}
\index{CODEINCD@\code{CODEINCD}|see{variables}}
\index{LIBDBGD@\code{LIBDBGD}|see{variables}}
\index{LIBOPTD@\code{LIBOPTD}|see{variables}}
\index{BINDBGD@\code{BINDBGD}|see{variables}}
\index{BINOPTD@\code{BINOPTD}|see{variables}}
\index{BINTESTD@\code{BINTESTD}|see{variables}}
\index{ARCHTMPD@\code{ARCHTMPD}|see{variables}}
\index{ARCHAUXD@\code{ARCHAUXD}|see{variables}}
\index{ARCHBIND@\code{ARCHBIND}|see{variables}}
\index{ARCHDOCD@\code{ARCHDOCD}|see{variables}}
\index{ARCHMAN1@\code{ARCHMAN1}|see{variables}}
 
A standard set of variable names is used in \aipspp\ scripts, makefiles and
elsewhere to refer to \aipspp\ directories (\sref{Directories}).  Except for
\code{\$AIPSPATH}, these are not stored in the environment but instead are
rederived from \code{\$AIPSPATH} whenever needed.  

These variable names are freely used in this manual with the following
typographic conventions:

\begin{description}
   \item{\code{AIPSPATH}}       ...the name of the variable.
   \item{\code{\$AIPSPATH}}     ...the value of the \code{AIPSPATH} variable,
      whether used as an environment variable or makefile variable.
   \item{\code{\$(AIPSPATH)}}   ...the value of the \code{AIPSPATH} variable
      used as a makefile variable.
\end{description}
 
Additional variables used by the \aipspp\ makefiles are described in the
entries for \filref{makedefs} and \filref{makefiles}.
 
\noindent
\verb+   ROOT = +the \aipspp\ home directory, also referred to as
                 \verb+~+\file{aips++}\\
\verb+   ARCH = +the machine architecture, e.g. \code{sun4}, \code{convex},
                 etc.\\
\verb+   SITE = +the local site name\\
\verb+   HOST = +the host name
 
\begin{verbatim}
   AIPSPATH = $ROOT $ARCH $SITE $HOST
 
   AIPSROOT = $ROOT
   AIPSARCH = $ROOT/$ARCH
   AIPSSITE = $ROOT/$ARCH/$SITE
   AIPSHOST = $ROOT/$ARCH/$SITE/$HOST
 
   AIPSMSTR = $AIPSROOT/master
   AIPSLAVE = $AIPSROOT/slave
   AIPSRCS  = $AIPSROOT/rcs
   AIPSCODE = $AIPSROOT/code
   AIPSDOCS = $AIPSROOT/docs
 
   MSTRETCD = $AIPSMSTR/etc
 
   CODEINSD = $AIPSCODE/install
   CODEINCD = $AIPSCODE/include

   INSTARCH = $CODEINSD/$ARCH
 
   LIBDBGD  = $AIPSARCH/libdbg
   LIBOPTD  = $AIPSARCH/lib
   BINDBGD  = $AIPSARCH/bindbg
   BINOPTD  = $AIPSARCH/bin
   BINTESTD = $AIPSARCH/bintest
 
   ARCHTMPD = $AIPSARCH/tmp
   ARCHAUXD = $AIPSARCH/aux
   ARCHBIND = $AIPSARCH/bin
   ARCHDOCD = $AIPSARCH/doc
   ARCHMAN1 = $ARCHDOCD/man1
\end{verbatim}

% ----------------------------------------------------------------------------

\section{Configuration files}
\label{Configuration files}
\index{system!configuration}
\index{configuration|see{system, configuration}}

Two sets of configuration files are used in \aipspp, the \file{aipsrc} and
\file{makedefs} files.

The \file{aipsrc} files store \code{keyword:value} entries used by \aipspp\ 
scripts and programs.  The mechanism is superficially similar to that of
\file{.Xdefaults} on which it is modelled.  The \file{aipsrc} files are
hierarchical - a user's \file{.aipsrc} is usually the one first consulted, if
no keyword definition is found therein the search continues to host-specific,
site-specific, and default \filref{aipsrc} files.

The \file{makedefs} files are \gnu\ makefiles which define
installation-specific variables.  All \aipspp\ \filref{makefiles} begin by
including \file{\$AIPSARCH/}\filref{makedefs} (\sref{variables}) which
contains default definitions and also some generally applicable rules and
targets.  This ``default'' \file{makedefs} in turn includes
\file{\$AIPSSITE/makedefs} which allows the default definitions to be
overridden.

% ----------------------------------------------------------------------------

\section{AIPS++ accounts and groups}
\label{Accounts and groups}
\index{aips2adm@\acct{aips2adm}}
\index{aips2mgr@\acct{aips2mgr}}
\index{aips2prg@\acct{aips2prg}}
\index{aips2usr@\acct{aips2usr}}
\index{code!management!accounts}
\index{code!management!uid}
\index{code!management!groups}
\index{code!management!gid}
\index{master host}
\index{accounts|see{code, management}}
\index{groups|see{code, management}}
\index{uid|see{code, management}}
\index{gid|see{code, management}}

A set of \aipspp\ accounts and groups have been defined to perform particular
functions.  The following descriptions include the standard names, user ids,
and group ids, but different names may be defined via the \filref{aipsrc}
mechanism:

\begin{itemize}
\item
    \acct{aips2adm} (uid=31414, gid=31414): Owns the master \aipspp\ 
    directories, \rcs\ version files and \unixexe{ftp} distribution files.
    \acct{aips2adm} is responsible for maintaining the \unixexe{ftp}
    directories, and in particular, runs the \exeref{exhale} \unixexe{cron}
    job on the master host \host{aips2.nrao.edu}.  \acct{aips2adm}'s other
    functions are described in chapter~\ref{Central services}.  This account
    usually need only be created in Socorro.

\item
    \acct{aips2mgr} (uid=31415, gid=31415): Owns an \aipspp\ installation,
    including directories, source files and binaries.  At development sites,
    \acct{aips2mgr}, and those in the \acct{aips2mgr} group, are \rcs\ and
    \unixexe{ftp} administrators for an \aipspp\ installation.  In particular,
    they can directly invoke the \unixexe{rcs} command on the \aipspp\ \rcs\ 
    files.  \acct{aips2mgr} has an active login account but this may be
    restricted to selected machines.  \acct{aips2mgr} runs the daily
    \exeref{inhale} \unixexe{cron} job.

\item
    \acct{aips2prg} (uid=31416, gid=31416): A generic programmer account and
    group.  In practice the account may not often be used but the
    \acct{aips2prg} group lists those who have permission to check sources
    into \aipspp, or check them out with a write lock (see
    \sref{Code management});
    there are no restrictions on checking sources out without a write lock.

\item
    \acct{aips2usr} (uid=31417, gid=31417): The \acct{aips2usr} account is
    intended to serve as a generic \aipspp\ usage account.  The
    \acct{aips2usr} group could be used to control access to \aipspp\ 
    executables and \aipspp\ data areas if this degree of control was desired
    by the \aipspp\ manager (the alternative being to give world execute
    and/or write permission).  Neither the account nor the group are currently
    recognized by the system.

\end{itemize}
