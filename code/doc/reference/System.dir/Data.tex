\chapter{Data Repository}
\label{data repository}
\index{data repository}
\index{installation|see{system, installation}}

This chapter \footnote{Last change:
$ $Id$ $}
describes how to install and contribute to the \aipspp\ data repository.

The intended audience for this repository includes both \aipspp\ developers
who use data to implement their applications and users who access standard
catalogs and images which are components of the repository. The data is
typically provided as \aipspp\ tables, but some data is provided as FITS
files.

In the future, the goal is that components of the repository can be
automatically mirrored to the user's personal workspace when they are not
available in the the local \aipspp\ installation. This will minimize the
amount of data which all sites must keep locally while making the entire
repository available to users.

% ----------------------------------------------------------------------------

\section{Mirroring the Repository}
\label{data repository mirroring}
\index{data repository!mirroring}

The \aipspp\ data repository is generated at a central site and is then
mirrored by all consortium and user sites. This allows the central site
to build and maintain the data, and it allows user sites to easily stay
in sync with the master repository. In fact, it is easy for these secondary
sites to act as mirrors from which tertiary sites can retrieve the data.

The first step to retrieving the data repository is to install
\htmladdnormallinkfoot{CVSup}{http://www.polstra.com/projects/freeware/CVSup/}. While
CVSup is optimized for mirroring \htmladdnormallinkfoot{CVS}{http://www.cyclic.com/}
repositories, it can mirror arbitrary collections of files. CVSup is the mirroring
tool used by \htmladdnormallinkfoot{FreeBSD}{http://www.freebsd.org/}.

It is easiest to just install CVSup binaries because CVSup is written in
\htmladdnormallinkfoot{Modula-3}{http://www.research.digital.com/SRC/modula-3/html/home.html}.
CVSup binaries for many architectures can be found at the
\htmladdnormallinkfoot{FreeBSD FTP site}{ftp://ftp.FreeBSD.org/pub/FreeBSD/development/CVSup/binaries/}.
You can choose between versions which include or exclude the GUI. However, the GUI
version is recommended since the GUI is nice and can be disabled with a command
line flag.

Once you have installed CVSup, you are ready to create a supfile which describes
how to retrieve and install the data repository. First log in as \acct{aips2mgr}
and create a directory where the repository should be mirrored, e.g. \file{/aips++/data}.
It is fine for many \aipspp\ installations to share the same data repository. So if
you have more than one \aipspp\ installation, you will likely want to mirror the
repository in a central location and then just create symbolic links from the
\aipspp\ installations to the repository. If you have only one \aipspp\ installation,
then calling the directory \file{data} and putting it in the root of the
\aipspp\ installation (at the same level as \file{code}) is probably the best choice.

Next edit a file called \file{supfile} in the directory just created. To retrieve the whole
repository, which currently isn't so large, you would put the following in the supfile:

\begin{verbatim}
   *default host=aips2.nrao.edu compress
   *default base=YOUR_DIRECTORY_HERE
   *default release=all
   *default delete use-rel-suffix
   data
\end{verbatim}

\noindent
You should replace \code{YOUR\_DIRECTORY\_HERE} in the \code{base} assignment with
the fully qualified path to the directory you just created to contain the repository.

After you have created the supfile, you are ready to install the repository. Just
run:

\begin{verbatim}
   yourhost% cvsup supfile
\end{verbatim}

\noindent
If you don't run this from the newly created directory, you will
have to fully qualify the path to the supfile for \unixexe{cvsup}. This should run
and create a mirror of the data provided on the central \aipspp\ server.

If you want to continue to keep your data repository in sync with the repository
on the master \aipspp\ server, you need to create a \unixexe{cron} job entry to
run \unixexe{cvsup} regularly. This entry might look like:

\begin{verbatim}
   00 6 * * 0 /usr/local/bin/cvsup -g /aips++/data/supfile 2>&1 | mail aips2mgr
\end{verbatim}

\noindent
This would update the data repository once per week on Sunday at 6AM.

If you have multiple \aipspp\ installations which will be sharing a single
data repository, create a symbolic link called \file{data} in the root of
each installation (at the same level as \file{code}) to the data repository.
% ----------------------------------------------------------------------------

\section{Repository Structure}
\label{data repository structure}
\index{data repository!structure}

Data in the repository is divided based on content. The top layer of the
repository divides the data into broad categories:

\begin{list}{}{\setlength{\rightmargin}{\leftmargin}}

\item{\textbf{catalogs/lines}} spectral line catalogs, e.g. the Poynter and Pickett \textit{Submillimeter, Millimeter, and Microwave Spectral Line Catalogue}.

\item{\textbf{catalogs/sources}} pulsar, continuum, and spectral line source catalogs.

\item{\textbf{geodetic}} data related to the earth, e.g. global positions, orientation,
magnetic field, etc.

\item{\textbf{ephemerides}} data related to the movement of planets, comets, and satellites.

\item{\textbf{demo}} data used in tests and demos of \aipspp.

\item{\textbf{nrao}} data specific to nrao facilities.

\item{\textbf{\textit{other observatory}}} data specific to an \textit{other observatory}.

\end{list}

\noindent
Beneath this broad categorization is the actual data which is used by
\aipspp\ applications.

% ----------------------------------------------------------------------------

\section{Contributing to the Repository}
\label{data repository contributing}
\index{data repository!contributing}

It is more difficult to setup things to contribute to the \aipspp\ data
repository. The first requirement is that the user must have an account
on the \aipspp\ central server, \host{aips2.nrao.edu}, and the user id on
\host{aips2.nrao.edu} must be a member of the \acct{aips2prg} group.
In addition, CVS and SSH must be installed on your local system. These
packages are widely available.

With \unixexe{ssh} you create your pair of keys using \unixexe{ssh-keygen}.
This will create \verb+~+\file{/.ssh/identity.pub} on your local machine. You then
add this file to \verb+~+\file{/.ssh/authorized\_keys} on \host{aips2.nrao.edu}.Make
sure \file{authorized\_keys} is readable only by the user. After that, you
are ready to try \unixexe{ssh}; try something like:

\begin{verbatim}
   yourhost% ssh aips2.nrao.edu which cvs
   yourhost% ssh aips2.nrao.edu which co
\end{verbatim}

\noindent
from your local machine. You must get these to return the path to the
\unixexe{cvs} and \unixexe{co} binaries before going further. These binaries
are located in \file{/opt/local/gnu/bin} on \host{aips2.nrao.edu}. To make
this path available, you may need to edit \verb+~+\file{/.ssh/environment} on
\host{aips2.nrao.edu}, and add a line like:

\begin{verbatim}
    PATH=/usr/bin:/opt/local/bin:/opt/local/gnu/bin
\end{verbatim}

\noindent
Shell variables, e.g. \verb+$PATH+, are not expanded.

Once you have resolved all of the problems involved with running remote
commands on \host{aips2.nrao.edu}, you are ready to check out a piece
of the data repository. First you must tell your local \unixexe{cvs}
installation to go to \host{aips2.nrao.edu} for the repository and to
use \unixexe{ssh} instead of \unixexe{rsh} to run remote commands. For
the Bourne-like shells, you do this like:

\begin{verbatim}
    yourhost% CVS_RSH=ssh
    yourhost% CVSROOT=:ext:aips2.nrao.edu:/aips++/cvs
    yourhost% export CVS_RSH CVSROOT
\end{verbatim}

\noindent
for C-like shells, you would do:

\begin{verbatim}
    yourhost% setenv CVS_RSH ssh
    yourhost% setenv CVSROOT :ext:aips2.nrao.edu:/aips++/cvs
\end{verbatim}

Now to check out a piece of the source from which the data repository is
built, you would do something like:

\begin{verbatim}
    yourhost% cvs co data/nrao
\end{verbatim}

\noindent
This would check out the data specific to NRAO. Unlike RCS, when you check
out files from CVS the files are not locked. Other users can check out and
check in changes to the same files. Checking out the files just provides
a copy of the current state of the files to work with. Any conflicts
between your changes and changes which have occurred since you checked
out the files are addressed at check in time.

% ----------------------------------------------------------------------------

\section{Using CVS}
\label{data repository cvs}
\index{data repository!cvs}

CVS is similar to RCS. However, the format for running CVS commands is:

\begin{verbatim}
   yourhost% cvs <COMMAND> <OPTIONS>
\end{verbatim}

\noindent
The command for checking out code is \code{co} (abbreviation for \code{checkout}).
The command for checking in code is \code{commit}. And the command for printing
the differences between your copy and the version you checked out is \code{diff}.

All commands in CVS are recursive by default. So if you wanted to list all of your
changes, you would go to the top of the tree and run:

\begin{verbatim}
   yourhost% cvs diff .
\end{verbatim}

\noindent
which says give me the difference in this directory including every directory beneath
this directory.

When you check out files, you can check out all of the repository source:

\begin{verbatim}
   yourhost% cvs co data
\end{verbatim}

\noindent
or only a small portion:

\begin{verbatim}
   yourhost% cvs co data/geodetic/IGRF
\end{verbatim}

After you have had code checked out for some time, you may want to bring it
up to date with the current state of the code. You use \code{update} to do
this, e.g.

\begin{verbatim}
   yourhost% cvs update .
\end{verbatim}

\noindent
\code{update} will not trample over your changes. It updates all files
files which have not been modified, and where possible it merges in
differences to files which you have modified without any problems. Where
conflicts occur, it includes your changes and other changes in
a \textit{diff-like} format. When you get a warning about this problem,
you must edit the file by hand and reconcile your changes and the other
changes.

For more information about CVS, see
the \htmladdnormallinkfoot{support page}{http://www.cyclic.com/CVS/support}
at \host{cyclic.com}.

% ----------------------------------------------------------------------------

\section{Repository Conventions}
\label{data repository conventions}
\index{data repository!conventions}

There are certain conventions which must be followed when contributing to
the repository. In general, the makefiles which are used to build the
data repository are much simpler, by design, then those used to build
the \aipspp\ system. This is possible because the data repository makefiles
only build the repository on one system and the work they must do is much
more limited than that of the \aipspp\ makefiles.

\subsection*{Organization}
All of the scripts, makefile includes, etc.~which are used to build the data
are kept beneath \file{config} directories. The directories which may be
found beneath \file{config} are \file{bin} for executable scripts,
\file{libexec} for scripts which are included rather than executed directly,
and \file{makedefs} for the small bundles of functionality which are
included by other makefiles to allow them to do their work. A \file{config}
tree can occur anywhere in the data repository source tree. The principal
is that the \file{config} files should be placed as far down on the
source tree as possible. For example, a script would only be placed in
the top level \file{data/config/libexec} directory if it is included
by scripts in more than one of the broad top level categories, e.g.
\file{measures\_util.g} is used in both \file{ephemerides} and
\file{geodetic} so it must be placed in this top level config directory.
If it were used only in \file{geodetic}, it would be moved down to
the \file{data/geodetic/config/libexec} directory. The reason for this
rule is so that wherever possible checking out the lowest level \file{config}
directory along with the top \file{config} directory and the source for a
particular data component of the repository will be sufficient to build
that particular component.

\subsection*{Using Scripts}
In many cases, an \aipspp\ table is simply copied from the data source
tree to the repository. There are, however, instances where an
\aipspp\ table is created (and updated) from files retrieved from remote sites.
During the build, all of the tools available in an \aipspp\ installation
will be available for the makefiles and scripts to use. However,
the plan is that no compilation, e.g. of \cplusplus\ code, should be
necessary to build the data repository. It should all be done with
scripts. Currently, it is done with Glish scripts. Should there be
cases where this is impossible this rule can be revisited.

When using scripts which build a piece of data, the exit status
of the script indicates the result. In particular, an exit status
of \code{0} indicates that the script completed successfully and
that the data item \textit{was} updated, an exit status of \code{1}
indicates that the script completed successfully but the data item
\textit{was not} updated, and finally, an exit status of \code{-1}
indicates that an error occurred during the execution of the script.
The exit status of the scripts is used to generate logs containing
only a list of the updates or errors which can then be used to keep
tabs on the system without getting daily email messages.

\subsection*{Makefiles}
In general, there is a one-to-one mapping between directories in
the data source tree and directories in the data repository tree.
The one slight deviation from this general rule is with tables.
For tables, there is a replication of the table directory. So for
example if we have a table called \file{demo/3C273XC1} in the
repository tree, this table would actually correspond to
\file{demo/3C273XC1/3C273XC1} in the source tree. The makefile,
scripts (which need not be beneath a \file{config/libexec} directory),
text files, etc.~which install and perhaps create this table are
found in the first \file{3C273XC1} directory while the \file{3C273XC1}
subdirectory only contains the table which is installed in the
repository.

When writing makefiles, there are some basic tools which should be
used. \code{\$(RUN)} \textit{must} be used to execute the scripts
used to generate a data component from scratch. This collects the
exit status of the script and then uses the status to produce the
update and error logs. Another tool which is often useful is
\code{\$(MKHIER)}. When a fully qualified path is passed in, this
will create the entire path if it doesn't exist, e.g. a makefile
rule like:

\begin{verbatim}
   doit:
           $(MKHIER) /tmp/just/trying/it/out
\end{verbatim}

\noindent
would cause all of the directories in \file{/tmp/just/trying/it/out}
to be created.

In addition to these tools, there are some variables which are useful:

\begin{list}{}{\setlength{\rightmargin}{\leftmargin}}

\item{\textbf{\$(DATA\_ROOT)}} fully qualified path to the root of the
data tree, \textit{set in each makefile}.

\item{\textbf{\$(DATA\_THISDIR)}} fully qualified path to the current
directory, \textit{set in each makefile}.

\item{\textbf{\$(DATA\_INSTALLTHISDIR)}} the install point in the data
repository which corresponds to the current directory.

\item{\textbf{\$(AIPSROOT)}} fully qualified path to the root of the
\aipspp\ installation available for use in building the data repository.

\item{\textbf{\$(AIPSARCH)}} fully qualified path to the architecture
directory of the \aipspp\ installation.

\end{list}

When writing makefiles in the data system, there are two targets which
are recursively built these are \code{build} and \code{install}. \code{build}
is the target which builds data elements from scratch. \code{install} is
the target which moves the data element from the source tree to the
mirror point for the repository. Obviously all makefiles will specify
an \code{install} target, but not all makefiles will specify an \code{build}
target.

Much of the standard work done by the makefiles is found in \textit{makedef}
components which are included in the makefiles for individual data
elements. All makefiles include a standard preample which does the minimum
setup required to find the \textit{makedef} include files:

\begin{verbatim}
   DATA_THISDIR  := $(shell pwd | sed -e 's=^/tmp_mnt/=/=')
   DATA_ROOT := $(shell echo $(DATA_THISDIR)/ | sed -e '{s@/data/.*$$@/data@;}')
   include $(DATA_ROOT)/config/makedefs/basic
\end{verbatim}

\noindent
All makefiles should simply include this at the beginning. The \file{basic} makedef
sets up all of the standard variables and rules for recursively building the
\code{build} and \code{install} targets. Including the \file{install.table}
makedef defines the rule which will copy a table from the current directory
to the repository. It is assumed that the table is beneath the current directory
and has the same name as the current directory. Including the \file{install.fits}
makedef defines the rule which will copy all fits files in the current directory
to the corresponding directory in the repository.

So for example, the makefile to just copy a table which is not updated as
part of the \code{build} process to the repository would look like:

\begin{verbatim}
   DATA_THISDIR  := $(shell pwd | sed -e 's=^/tmp_mnt/=/=')
   DATA_ROOT := $(shell echo $(DATA_THISDIR)/ | sed -e '{s@/data/.*$$@/data@;}')
   include $(DATA_ROOT)/config/makedefs/basic
   include $(DATA_ROOT)/config/makedefs/install.table
\end{verbatim}

\noindent
The makefile to install FITS files would look like:

\begin{verbatim}
   DATA_THISDIR  := $(shell pwd | sed -e 's=^/tmp_mnt/=/=')
   DATA_ROOT := $(shell echo $(DATA_THISDIR)/ | sed -e '{s@/data/.*$$@/data@;}')
   include $(DATA_ROOT)/config/makedefs/basic
   include $(DATA_ROOT)/config/makedefs/install.fits
\end{verbatim}

\noindent
and the makefile to build and install a table might look like:

\begin{verbatim}
   DATA_THISDIR  := $(shell pwd | sed -e 's=^/tmp_mnt/=/=')
   DATA_ROOT := $(shell echo $(DATA_THISDIR)/ | sed -e '{s@/data/.*$$@/data@;}')
   include $(DATA_ROOT)/config/makedefs/basic

   build:
           @$(RUN) glish IERSeop97.g

   include $(DATA_ROOT)/config/makedefs/install.table
\end{verbatim}

\noindent
A good example of how to create and update tables can be found in the
scripts and makefiles beneath \file{data/geodetic/IERSeop97}. In this
case, Glish is used to create and maintain the \file{IERSeop97} table.
