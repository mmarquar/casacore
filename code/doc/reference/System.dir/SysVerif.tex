\chapter{System verification}
\label{System verification}
\index{system verification}

Utilities for verifying the integrity of \aipspp\ \footnote{Last change:
$ $Id$ $}.

% ----------------------------------------------------------------------------

\section{\exe{assay}}
\label{assay}
\index{installation}
\index{system generation -- verification}
\index{assay}
 
\aipspp\ test program initiator.

\newpage

% ----------------------------------------------------------------------------

\newpage
\section{\file{Diagnostic makefile rules}}
\label{Diagnostic makefile rules}
\index{compilation}
\index{diagnostics - makefile rules}
\index{makefile rules - diagnostics}
\index{rules, makefile - diagnostics}
\index{system verification - makefiles}
\index{makefiles}
\index{makefile - application}
\index{makefile - applications}
\index{makefile - checkout}
\index{makefile - class implemention}
\index{makefile - documentation}
\index{makefile - fortran directory}
\index{makefile - install directory}
\index{makefile - package}
\index{makefile - scripts directory}
\index{makefile - test directory}
\index{makefile - top-level}
\index{targets}
\index{makefile.app}
\index{makefile.aps}
\index{makefile.chk}
\index{makefile.doc}
\index{makefile.ftn}
\index{makefile.imp}
\index{makefile.pkg}
\index{makefile.scr}
\index{makefile.tst}

\textsc{gnu} makefiles used to rebuild \aipspp\ 

\subsection*{Synopsis}

\begin{synopsis}
   \file{makefile}\\
   \file{makefile.\{app,aps,chk,doc,ftn,imp,pkg,scr,tst\}}
\end{synopsis}

\subsection*{Description}


\subsection*{Diagnostic targets}

The \aipspp\ makefile targets are listed below by category.  These lists
are not exhaustive, but do aim to cover everything of practical use.  In
particular, they omit targets which are intended for the internal use of the
makefiles.

A target is labelled as ``recursive'' if it causes \exeref{gmake} to be
invoked in all subdirectories.  It is ``general'' if it applies to all
makefiles; such targets are defined in \file{makedefs}.  A target is
``specific'' if defined in a specific makefile.

Some targets such as \file{help} have a general meaning, the specific
behaviour of which differs for specific makefiles.  These are referred to as
``general/specific'' and where appropriate the details of a target's behaviour
are described for each of the generic makefiles, for the top-level makefile
(\file{top}), and the installation makefile (\file{ins}).

\textbf{Diagnostics targets:}

The diagnostics targets are used to report makefile variable definitions, to
list targets and dependencies deduced by the makefiles, to print help
information, and especially for debugging the makefiles.

\begin{itemize}
\item
   \code{command} : (general)
   \\ Invoke a command or sequence of commands specified by the \code{COMMAND}
   variable defined on the \aipsexe{gmake} command line.  This makes most
   sense if the command sequence involves the usage of \file{makedefs}
   variables.

\item
   \code{printenv} : (general)
   \\ Print out the environment as seen by the commands within makefile rules.
   This is especially useful for diagnosing problems related to makefile
   variables not being exported, unresolved recursively defined variables, and
   variables which cannot be exported to the environment because they contain
   non-alphanumeric characters.

\item
   \code{eval\_vars} : (general)
   \\ Print variables specified by \code{VARS} on the \aipsexe{gmake} command
   line in a form suitable for \unixexe{eval}'ing into the environment in
   Bourne shell.  This is used in particular by \exereff{ax\_master}{ax},
   \exeref{depend} and \exeref{updatelib} for getting \filref{makedefs}
   variable definitions when invoked in stand-alone mode (when invoked via a
   makefile rule the required variables are explicitly \code{export}ed to
   them).  An example of its use would be

\begin{verbatim}
   eval `gmake -f $AIPSARCH/makedefs VARS="AR ARFLAGS" eval_vars`
\end{verbatim}

   \noindent
   This would create environment variables \code{AR} and \code{ARFLAGS} and
   give them the values of the \file{makedefs} variables of the same name.

\item
   \code{diagnostics} : (general)
   \\ This target simply invokes the \code{versions} and \code{show\_global}
   targets.  It is used by the code distribution system on cumulative updates
   to produce a status report for consortium installations (see
   \exeref{sneeze}).

\item
   \code{versions} : (general)
   \\ Print the installed version of certain utilities required by \aipspp,
   the main one being \aipsexe{gmake} itself.

\item
   \code{show\_vars} : (general)
   \\ Report the value of the makedefs variables specified by \code{VARS} on
   the \aipsexe{gmake} command line.

\item
   \code{show\_prg} : (general)
   \\ Report the value of all variables defined in \file{makedefs} which are of
   immediate interest to \aipspp\ programmers.  This is particularly useful
   for reporting the compiler options set by the site-specific \file{makedefs}
   in response to use of the \code{OPT} variable or alternate programmer
   compilation flags.

\item
   \code{show\_all} : (general)
   \\ This target simply invokes the \code{show\_global} and \code{show\_local}
   targets.

\item
   \code{show\_global}  : (general)
   \\ This target simply invokes the \code{show\_sys}, \code{show\_prg} and
   \code{show\_aux} targets.

\item
   \code{show\_sys}  : (general)
   \\ Report the value of all system variables defined in \file{makedefs}.

\item
   \code{show\_prg}  : (general)
   \\ Report the value of all programmer variables defined in \file{makedefs}.

\item
   \code{show\_aux}  : (general)
   \\ Report the value of all auxilliary variables defined in \file{makedefs}.
   (Certain of these may have been redefined in the specific makefiles.)

\item
   \code{show\_local} : (specific)
   \\ Report the value of all variables defined within the specific makefile.

\item
   \code{help} : (general/specific)
   \\ Print an itemized summary of all general and specific targets categorized
   as ``programmer'', ``system'', or ``diagnostic''.  The general targets are
   reported by \file{makedefs}, and the specific targets are appended by the
   specific makefile.
\end{itemize}

\subsection*{Notes}

\begin{itemize}
\item
   The implement makefile includes the system dependency lists and so may
   start slowly as it checks to see whether the list is up-to-date (regardless
   of whether the particular target uses it or not).  Inclusion of the
   dependency list can be circumvented by setting the \code{NODEP} variable
   (to anything).  This causes the makefile to start considerably faster.

   \code{NODEP} is set automatically if \aipsexe{gmake} is invoked from a
   directory which does not reside under \file{\$AIPSROOT}, that is, a
   programmer invokation.  However, system-oriented targets may also be
   invoked from a programmer directory, and this would cause the dependency
   analysis to be circumvented, and the dependency list to be ignored.  This
   may be legitimate if the target invoked does not actually use the
   dependency list, for example \code{chkout}, \code{cleancode} or
   \code{cleansys}.  In fact, the code distribution system sets \code{NODEP}
   explicitly when it invokes the top-level makefile for these recursive
   targets for the system (see \exeref{inhale}).  However, \code{NODEP} must
   not be set for an invokation of the implement makefile for a target which
   does use the dependency list.  A proper resolution of this problem would
   require the facility for a makefile to examine its target list, but this is
   not currently possible.
\end{itemize}

\subsection*{Examples}


\subsection*{See also}

The \textsc{gnu} \code{Make} manual.\\
The \textsc{gnu} manual page for \unixexe{gmake}.\\
\aipspp\ variable names (\sref{variables}).
\exeref{gmake}, \textsc{gnu} make.\\
\filref{makedefs}, \aipspp\ makefile definitions.\\

% ----------------------------------------------------------------------------
 
\newpage
\section{\file{testdefs}}
\label{testdefs}
\index{testdefs -- makefile verification}
\index{makedefs -- verification}
\index{system generation -- testdefs}
\index{makefile -- verification}
 
Validate installation-specific definitions used by the \aipspp\ makefiles.
