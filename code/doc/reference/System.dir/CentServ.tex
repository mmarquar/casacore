\chapter{Central services}
\label{Central services}

\index{central services}
\index{services!central}
\index{administration}
\index{master host}

This chapter \footnote{Last change:
$ $Id$ $}
describes centrally provided \aipspp\ services.  These are associated with
the \acct{aips2adm} account on the master host, \host{aips2.nrao.edu} in
Socorro.  The relevant utilities are installed in \code{\$(MSTRETCD)} on the
master.

% ----------------------------------------------------------------------------
 
\section{AIPS++ web services}
\label{web services}

\index{web!master services}
\index{master host}

This section describes web services provided by the \aipspp\ master host.

\subsection*{Master services}

The \aipspp\ master host, \host{aips2.nrao.edu}, is set up as the web server
for \aipspp\ at NRAO Socorro.  The master web server directly provides a
number of services which may only be linked to by other \aipspp\ web servers:

\begin{itemize}
\item
   The master \rcs\ repository.

\item
   The archive \rcs\ repository.

\item
   The \aipspp\ anonymous \unixexe{ftp} areas.

\item
   The \aipspp\ email exploder archives.

\item
   Separately released \aipspp\ software components:
   \begin{itemize}
   \item
       \aipsexe{cxx2html}.
   \item
       FITS classes.
   \item
       Glish.
   \end{itemize}
\end{itemize}

\noindent
The \aipspp\ master home page is at
\htmladdnormallink{http://aips2.nrao.edu}{http://aips2.nrao.edu}.

\subsection*{Server configuration}
\index{web!master server configuration}
\index{web!robot exclusion}

\host{aips2.nrao.edu} runs NCSA's \unixexe{httpd}.  The essential features of
the server configuration are as follows:

\begin{itemize}
\item
   The server is installed in \file{/home/tarzan/httpd/...}.

\item
   The \unixexe{httpd} \code{DocumentRoot} directory is \file{/home/tarzan}.

\item
   The \unixexe{httpd} \code{ScriptAlias} directory is
   \file{/home/tarzan/httpd/cgi-bin}.

\item
   \file{/home/tarzan/robots.txt} contains entries which exclude web crawlers
   from certain parts of the document hierarchy, notably the email archives.
\end{itemize}

\noindent
The \aipspp\ links are enabled via unix symlinks in \code{DocumentRoot}:

\begin{verbatim}
   aips++/archive  -> /export/aips++/archive/
   aips++/ftp      -> /export/aips++/pub/
   aips++/mail     -> /export/aips++/Mail/
   aips++/master   -> /export/aips++/master/
   aips++/RELEASED -> ftp/RELEASED/
\end{verbatim}

\noindent
plus there are also links to the local \aipspp\ installations:

\begin{verbatim}
   aips++/daily  -> /aips++/daily/
   aips++/weekly -> /aips++/weekly/

   aips++/code   -> daily/code/
   aips++/docs   -> daily/docs/
\end{verbatim}

\noindent
The \aipspp\ links are presented in a \textsc{html} document
\file{/home/tarzan/index.html}.

\subsection*{See also}
 
Section \sref{email exploders}, \aipspp\ email exploders.\\
\href{http://hoohoo.ncsa.uiuc.edu}{, }{NCSA HTTPd}, etc.\\
\href{http://info.webcrawler.com/mak/projects/robots/robots.html}{, }
   {Web robots home page}.

% ----------------------------------------------------------------------------

\newpage
\section{AIPS++ email exploders}
\label{email exploders}

\index{email!exploders}
\index{master host}

\index{electronic mail|see{email}}
\index{exploder|see{email}}
\index{reflector|see{exploder}}

This section describes how the \aipspp\ electronic mail exploders are set up.

\subsection*{Exploder lists}
\index{email!exploders!lists}
\index{lists|see{email, exploders}}

The sources of the \aipspp\ exploder lists reside in
\file{\$(AIPSCODE)/admin/system} in files with a \file{.list} suffix, for
example \mbox{\file{aips2-workers.list}}.  \aipspp\ programmers may modify
these lists using \exeref{ao} and \exeref{ai} to check them in and out like
any other source file.  The lists are installed in \code{\$(MSTRETCD)} by the
\code{\$(MSTRETCD)} makefile which is invoked by \exeref{exhale} several times
a day.

\aipspp\ exploder email is addressed to \acct{nrao.edu}, for example
\mbox{\acct{aips2-workers@nrao.edu}}.  NRAO's email system is set up so that
this mail is received by \host{aips2.nrao.edu}, the \aipspp\ master host,
which distributes email to the list by virtue of entries of the following form
in \file{/etc/aliases}

\begin{verbatim}
   aips2-workers: :include:/export/aips++/master/etc/aips2-workers.list
\end{verbatim}

\subsection*{Exploder archives}
\index{email!exploders!archives}

The first entry in every \aipspp\ exploder list is an email alias which is
used to archive a copy of the mail.  These aliases have names of the form
\code{*-log}, where the asterisk corresponds to the exploder name.  For
example, \mbox{\file{aips2-workers.list}} contains
\mbox{\code{aips2-workers-log}}.

The \code{-log} aliases are defined in \file{/etc/aliases} on
\host{aips2.nrao.edu} so that the received mail is piped to a utility,
\exeref{aipsmail}, which does the archiving using the publically available
\textsc{mh} (message handler) system.  For example, the entry for the
\mbox{\acct{aips2-workers}} exploder is

\begin{verbatim}
   aips2-workers-log: "| /export/aips++/master/etc/aipsmail aips2-workers"
\end{verbatim}

\noindent
The \textsc{mh} folders reside in directories of the same name as the exploder
in the \file{Mail} directory beneath \acct{aips2adm}'s home directory.  For
example the \mbox{\acct{aips2-workers}} archive resides in

\begin{verbatim}
   /export/aips++/Mail/aips2-workers
\end{verbatim}

\noindent
\textsc{mh} stores each email in a separate file in this directory and gives
them numerical names corresponding to the sequence number of the message.

Mail which is more than 14 days old is deleted from the
\mbox{\acct{aips2-inhale}} exploder every evening by an \acct{aips2adm}
\unixexe{cron} job running on \host{aips2.nrao.edu}.

\subsection*{Web interface to the exploder archives}

\index{web!email exploders|see{email, web interface}}
\index{email!web interface}

The \textsc{mh} mail folders in which the exploder archives reside are made
publically accessible on the web via a symlink in the \unixexe{httpd}
\code{DocumentRoot} directory (\file{/home/tarzan}, see section
\sref{web services}):

\begin{verbatim}
   /home/tarzan/aips++/mail -> /export/aips++/Mail
\end{verbatim}

\noindent
The \aipspp\ home page contains a link to an \textsc{html} document,
\file{\$(AIPSDOCS)/html/email.html}, which in turn contains links to email
folder indexes generated by \exeref{scanhtml}.  \exe{scanhtml} invokes the
\textsc{mh} \unixexe{scan} command to produce an index and then converts it to
\textsc{html}.  \aipsexe{scanhtml} is invoked by \aipsexe{aipsmail} whenever
it receives mail so the indexes are always current.

Apart from containing links to each item of exploder mail, the indexes also
contain links to the currently installed copies of the exploder lists, and a
link to a form which allows searching of the exploder archive.

The link to the exploder list is implemented via a unix symbolic link called
\file{list} within the mail folder directory.  For example

\begin{verbatim}
   /export/aips++/Mail/aips2-workers/list -> ../../master/etc/aips2-workers.list
\end{verbatim}

Mail searching is implemented via a \textsc{cgi} (Common Gateway Interface)
script, \exeref{pickhtml}, which generates an \textsc{html} form to obtain
options for another \textsc{cgi} script, \exeref{scanpick}.  \aipsexe{scanpick}
invokes the \textsc{mh} \unixexe{pick} command to search the archive folder
and converts the \unixexe{pick} output to an \textsc{html} index.  The
\textsc{cgi} scripts are web-enabled via symlinks in the \unixexe{httpd}
\code{ScriptAlias} directory (\file{/home/tarzan/httpd/cgi-bin}):

\begin{verbatim}
   /home/tarzan/httpd/cgi-bin/pickhtml -> /export/aips++/master/etc/pickhtml
   /home/tarzan/httpd/cgi-bin/scanpick -> /export/aips++/master/etc/scanpick
\end{verbatim}

Refer to the \aipspp\ email exploder web page for a list of currently active
exploders.

\subsection*{See also}

The unix manual page for \unixexe{cron}(1).\\
The manual page for \unixexe{mh}(1), the message handler system.\\
The manual page for \unixexe{pick}(1), the \textsc{mh} search command.\\
The manual page for \unixexe{scan}(1), the \textsc{mh} index command.\\
\aipspp\ code management configuration (\sref{RCS directories}).\\
\aipspp\ variable names (\sref{variables}).\\
Section \sref{Accounts and groups}, \aipspp\ accounts and groups.\\
\exeref{ai} \aipspp\ code checkin utility.\\
\exeref{ao} \aipspp\ code checkout utility.\\
\exeref{exhale}, \aipspp\ code export utility.\\
Section \sref{web services}, \aipspp\ master web services.\\
\exeref{aipsmail}, \aipspp\ email exploder archive utility.\\
\exeref{pickhtml}, \aipspp\ \textsc{html} form generator for \exe{scanpick}.\\
\exeref{scanhtml}, \aipspp\ \textsc{html} indexing utility for exploder email.\\
\exeref{scanpick}, \aipspp\ \textsc{html} interface to \unixexe{pick}.\\
\href{http://hoohoo.ncsa.uiuc.edu/cgi}{, }{Common Gateway Interface}.

% ----------------------------------------------------------------------------

\newpage
\section{\exe{aipsmail}}
\label{aipsmail}

\index{aipsmail@\exe{aipsmail}}

\index{email!exploders!archiver|see{\exe{aipsmail}}}

Archive \aipspp\ exploder mail.

\subsection*{Synopsis}

\begin{synopsis}
   \code{| \exe{aipsmail} [folder]}
\end{synopsis}

\subsection*{Description}

\exe{aipsmail} archives \aipspp\ electronic mail.  It is invoked directly by
\unixexe{sendmail} on \host{aips2.nrao.edu} via entries in the
\file{/etc/aliases} file as described in \sref{email exploders}.

\exe{aipsmail} expects to receive the text of a mail message on \file{stdin}.
It first invokes \unixexe{rcvstore} to store this message in the specified
\textsc{mh} mail folder -- \code{general} by default.  It then invokes
\exeref{scanhtml} to update the \textsc{html} index for the folder.

\subsection*{Options}

None.

\subsection*{Notes}

\begin{itemize}
\item
   Since \unixexe{sendmail} normally invokes commands using the identity of
   the person who sent the mail, \exe{aipsmail} needs to run as setuid to
   \acct{aips2adm}.
\end{itemize}

\subsection*{Diagnostics}

Status return values
\\ \verb+   0+: success

\subsection*{See also}

The manual page for \unixexe{mh}(1), the message handler system.\\
The manual page for \unixexe{rcvstore}(1), the \textsc{mh} command to store
   mail.\\
Section \sref{Accounts and groups}, \aipspp\ accounts and groups.\\
Section \sref{email exploders}, \aipspp\ email exploders.

\subsection*{Author}

Original: 1995/07/18 by Mark Calabretta, ATNF

% ----------------------------------------------------------------------------

\newpage
\section{\exe{asco}}
\label{asco}

\index{asco@\exe{asco}}

\index{code!management!checkouts|see{\exe{asco}}}

Report \aipspp\ sources currently checked out.

\subsection*{Synopsis}
 
\begin{synopsis}
   \code{\exe{asco}}
\end{synopsis}
 
\subsection*{Description}
 
\exe{asco} reports \aipspp\ sources currently checked out.  A general summary
is posted to the \mbox{\acct{aips2-workers}} exploder, and personal summaries
are sent to each user who has a file locked.

\exe{asco} is invoked as an \acct{aips2adm} \unixexe{cron} job on
\host{aips2.nrao.edu} each Sunday evening.
 
\subsection*{Options}
 
None.
 
\subsection*{Notes}
 
\begin{itemize}
\item
   \exe{asco} uses \code{alog -m -L -t}.
\end{itemize}
 
\subsection*{Diagnostics}
 
Status return values
\\ \verb+   0+: success
\\ \verb+   1+: initialization error
 
\subsection*{See also}
 
The unix manual page for \unixexe{cron}(1).\\
Section \sref{Accounts and groups}, \aipspp\ accounts and groups.\\
\exeref{ai}, \aipspp\ code checkin utility.\\
\exeref{ao}, \aipspp\ code checkout utility.\\
\exeref{alog}, \aipspp\ change log reporting utility.
 
\subsection*{Author}
 
Original: 1995/03/20 by Mark Calabretta, ATNF

% ----------------------------------------------------------------------------

\newpage
\section{\exe{astat}}
\label{astat}

\index{astat@\exe{astat}}
\index{master host}

\index{code!management!revision statistics|see{\exe{astat}}}

Collate \aipspp\ revision statistics from an \exe{alog} report.

\subsection*{Synopsis}
 
\begin{synopsis}
   \code{\exe{alog} [options] | \exe{astat}}
\end{synopsis}
 
\subsection*{Description}
 
\exe{astat} is a \unixexe{sed}/\unixexe{awk} filter which collates revision
statistics for each \aipspp\ programmer from an \exeref{alog} report.  It
reports separately the number of new files checked in, revisions of previously
checked in files, and the total of these.

For each revision made (but not for the initial checkin) \rcs\ provides a
rough indication of the extent of the change by reporting the number of lines
added and the number of lines deleted.  Cumulative totals for these are
provided in the output, but note that these numbers can be misleading.  Large
values may, for example, result from checking in a changed \textsc{PostScript}
file, or test program output.
 
\subsection*{Options}
 
None.
 
\subsection*{Notes}
 
\begin{itemize}
\item
   The \aipspp\ master and slave \rcs\ files contain only the current and
   preceeding versions (see \exeref{exhale}).  The archive on the \aipspp\
   master host, \host{aips2.nrao.edu},  contains all versions up to the base
   release of the current major version.

\item
   Account names for some checkins from remote sites are normalized to
   standard form.

\item
   Only a fraction of the efforts of AIPS++ workers is reflected in the
   checkin statistics.  This fraction may be quite low for some but high for
   others.
\end{itemize}

\subsection*{Examples}

\begin{verbatim}
   alog -b --package=aips | astat
\end{verbatim}

\noindent
Collects statistics for the \code{aips} package for the latest \aipspp\ major
version.
 
\subsection*{See also}
 
\exeref{alog}, \aipspp\ change log reporting utility.
 
\subsection*{Author}
 
Original: 1994/02/10 by Mark Calabretta, ATNF

% ----------------------------------------------------------------------------

\newpage
\section{\exe{parseform}}
\label{parseform}

\index{parseform@\exe{parseform}}

\index{html@\textsc{html}!forms output decoder|see{\exe{parseform}}}

Decode \textsc{html} forms output.

\subsection*{Synopsis}
 
\begin{synopsis}
   \code{\exe{eval `parseform`}}
\end{synopsis}
 
\subsection*{Description}
 
\exe{parseform} decodes the output from an \textsc{html} form reporting it on
\file{stdout} in a format suitable for \unixexe{eval}'ing into the environment
in a Bourne shell \textsc{cgi} script.  Both the \code{GET} and \code{POST}
methods of \textsc{html} form output are supported.
 
\subsection*{Options}
 
None.
 
\subsection*{Notes}
 
\begin{itemize}
\item
   \exe{parseform} only works when called from a \textsc{cgi} script.  Note
   that the form output generated by the \code{POST} method appears on
   \file{stdin} and \exe{parseform} consumes this.  Form output generated by
   the \code{GET} method is obtained from environment variable
   \code{QUERY\_STRING}.

\item
   \exe{parseform} is implemented as a \unixexe{perl} script.
\end{itemize}
 
\subsection*{Diagnostics}
 
Status return values
\\ \verb+   0+: success

\subsection*{Examples}

\exe{scanpick} uses the following to get and parse the form output generated
by \exe{pickhtml}:

\begin{verbatim}
   # Get the form output.
     eval `$MSTRETCD/parseform`
\end{verbatim}
 
\subsection*{See also}
 
Section \sref{email exploders}, \aipspp\ email exploders.\\
\exeref{pickhtml}, \aipspp\ \textsc{html} form generator for \exe{scanpick}.\\
\exeref{scanpick}, \aipspp\ \textsc{html} interface to \unixexe{pick}.\\
\href{http://hoohoo.ncsa.uiuc.edu/cgi/forms.html}{, }
   {Decoding FORMs with CGI}.
 
\subsection*{Author}
 
Original: 1996/05/21 by Mark Calabretta, ATNF

% ----------------------------------------------------------------------------

\newpage
\section{\exe{pickhtml}}
\label{pickhtml}

\index{pickhtml@\exe{pickhtml}}

\index{email!web interface!search form|see{\exe{pickhtml}}}

\textsc{html} forms based interface to search an \aipspp\ email archive.

\subsection*{Synopsis}

\begin{synopsis}
   \code{\exe{pickhtml} [folder]}
\end{synopsis}

\subsection*{Description}

\exe{pickhtml} is a \textsc{cgi} script which generates an \textsc{html} form
to obtain options for the \textsc{mh} \unixexe{pick} command to search an
\aipspp\ email archive folder.  These options are passed on to
\exeref{scanpick} via the \code{POST} method.

\exeref{scanhtml} creates a link to \exe{pickhtml} in the \textsc{html} index
to each \aipspp\ email archive folder.

If not supplied the folder name defaults to \code{general}.

\subsection*{Options}

None.

\subsection*{Notes}

\begin{itemize}
\item
   \exe{pickhtml} is web-enabled via a symlink in
   \file{/home/tarzan/httpd/cgi-bin}, the \unixexe{httpd} \code{ScriptAlias}
   directory:

\begin{verbatim}
 /home/tarzan/httpd/cgi-bin/pickhtml -> /export/aips++/master/etc/pickhtml
\end{verbatim}
\end{itemize}

\subsection*{Diagnostics}

Status return values
\\ \verb+   0+: success

\subsection*{See also}

The manual page for \unixexe{mh}(1), the message handler system.\\
The manual page for \unixexe{pick}(1), the \textsc{mh} search command.\\
Section \sref{email exploders}, \aipspp\ email exploders.\\
\exeref{scanpick}, \aipspp\ \textsc{html} interface to \unixexe{pick}.

\subsection*{Author}

Original: 1996/05/21 by Mark Calabretta, ATNF

% ----------------------------------------------------------------------------

\newpage
\section{\exe{reap}}
\label{reap}

\index{reap@\exe{reap}}

\index{reports, programmer!collate and disseminate|see{\exe{reap}}}

Collate and disseminate \aipspp\ weekly reports.

\subsection*{Synopsis}
 
\begin{synopsis}
   \code{\exe{reap}}
\end{synopsis}
 
\subsection*{Description}
 
\exe{reap} collates and disseminates \aipspp\ reports.  It maintains a
timestamp file, \file{.reap.time}, within the \mbox{\acct{aips2-reports}} mail
folder, collects all reports newer than the timestamp, strips out the mail
headers, concatenates them and posts the result to the
\mbox{\acct{aips2-weekly-reports}} exploder.  It is invoked regularly by an
\acct{aips2adm} \unixexe{cron} job running on \host{aips2.nrao.edu}.

\subsection*{Options}
 
None.

\subsection*{Diagnostics}
 
Status return values
\\ \verb+   0+: success
\\ \verb+   1+: initialization error

\subsection*{See also}

The unix manual page for \unixexe{cron}(1).\\
Section \sref{Accounts and groups}, \aipspp\ accounts and groups.\\
Section \sref{email exploders}, \aipspp\ email exploders.\\
\file{report\_form}, \aipspp\ weekly report form.

\subsection*{Author}
 
Original: 1996/05/05 by Mark Calabretta, ATNF

% ----------------------------------------------------------------------------

\newpage
\section{\file{report\_form}}
\label{report_form}

\index{report\_form@\file{report\_form}}

\index{reports, programmer!form|see{\file{report\_form}}}

\aipspp\ weekly report form.

\subsection*{Synopsis}
 
\begin{synopsis}
   \code{\exe{/usr/lib/sendmail} -t < /export/aips++/master/etc/report\_form}
\end{synopsis}
 
\subsection*{Description}

\file{report\_form} is sent to the \mbox{\acct{aips2-weekly-reports}} email
exploder each week by an \acct{aips2adm} \unixexe{cron} job running on
\host{aips2.nrao.edu}.  It solicits a report which is to be sent to the
\mbox{\acct{aips2-reports}} exploder where they are collected by \exeref{reap}
which concatenates and mails them back to \mbox{\acct{aips2-weekly-reports}}.

\subsection*{Notes}
 
\begin{itemize}
\item
   The \file{report\_form} is fed directly to \unixexe{sendmail}.
\end{itemize}

\subsection*{See also}
 
The unix manual page for \unixexe{cron}(1).\\
Section \sref{Accounts and groups}, \aipspp\ accounts and groups.\\
Section \sref{email exploders}, \aipspp\ email exploders.\\
\exeref{reap}, \aipspp\ report collator.
 
\subsection*{Author}
 
Original: Jim Horstkotte, NRAO
 
% ----------------------------------------------------------------------------

\newpage
\section{\exe{scanhtml}}
\label{scanhtml}

\index{scanhtml@\exe{scanhtml}}

\index{email!web interface!index|see{\exe{scanhtml}}}

Produce an \textsc{html} index of an \aipspp\ mail folder.

\subsection*{Synopsis}

\begin{synopsis}
   \code{\exe{scanhtml} [folder]}
\end{synopsis}

\subsection*{Description}

\exe{scanhtml} produces an \textsc{mh} \unixexe{scan} listing of an \aipspp\
exploder mail archive folder and converts it to an \textsc{html} index with
links to the individual messages.  The index, \file{index.html}, is deposited
in the folder directory.  Any pre-existing index is overwritten.

As described in \sref{email exploders} the index generated by \exe{scanhtml}
also contains links to the currently installed copies of the exploder lists,
and a link to a form which allows searching of the exploder archive.

If not supplied the folder name defaults to \code{general}.

\subsection*{Options}

None.

\subsection*{Notes}

\begin{itemize}
\item
   \exe{scanhtml} is invoked by \exeref{aipsmail} immediately after new
   exploder mail is archived.  It is also used by the \acct{aips2adm}
   \unixexe{cron} job which deletes old mail from the
   \mbox{\acct{aips2-inhale}} archive.  It can be invoked manually (by
   \acct{aips2adm}) at any time if required.
\end{itemize}

\subsection*{Diagnostics}

Status return values
\\ \verb+   0+: success

\subsection*{See also}

The unix manual page for \unixexe{cron}(1).\\
The manual page for \unixexe{mh}(1), the message handler system.\\
The manual page for \unixexe{scan}(1), the \textsc{mh} index command.\\
Section \sref{Accounts and groups}, \aipspp\ accounts and groups.\\
Section \sref{email exploders}, \aipspp\ email exploders.

\subsection*{Author}

Original: 1995/07/18 by Mark Calabretta, ATNF

% ----------------------------------------------------------------------------

\newpage
\section{\exe{scanpick}}
\label{scanpick}

\index{scanpick@\exe{scanpick}}

\index{email!web interface!search results|see{\exe{scanpick}}}

Produce an \textsc{html} index of selected mail in an \aipspp\ folder.

\subsection*{Synopsis}

\begin{synopsis}
   \code{\exe{scanpick} [folder]}
\end{synopsis}

\subsection*{Description}

\exe{scanpick} is a \textsc{cgi} script which produces a \unixexe{scan}
listing of messages in an \aipspp\ \textsc{mh} email archive folder selected
according to \unixexe{pick} arguments.

The \unixexe{pick} arguments are acquired by another \textsc{cgi} script,
\exeref{pickhtml}, which generates an \textsc{html} form and passes the result
to \exe{scanpick} via the \code{POST} method.

\exe{scanpick} uses an auxiliary \unixexe{perl} script called
\exeref{parseform} to decode the form output.

\subsection*{Options}

None.

\subsection*{Notes}

\begin{itemize}
\item
   \exe{scanpick} is web-enabled via a symlink in
   \file{/home/tarzan/httpd/cgi-bin}, the \unixexe{httpd} \code{ScriptAlias}
   directory:

\begin{verbatim}
 /home/tarzan/httpd/cgi-bin/scanpick -> /export/aips++/master/etc/scanpick
\end{verbatim}
\end{itemize}

\subsection*{Diagnostics}

Status return values
\\ \verb+   0+: success

\subsection*{See also}

The manual page for \unixexe{mh}(1), the message handler system.\\
The manual page for \unixexe{pick}(1), the \textsc{mh} search command.\\
The manual page for \unixexe{scan}(1), the \textsc{mh} index command.\\
Section \sref{email exploders}, \aipspp\ email exploders.\\
\exeref{pickhtml}, \aipspp\ \textsc{html} form generator for \exe{scanpick}.\\
\exeref{parseform}, \aipspp\ decoder for \textsc{html} forms output.

\subsection*{Author}

Original: 1996/05/21 by Mark Calabretta, ATNF
