\chapter{Installation}
\label{Installation}
\index{system!installation}
\index{installation|see{system, installation}}

This chapter \footnote{Last change:
$ $Id$ $}
describes how to install \aipspp.

The installation of end-user (production-line) and \aipspp\ consortium
(code-development) systems are sufficiently different that separate
instructions are provided for each.

If you contemplate installing \aipspp\ for the Linux operating system using
the \gnu\ ``binutils'' package (e.g. \exe{ar}, \exe{ld}), then you should be
aware that there is a bug in handling large archive (\file{.a}) files in
versions 2.7 (and earlier).  You can find out what version of binutils you
have by running \exe{ar -V}.  If it is the incorrect version, either upgrade
to version 2.8 (or later) or contact \acct{aips2-request@nrao.edu} for a patch
(all that is required is to change the \code{\#if 1} that occurs at about line
823 of \file{bfd/archive.c} to \code{\#if 0}).

% ----------------------------------------------------------------------------

\section{External Developer's Release}
\label{developers-release}
\index{developers release}
\index{installation|see{system, installation}}

This section describes how to create an installation for external developers,
i.e. developers who are not \aipspp\ onsortium members. This sort
of \aipspp\ installation differs from the regular consortium
installation in that:
\begin{enumerate}
\item It is intended as a stand-alone installation which allows developers
      to keep the installation in sync with the master repositories, but
      does not allow checking code into the system.
\item It is easy to configure.
\item It is made available for the most common architectures. Though currently
      it is {\bf only} available for linux.
\end{enumerate}
The developers installation operates by using
\htmladdnormallinkfoot{CVSup}{http://www.polstra.com/projects/freeware/CVSup/} to
both bootstrap the installation and keep the documentation, RCS source code
repository, and data repository in sync with the master repositories at the
AIPS++ center.

\subsection{Getting Started}
\label{developer release setting up}
\index{developer release!setup}

The first step to installing the \aipspp\ external developer's release is to
install the required software upon which \aipspp\ depends. This includes
\htmladdnormallinkfoot{CVSup}{http://www.polstra.com/projects/freeware/CVSup/},
\htmladdnormallinkfoot{RPFITS}{http://www.atnf.csiro.au/computing/software/rpfits.html},
\htmladdnormallinkfoot{CFITSIO}{http://heasarc.gsfc.nasa.gov/docs/software/fitsio/fitsio.html}, and
\htmladdnormallinkfoot{PGPLOT}{http://astro.caltech.edu/\~{}tjp/pgplot/}. AIPS++
\htmladdnormallinkfoot{distributes RPMs}{ftp://aips2.nrao.edu/pub/import/Linux/RedHat/}
RPMs built upon various releases of RedHat Linux which greatly simplify the installation
of these packages for Linux systems which use RPM. The RPMs are built for specific
RedHat releases and must be installed as {\em root}. If you're using RPM, you need to
install:
\begin{itemize}
\item \verb+cvsup-16.1e-2ds.i386.rpm+
\item \verb+rpfits-2.4-1ds.i386.rpm+
\item \verb+cfitsio-2.401-1ds.i386.rpm+
\item \verb+pgplot-5.2.0-9ds.i386.rpm+
\item \verb+pgplot-devel-5.2.0-9ds.i386.rpm+
\item \verb+pgplot-motif-5.2.0-9ds.i386.rpm+
\end{itemize}
In addition to these, you need to ensure that a few packages which are normally
part of a Linux installation are installed:
\begin{itemize}
\item \htmladdnormallinkfoot{GNU make}{http://www.gnu.org/software/make/}
\item either \htmladdnormallinkfoot{Open MOTIF}{http://www.opengroup.org/openmotif/} or
      \htmladdnormallinkfoot{LessTif}{http://www.lesstif.org/} ({\em or just about any
      other Motif you might have installed})
\item \htmladdnormallinkfoot{LAPACK}{http://www.netlib.org/lapack/} and BLAS
\item all of the standard header files, e.g. \verb+glibc-devel+
\item all of the X11 header files, e.g. \verb+XFree86-devel+
\item \verb+tcl+ and \verb+tk+
\item \htmladdnormallinkfoot{bison}{http://www.gnu.org/software/bison/} and
      \htmladdnormallinkfoot{flex}{http://www.gnu.org/software/flex/}
\end{itemize}
A good way to check is to do something like:
\begin{verbatim}
    rpm -qa | egrep '^openmotif|^lesstif|^make|^glibc|^XFree|^lapack|^blas|^tcl|^tk|^bison|^flex'
\end{verbatim}
If the output of this seems to list all requirements, then you are in good shape.

Finally, you need to have the gcc 2.95 variety of GNU gcc compiler installed. This is
necessary because we have found precision problems with g77 in the 2.96 release, and
we do not yet support the 3.1 release. To simplify this, RPMs for \verb+gcc-2.95.3+
are also available from the
\htmladdnormallinkfoot{\aipspp\ ftp site}{ftp://aips2.nrao.edu/pub/import/Linux/RedHat/}.
You need:
\begin{itemize}
\item {\tt gcc+2-2.95.3-52ds.i386.rpm}
\item {\tt gpp+2-2.95.3-52ds.i386.rpm}
\item {\tt libgpp+2-2.95.3-52ds.i386.rpm}
\item {\tt gppshare+2-2.95.3-52ds.i386.rpm}
\item {\tt g77+2-2.95.3-52ds.i386.rpm}
\end{itemize}
{\bf These RPMs install into \verb+/usr/gcc/2.95.3+, e.g. \verb+/usr/gcc/2.95.3/bin/gcc+}, and currently
they are not relocatable. Otherwise, you can get gcc 2.95 from the
\htmladdnormallinkfoot{GNU ftp site}{ftp://ftp.gnu.org/pub/gnu/gcc/gcc-2.95.3.tar.gz} and
install it by hand.

The next step is to select a directory where you would like your \aipspp\ installation
to reside. For the examples in this section, {\tt /usr/local/aips++} will be
used; if you choose a differnt location, just substitute your path for
{\tt /usr/local/aips++}.  So first create the directory for the initial
CVSup bootstrap file, for example:
\begin{verbatim}
    bash$ mkdir -p /usr/local/aips++/sup/files
\end{verbatim}
With an editor, create a file in this directory named for the architecture
being installed, e.g. {\tt /usr/local/aips++/sup/files/linux}. In this file, put:
\begin{verbatim}
    bootstrap release=linux host=aips2.nrao.edu base=/usr/local/aips++
\end{verbatim}
If you chose a root directory other than {\tt /usr/local/aips++} for your
installation, you should substitute your directory in the line above. This
file specifies which set of bootstrap files should be fetched.

The next step is to use the bootstrap specification file you just created to
fetch the bootstrap files:
\begin{verbatim}
    bash$ cd /usr/local/aips++
    bash$ cvsup sup/files/linux &
\end{verbatim}
A GUI should pop up, when it does you should only need to press the {\em go}
button, i.e. the button with the little {\bf green triangle} beside the 
{\em exit} button. This will fetch all of the \aipspp\ bootstrap files for the
given architecture..

The next step is to run the bootstrap \verb+setup+ script. This script not only
takes care of configuration details, but it also fetches the \aipspp\ source code:
\begin{verbatim}
    bash$ cd /usr/local/aips++
    bash$ linux/setup
\end{verbatim}
This script will first verify the \aipspp\ root directory, and then it will
commence with downloading the source code. This may take some time, so
patience, as well as bandwidth, is required. After the initial installation
of the source code, however, subsequent updates should only download source
code modifications, {\bf not} all of the source code.

Lastly, all you need to do is build \aipspp:
\begin{verbatim}
    bash$ cd /usr/local/aips++/code
    bash$ gmake allsys
\end{verbatim}


\subsection{Updating the Source Code}
\label{developer release source code}
\index{developer release!source code}

If you later wish to bring your source code up to date with the current version
available from the master repository at the \aipspp\ center, you can do this
easily with CVSup and the ontrol scripts which the setup step generated:
\begin{verbatim}
    bash$ cd /usr/local/aips++
    bash$ cvsup sup/files/code
    bash$ cd code
    bash$ gmake allsys
\end{verbatim}
This procedure can be repeated whenever you wish, and subsequent updates will
only download those portions of the code repository which have been modified
or added. It {\bf will not} download everything again.

\subsection{Updating the Data Repository}
\label{developer release data repository}
\index{developer release!data repository}

In addition to a successful build of the surce code, the \aipspp\ data repository
is required to run \aipspp. This repository contains all of the data files needed
to run \aipspp\ as well as some data files used for testing and demos.

To update the data repository, you use CVSup:
\begin{verbatim}
    bash$ cd /usr/local/aips++
    bash$ cvsup sup/files/data
\end{verbatim}
This will create a local mirror of the \aipspp\ data repository. This can be
redone later to update the repository. Subsequent updates will only download
those portions of the data repository which have been modified or added. It
{\bf will not} download everything again.

\subsection{Updating the Documentation }
\label{developer release documentation}
\index{developer release!documentation}

If you would like to have a local copy of the documentation (which is recommended),
you use CVSup to get the documentation as well:
\begin{verbatim}
    bash$ cd /usr/local/aips++
    bash$ cvsup sup/files/docs
\end{verbatim}
This will create a local mirror of the \aipspp\ documentation and can be
used to keep the documentation up to date, as with the code and data
repositories.

\subsection{Final Notes}

CVSup provides an efficient method for maintaining synchroized mirrors, and
the \aipspp\ external developer's release makes heavy use of it. All of the
control files for CVSup are kept beneath \verb+sup/files+. These files are
simple and reasonably straight forward to understand if a bit of time is
spent with the
\htmladdnormallinkfoot{CVSup documentation}{http://www.polstra.com/projects/freeware/CVSup/faq.html}.
In addition to the control files mentioned above, the initial setup script
also creates a \verb+sup/files/update+ script. This CVSup control script
can be used to update the code and data repositories as well as the documentation
with one CVSup invocation. After that, all that remains is the \verb+gmake allsys+
to build \aipspp.

% ----------------------------------------------------------------------------

\section{End-user AIPS++ installations}
\label{End-user installation}
\index{system!installation!end-user}
\index{system!installation!production-line}

A set of linux RPMS is available from 
\htmladdnormallink{ftp://ftp.cv.nrao.edu/pub/aips++}{ftp://ftp.cv.nrao.edu/pub/aips++}.
Please be aware that we are providing only minimal user support at this time (February 2004).

% ----------------------------------------------------------------------------

\section{Consortium AIPS++ installations}
\label{Consortium installation}
\index{system!installation!consortium}
\index{system!installation!code-development}
\index{code!configuration}
\index{code!management!configuration}

Consortium installations differ from end-user installations in having a local
copy of the master \cvs\ source code repositories. 
The local code tree is updated regularly by the \aipspp\ code
distribution system via a procedure called \exeref{inhale}.  Consortium
installations also have a mechanism for checking sources out of, and in to,
the master \cvs\ repositories.

\subsection*{Step 1. Create AIPS++ accounts}

Before starting a consortium installation the following accounts must be
created by the unix system administrator with the specified user and group ids
in \file{/etc/passwd} (see \sref{Accounts and groups}):

\begin{verbatim}
   aips2mgr   uid=31415   gid=31415
   aips2prg   uid=31416   gid=31416
   aips2usr   uid=31417   gid=31417
\end{verbatim}

\noindent
The user and group ids correspond with those of the master sources and
although different ids may be used it is {\em highly desirable} that they
match.  The home directory for the accounts should be set to the root
directory of the \aipspp\ tree.  This can be anything but \file{/aips++} is
preferred and is assumed in the following examples.

The following groups must be created with the matching group id and membership
(in \file{/etc/group}):

\begin{verbatim}
   aips2mgr   gid=31415
   aips2prg   gid=31416   aips2mgr
   aips2usr   gid=31417   aips2mgr aips2prg
\end{verbatim}

\noindent
You should also add your account name and the names of any other local
\aipspp\ managers to the \acct{aips2mgr} group membership list.  Do not add
everyone to the \acct{aips2mgr} group, it grants permission to directly
manipulate the \rcs\ sources.  You should also add the names of all local
\aipspp\ programmers to the \acct{aips2prg} group.  This will allow them to
check out and modify the \aipspp\ sources.  The \acct{aips2mgr} account and the
\acct{aips2mgr} and \acct{aips2prg} groups will be used during the installation.

Now create the \aipspp\ root directory:

\begin{verbatim}
   yourhost% mkdir /aips++
   yourhost% chown aips2mgr /aips++
   yourhost% chgrp aips2prg /aips++
   yourhost% chmod ug=rwx,o=rx,g+s /aips++
\end{verbatim}

\noindent
In practice \file{/aips++} will often be a self-contained filesystem, usually
on a separate disk.  Allow 1\,Gbyte of disk space for the \aipspp\ system; any
short-term surplus may be used for programmer workspaces.

This is as much as needs doing by the system administrator at this stage.  The
remainder of the initial part of the installation can be done by
\acct{aips2mgr}.

\subsection*{Step 2. Fetch the \file{install} scripts}

First fetch the source tree via anonymous CVS:

\begin{verbatim}
   yourhost% cd /aips++
   yourhost% cvs -r -d :pserver:anonymous@cvs.cv.nrao.edu:/home/cvs checkout -d code aips++/code
\end{verbatim}

\noindent
This will fetch the main development branch of AIPS++. After the checkout completes you will
need to configure the source installation

\begin{verbatim}
   yourhost% chmod 544 install/configure
   yourhost% install/configure
\end{verbatim}

\noindent
You will be asked a series of questions, most of which have sensible defaults,
aimed at constructing the \aipspp\ \filref{aipshosts} file.  You then have to
edit your site-specific \filref{aipsrc} and \filref{makedefs} files.  Template
versions of these files are supplied by \exeref{configure} and you should read
the instructions carefully.  After making your site-specific definitions
\aipsexe{configure} will run some tests to check whether your \file{makedefs}
definitions look sensible.  Your \file{install} directory will then be made.
This consists of a few \textsc{c} compilations and installation of some shell
scripts, the most important of which is \exeref{inhale} itself.  Ignore any
error message from \exeref{gmake} concerning the non-existence of various
subdirectories of \file{/aips++/code}.

\subsection*{Step 3. Run \aipsexe{sneeze}}

At this point your \aipspp\ installation has been bootstrapped to a state
where \exeref{sneeze} can be run.

You should also have a \cplusplus\ compiler, and a \TeX\ installation which
includes \LaTeX, \unixexe{dvips}, \textsc{MetaFont} and \textsc{latex2html}.
Unset the \code{DOCSYS} variable in \file{makedefs} if you
don't have \TeX, it will prevent compilation of the \aipspp\ documentation.
The documentation may be downloaded using the \aipsexec{aupdate} command.

Users of SysV based systems such as Solaris should be warned that
\exeref{inhale} requires the BSD version of \unixexe{sum} for computing
checksums.  You must ensure that the BSD version will be found ahead of the
SysV version in \acct{aips2mgr}'s \code{PATH}.  The \gnu\ version of
\unixexe{sum} (in the \gnu\ ``fileutils'' kit) provides both algorithms and
uses BSD by default.  Less salubrious possibilities are to put \file{/usr/ucb}
(Solaris) or \file{/usr/bsd} (IRIX) ahead of \file{/usr/bin} in
\acct{aips2mgr}'s \code{PATH}, or to create a symlink to the BSD version of
\unixexe{sum} in the \aipspp\ \file{bin} area.  

First invoke \exeref{aipsinit} to add the \aipspp\ \file{bin} directory to
your \code{PATH}.  If your interactive shell is a C-like shell (\unixexe{csh},
\unixexe{tcsh}) you would use

\begin{verbatim}
   yourhost% source /aips++/aipsinit.csh
\end{verbatim}

\noindent
whereas for Bourne-like shells (\unixexe{sh}, \unixexe{bash}, \unixexe{ksh})
you would use

\begin{verbatim}
   yourhost% . /aips++/aipsinit.sh
\end{verbatim}

\noindent
If you use some other shell you'll have to revert to one of the above for the
remainder of the installation.  Now invoke \exeref{sneeze}

\begin{verbatim}
   yourhost% sneeze -l -m cumulative&
\end{verbatim}

\noindent
This will build and install
the latest version of the sources which are under active development.  If you
made any mistakes in your \file{aipsrc} or \file{makedefs} definitions some of
these may become apparent during the installations.  After fixing them you can
recover via

\begin{verbatim}
   yourhost% gmake -C /aips++/code allsys
\end{verbatim}

\noindent
The \code{allsys} target will compile all \aipspp\ sources, including
documentation (assuming of course that you have the compilers).  If you just
wanted to compile the documentation alone you could use

\begin{verbatim}
   yourhost% gmake -C /aips++/code docsys
\end{verbatim}

If everything has gone properly you should now have an up-to-date \aipspp\ 
installation.  However, in order to keep it up-to-date you must define a
\unixexe{cron} job to run \exeref{inhale} on a regular basis.  The normal
procedure is to do a cumulative update every Saturday evening.  However, you
may wish to maintain a (possibly separate) system which is updated on a daily
basis.  Half-daily updates are also possible, but note that 12 hours may not
be sufficient time to rebuild the system.

The exact timing depends on your timezone with respect to the master.  New
updates are produced by 0700 and 1900 Socorro time (MST or MDT) but you should
allow at least an hour's grace before collecting them.  An example
\file{crontab} file might resemble the following:

\begin{verbatim}
   # Cumulative update of the AIPS++ directories each Saturday evening.
   00 22 * * 6   (. $HOME/.profile ; inhale.cvs -c) 2>&1 | \
      mail aips2mgr aips2-inhale@nrao.edu
\end{verbatim}

\noindent
(Note that all \unixexe{cron} entries must be one-liners but they are broken
here for clarity.)  You may need to add the \exe{-n} option to \exeref{inhale}
accordingly.  Note that, as in the above example, the log produced by
\aipsexe{inhale} is generally forwarded to \acct{aips2-inhale@nrao.edu}.
These logs are archived for about 10 days and are accessible via the \aipspp\ 
home page \url{http://aips2.nrao.edu/aips++/docs/html/aips++.html}.  This is
particularly useful for verifying code portability, especially on platforms
that a programmer doesn't have ready access to.  You should also add the email
address of a local person who will monitor the \aipsexe{inhale} logs
(\acct{aips2mgr} in the above example).

\subsection*{Step 4. CVS access to the development tree}
\index{nfs@\textsc{nfs}}
\index{automount, \textsc{nfs}}

For active developers, you will need a CVS account on the main CVS repository server.
Please send an encyrted password to aips2mgr@aoc.nrao.edu. For more information about CVS
visit the webpage at \htmladdnormallink{http://www.cvshome.org}{http://www.cvshome.org}.
