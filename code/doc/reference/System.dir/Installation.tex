\chapter{Installation}
\label{Installation}

\index{system!installation}

\index{installation|see{system, installation}}

This chapter \footnote{Last change:
$ $Id$ $}
describes how to install \aipspp.

The installation of end-user (production-line) and \aipspp\ consortium
(code-development) systems are sufficiently different that separate
instructions are provided for each.

If you contemplate installing \aipspp\ for the Linux operating system using
the \gnu\ ``binutils'' package (e.g. \exe{ar}, \exe{ld}), then you should be
aware that there is a bug in handling large archive (\file{.a}) files in
versions 2.7 and earlier.  You can find out what version of binutils you have
by running \exe{ar -V}.  If it is the incorrect version, either upgrade to a
later version if one has been released (it hasn't as of 1996/12/23) or contact
\acct{aips2-request@nrao.edu} for a patch (all that is required is to change
the \code{\#if 1} that occurs at about line 823 of \file{bfd/archive.c} to
\code{\#if 0}).

% ----------------------------------------------------------------------------

\section{End-user AIPS++ installations}
\label{End-user installation}

\index{system!installation!end-user}
\index{system!installation!production-line}

The first step in installing \aipspp\ is to fetch the compressed \unixexe{tar}
files containing the \aipspp\ sources.  Make sure you're logged in to the
account which will own the \aipspp\ source code.  The home directory of this
account is referred to as \file{/aips++} below although you can put the
\aipspp\ root directory wherever you like.  However, you {\em must} fetch the
distribution files into a subdirectory called \file{code}.

After creating \file{/aips++} you should give the directory suitable user and
group ownership.  These can be anything, but you might wish to set up a
special \aipspp\ manager account and group, the standard name for each of
which is \acct{aips2mgr} (see \sref{Accounts and groups}).

The complete \unixexe{ftp} sequence should look something like

\begin{verbatim}
   yourhost% cd /aips++
   yourhost% chgrp aips2mgr .
   yourhost% chmod ug=rwx,o=rx,g+s .
   yourhost% mkdir code
   yourhost% cd code
   yourhost% ftp aips2.nrao.edu
   Connected to tarzan.aoc.nrao.edu.
   220 tarzan FTP server (Version wu-2.4(5) Fri Jul 14 14:30:10 MDT 1995) ready.
   Name (aips2.nrao.edu:you): ftp
   331 Guest login ok, send your complete e-mail address as password.
   Password:you@yourhost.there
        :
     ftp welcome messages
        :
   230 Guest login ok, access restrictions apply.
   ftp> cd /pub/code
   250 CWD command successful.
   ftp> binary
   200 Type set to I.
   ftp> prompt
   Interactive mode off.
   ftp> hash
   Hash mark printing on (8192 bytes/hash mark).
   ftp> mget *
        :
     lots of messages describing what ftp is fetching
        :
   ftp> quit
   221 Goodbye.
   yourhost%
\end{verbatim}

\noindent
If you only want a subset of the \aipspp\ sources then you need only fetch the
\file{README}, \file{VERSION}, and \file{configure} files, and the
\file{install} and \file{aips} distribution \unixexe{tar} files, plus whatever
else you want.  Don't forget to use the \code{binary} option to \unixexe{ftp}.
The \code{hash} option is not required but may make you feel better about the
fetch operation.

The distribution contains two special files, \file{VERSION} and
\file{configure}.  \file{VERSION} records the version number and date of the
distribution.  The \exeref{configure} utility attempts to extract and install
\aipspp\ in a semi-automatic way.  After fetching the distribution as
described above, you should run \exeref{configure}

\begin{verbatim}
   yourhost% chmod 544 configure
   yourhost% ./configure
\end{verbatim}

\noindent
The first thing \aipsexe{configure} will do is attempt to extract the
compressed \unixexe{tar} files.  It then asks a series of questions aimed at
constructing the \aipspp\ \filref{aipshosts} file.  The procedure should be
self-explanatory, and most of the questions have sensible defaults.

You will then be asked to edit your site-specific \filref{aipsrc} and
\filref{makedefs} files.  Template versions of these files are supplied by
\exeref{configure} and you should read the instructions carefully.  After
making your site-specific definitions \exeref{configure} will run some tests
to check whether your \file{makedefs} definitions look sensible.  It will then
invoke \exeref{gmake} to build the \aipspp\ system for you.  If you made any
mistakes in \file{makedefs} or \file{aipsrc} which were not caught by the
diagnostics tests these may become apparent at this stage.  After fixing them
you can restart the build by

\begin{verbatim}
   yourhost% gmake -C /aips++/code allsys
\end{verbatim}

% ----------------------------------------------------------------------------

\section{Consortium AIPS++ installations}
\label{Consortium installation}

\index{system!installation!consortium}
\index{system!installation!code-development}
\index{code!configuration}
\index{code!management!configuration}

Consortium installations differ from end-user installations in having a local
slave copy of the master \rcs\ source code repositories (see
\sref{RCS directories}).  The slave is updated regularly by the \aipspp\ code
distribution system via a procedure called \exeref{inhale}.  Consortium
installations also have a mechanism for checking sources out of, and in to,
the master \rcs\ repositories.

\subsection*{Step 1. Create AIPS++ accounts}

Before starting a consortium installation the following accounts must be
created by the unix system administrator with the specified user and group ids
in \file{/etc/passwd} (see \sref{Accounts and groups}):

\begin{verbatim}
   aips2mgr   uid=31415   gid=31415
   aips2prg   uid=31416   gid=31416
   aips2usr   uid=31417   gid=31417
\end{verbatim}

\noindent
The user and group ids correspond with those of the master sources and
although different ids may be used it is {\em highly desirable} that they
match.  The home directory for the accounts should be set to the root
directory of the \aipspp\ tree.  This can be anything but \file{/aips++} is
preferred and is assumed in the following examples.

The following groups must be created with the matching group id and membership
(in \file{/etc/group}):

\begin{verbatim}
   aips2mgr   gid=31415
   aips2prg   gid=31416   aips2mgr
   aips2usr   gid=31417   aips2mgr aips2prg
\end{verbatim}

\noindent
You should also add your account name and the names of any other local
\aipspp\ managers to the \acct{aips2mgr} group membership list.  Do not add
everyone to the \acct{aips2mgr} group, it grants permission to directly
manipulate the \rcs\ sources.  You should also add the names of all local
\aipspp\ programmers to the \acct{aips2prg} group.  This will allow them to
check out and modify the \aipspp\ sources.  The \acct{aips2mgr} account and the
\acct{aips2mgr} and \acct{aips2prg} groups will be used during the installation.

Now create the \aipspp\ root directory:

\begin{verbatim}
   yourhost% mkdir /aips++
   yourhost% chown aips2mgr /aips++
   yourhost% chgrp aips2prg /aips++
   yourhost% chmod ug=rwx,o=rx,g+s /aips++
\end{verbatim}

\noindent
In practice \file{/aips++} will often be a self-contained filesystem, usually
on a separate disk.  Allow 1\,Gbyte of disk space for the \aipspp\ system; any
short-term surplus may be used for programmer workspaces.

This is as much as needs doing by the system administrator at this stage.  The
remainder of the initial part of the installation can be done by
\acct{aips2mgr}.

\subsection*{Step 2. Fetch the \file{install} scripts}

First create the \file{code} subdirectory:

\begin{verbatim}
   yourhost% cd /aips++
   yourhost% mkdir code
   yourhost% chmod ug=rwx,o=rx,g+s code
   yourhost% cd code
\end{verbatim}

\noindent
Now you're ready to fetch the sources.  However, instead of fetching
everything you will only need the \file{install} package plus a few other
files (the rest will be fetched automatically by \exeref{inhale}):

\begin{verbatim}
   yourhost% ftp aips2.nrao.edu
   Connected to tarzan.aoc.nrao.edu.
   220 tarzan FTP server (Version wu-2.4(5) Fri Jul 14 14:30:10 MDT 1995) ready.
   Name (aips2.nrao.edu:you): ftp
   331 Guest login ok, send your complete e-mail address as password.
   Password:you@yourhost.there
        :
     ftp welcome messages
        :
   230 Guest login ok, access restrictions apply.
   ftp> cd /pub/code
   250 CWD command successful.
   ftp> binary
   200 Type set to I.
   ftp> prompt
   Interactive mode off.
   ftp> hash
   Hash mark printing on (8192 bytes/hash mark).
   ftp> mget README VERSION configure install-*
        :
     lots of messages describing what ftp is fetching
        :
   ftp> quit
   221 Goodbye.
   yourhost%
\end{verbatim}

\noindent
(The \file{README} file contains a plain-text copy of these instructions.)
Then invoke \exeref{configure}:

\begin{verbatim}
   yourhost% chmod 544 configure
   yourhost% ./configure
\end{verbatim}

\noindent
You will be asked a series of questions, most of which have sensible defaults,
aimed at constructing the \aipspp\ \filref{aipshosts} file.  You then have to
edit your site-specific \filref{aipsrc} and \filref{makedefs} files.  Template
versions of these files are supplied by \exeref{configure} and you should read
the instructions carefully.  After making your site-specific definitions
\aipsexe{configure} will run some tests to check whether your \file{makedefs}
definitions look sensible.  Your \file{install} directory will then be made.
This consists of a few \textsc{c} compilations and installation of some shell
scripts, the most important of which is \exeref{inhale} itself.  Ignore any
error message from \exeref{gmake} concerning the non-existence of various
subdirectories of \file{/aips++/code}.


\subsection*{Step 3. Run \aipsexe{inhale}}

At this point your \aipspp\ installation has been bootstrapped to a state
where \exeref{inhale} can be run.  However, before invoking it you must ensure
that \rcs\ is installed.  If it isn't, you can get the sources via anonymous
\unixexe{ftp} from any \gnu\ archive site or from \host{aips2.nrao.edu} in
\file{/pub/import}.  Note that the \rcs\ installation goes through a prolonged
period of apparent inactivity, be patient!

You should also have a \cplusplus\ compiler, and a \TeX\ installation which
includes \LaTeX, \unixexe{dvips}, \textsc{MetaFont} and \textsc{latex2html}.
If you don't have a \cplusplus\ compiler you can still run \exeref{inhale} but
should pass it the \exe{-n} option so that it doesn't try to recompile the
sources.  Likewise, unset the \code{DOCSYS} variable in \file{makedefs} if you
don't have \TeX, it will prevent compilation of the \aipspp\ documentation.

Users of SysV based systems such as Solaris should be warned that
\exeref{inhale} requires the BSD version of \unixexe{sum} for computing
checksums.  You must ensure that the BSD version will be found ahead of the
SysV version in \acct{aips2mgr}'s \code{PATH}.  The \gnu\ version of
\unixexe{sum} (in the \gnu\ ``fileutils'' kit) provides both algorithms and
uses BSD by default.  Less salubrious possibilities are to put \file{/usr/ucb}
(Solaris) or \file{/usr/bsd} (IRIX) ahead of \file{/usr/bin} in
\acct{aips2mgr}'s \code{PATH}, or to create a symlink to the BSD version of
\unixexe{sum} in the \aipspp\ \file{bin} area.  

First invoke \exeref{aipsinit} to add the \aipspp\ \file{bin} directory to
your \code{PATH}.  If your interactive shell is a C-like shell (\unixexe{csh},
\unixexe{tcsh}) you would use

\begin{verbatim}
   yourhost% source /aips++/aipsinit.csh
\end{verbatim}

\noindent
whereas for Bourne-like shells (\unixexe{sh}, \unixexe{bash}, \unixexe{ksh})
you would use

\begin{verbatim}
   yourhost% . /aips++/aipsinit.sh
\end{verbatim}

\noindent
If you use some other shell you'll have to revert to one of the above for the
remainder of the installation.  Now invoke \exeref{inhale}

\begin{verbatim}
   yourhost% inhale
\end{verbatim}

\noindent
or if you don't want to rebuild

\begin{verbatim}
   yourhost% inhale -n
\end{verbatim}

\noindent
This will fetch all required sources from the master via \unixexe{ftp} and
install them.  If you are installing \aipspp\ from scratch \aipsexe{inhale}
will first fetch and install the ``base'' release.  This is the frozen version
of the source code distributed to end-users.  It will then fetch and install
the latest version of the sources which are under active development.  If you
made any mistakes in your \file{aipsrc} or \file{makedefs} definitions some of
these may become apparent during the installations.  After fixing them you can
recover via

\begin{verbatim}
   yourhost% gmake -C /aips++/code allsys
\end{verbatim}

\noindent
The \code{allsys} target will compile all \aipspp\ sources, including
documentation (assuming of course that you have the compilers).  If you just
wanted to compile the documentation alone you could use

\begin{verbatim}
   yourhost% gmake -C /aips++/code docsys
\end{verbatim}

If everything has gone properly you should now have an up-to-date \aipspp\ 
installation.  However, in order to keep it up-to-date you must define a
\unixexe{cron} job to run \exeref{inhale} on a regular basis.  The normal
procedure is to do a cumulative update every Saturday evening.  However, you
may wish to maintain a (possibly separate) system which is updated on a daily
basis.  Half-daily updates are also possible, but note that 12 hours may not
be sufficient time to rebuild the system.

The exact timing depends on your timezone with respect to the master.  New
updates are produced by 0700 and 1900 Socorro time (MST or MDT) but you should
allow at least an hour's grace before collecting them.  An example
\file{crontab} file might resemble the following:

\begin{verbatim}
   # Cumulative update of the AIPS++ slave directories each Saturday evening.
   00 22 * * 6   (. $HOME/.profile ; inhale -c) 2>&1 | \
      mail aips2mgr aips2-inhale@nrao.edu
\end{verbatim}

\noindent
(Note that all \unixexe{cron} entries must be one-liners but they are broken
here for clarity.)  You may need to add the \exe{-n} option to \exeref{inhale}
accordingly.  Note that, as in the above example, the log produced by
\aipsexe{inhale} is generally forwarded to \acct{aips2-inhale@nrao.edu}.
These logs are archived for about 10 days and are accessible via the \aipspp\ 
home page \url{http://aips2.nrao.edu/aips++/docs/html/aips++.html}.  This is
particularly useful for verifying code portability, especially on platforms
that a programmer doesn't have ready access to.  You should also add the email
address of a local person who will monitor the \aipsexe{inhale} logs
(\acct{aips2mgr} in the above example).

\subsection*{Step 4. Create the NFS automount}

\index{nfs@\textsc{nfs}}
\index{automount, \textsc{nfs}}

You should now have a living breathing \aipspp\ installation, but as yet won't
be able to check code in to, or out of, the master source code repositories
from your local machine.  Note, however, that a perfectly acceptable although
somewhat tedious alternative if you have an account on a machine in Socorro is
to log in remotely using either \unixexe{rlogin} or \unixexe{telnet} and check
the files in and out of the master directly.  There are also a set of
utilities which automate this mechanism somewhat (see \sref{rai}).

The usual mechanism for local checkin and checkout of \aipspp\ source files
from the master \rcs\ repository is that of an \textsc{nfs} mount on the local
\aipspp\ machine of the master directories from \host{aips2.nrao.edu}, the
host on which the master resides.

Note that the only traffic flowing via the \textsc{nfs} mount is that
associated with sources being checked in and out and is quite modest.  Source
code updates fetched by \exeref{inhale} do not travel by this mount but rather
via \unixexe{ftp} as compressed tar files.

At \aipspp\ sites other than Socorro itself the \textsc{nfs} mount runs
across the internet.  This is not particularly difficult to configure.
However, the \textsc{nfs} mount must be set up on every machine on which
checkins and checkouts are to be enabled, and \acct{root} privilege is
required to do this.  Therefore, your system administrator's assistance will
again be required.

The preferred way of implementing the \textsc{nfs} mount involves using the
\unixexe{automount} mechanism.  Permanent \textsc{nfs} mounts (implemented as
entries in \file{/etc/fstab} or \file{/etc/vfstab}) are not recommended since
they may hang if the server doesn't respond, whereas automounts timeout within
a reasonable period.

Permission to mount the \aipspp\ master \rcs\ repositories can be obtained by
sending a request via email to \acct{aips2adm@nrao.edu}.  You should specify
the full internet address, both symbolic and numerical, of the machines which
are to have access.  For example,

\begin{verbatim}
   grus.atnf.csiro.au     130.155.194.37
   bootes.atnf.csiro.au   130.155.194.40
\end{verbatim}

\noindent
These should be strictly limited to machines assigned for \aipspp\ programmer
use.

There are several ways in which the automount may be implemented.  The
recommended method is a ``direct'' automount local to the \aipspp\ programmer
machines; that is, one that is not distributed via the \textsc{nis}
\file{auto.master} and \file{auto.direct} maps.

In Solaris 2.x, the automounter is multi-threaded and no special precautions
need be taken.  Simply add the following line to \file{/etc/auto\_direct}:

\begin{verbatim}
   /aips++_master -rw,grpid,hard,intr aips2.nrao.edu:/export/aips++/master
\end{verbatim}

On the other hand, the automounter supplied with SunOS 4.x is single-threaded.
Since the \aipspp\ master may sometimes be inaccessible due to internet
connectivity problems, the best strategy is to invoke a separate instance of
the \unixexe{automount} daemon which deals exclusively with it.

Do the following for each of your \aipspp\ programmer machines (the directory
and file names are provided for illustration only; for example, you may prefer
to put the files in \file{/etc} instead of \file{/usr/local/etc}).  Create a
file \file{/usr/local/etc/auto.master\_aips++} containing the following single
entry:

\begin{verbatim}
   /-  /usr/local/etc/auto.direct_aips++  -rw,grpid,hard,intr
\end{verbatim}

\noindent
Now create \file{/usr/local/etc/auto.direct\_aips++} containing:

\begin{verbatim}
   /aips++_master  aips2.nrao.edu:/export/aips++/master
\end{verbatim}

\noindent
and create an additional invokation of the \unixexe{automount} daemon by
adding the following command to \file{/etc/rc.local} at the appropriate point
(near the end):

\begin{verbatim}
   automount -m -f /usr/local/etc/auto.master_aips++
\end{verbatim}

You should be able to adapt these strategies to other operating systems and
automounters without too much trouble.

\noindent
If you are not running the domain nameserver you will need to add an entry to
your \textsc{nis} \file{hosts} map for \host{aips2.nrao.edu} (or to
\file{/etc/hosts} if you're not running \textsc{nis}):

\begin{verbatim}
   146.88.10.7   aips2.nrao.edu
\end{verbatim}

\noindent
Note that this internet address may change in future.  Start the
\unixexe{automount} daemon by issuing the \unixexe{automount} command above as
\acct{root}, or alternatively reboot the machine.

Finally, as \acct{aips2mgr}, create a symbolic link to the master repositories

\begin{verbatim}
   yourhost% ln -s /aips++_master /aips++/master
\end{verbatim}

\noindent
To test whether the mount is working try doing

\begin{verbatim}
   yourhost% ls /aips++/master
\end{verbatim}

\noindent
After a little while you should get a listing of the master directory.  You
should also be able to run \exeref{ao} and \exeref{ai}, the \aipspp\ checkout
and checkin utilities.

If once you have the mount set up you encounter problems relating to
\textsc{nfs} operations timing out, for example as indicated by messages such
as "\code{NFS getattr failed for server aips2.nrao.edu: RPC: Timed out}", then
you may need to increment the \textsc{nfs} retransmission and timeout
parameters via the \code{retrans} and \code{timeo} mount options in the
automounter map.

Check the man page for \unixexe{mount} for further information.
