\chapter{Miscellaneous Duties}
\section{ClearDDTS}
We use ClearDDTS as our defect tracking system.  There is fairly extensive
written and web base documentation for the administrator.  ClearDDTs defects
reside in /users/ddts and is licensed to run on tarzan. A cron job runs four
times a day looking for lost defects.

From time-to-time, and email address will need to be corrected. The procedure follows:
\begin{enumerate}
\item logon on to tarzan as ddts
\item run patchbug .i.e.
\begin{verbatim}
patchbug -a Submitter-mail right@address AOCsoXXXXX
\end{verbatim}
\item run adminbug; and inside adminbug run dbms.
\end{enumerate}
Note: this can be done for any field in the ClearDDTs bug system.  I suggest  
consulting the ClearDDTs Administrator's Guide.
\section{Linkscan}
We use linkscan software to scan the AIPS++ web pages for broken and missing
links.  As aips2mgr on tarzan; cd /home/tarzan/httpd/linkscan; ./linkscan.pl.
Once the script has run it produces a series of reports that can be accessed
at http://aips2.nrao.edu/linkscan.

Configuration scripts for linkscan are found in
/home/tarzan/httpd/linkscan/default and userdoc.  Default is setup to scan the
entire aips++ web site, where userdoc will scan only the release doc tree.
\section{Htdig}
Htdig provides the search engine for the AIPS++ site.  Several cron jobs 
(run as wyoung) 
on tarzan that scan the AIPS++ code and docs tree.  Configuration files are found
in /aips++/local/htdig-3.1/conf.  Details on the care and feeding of htdig maybe
found at it's web site http://www.htdig.org.

\section{Email}
Several email aliases are maintained on tarzan in /etc/mail/tarzan.aliases-tail.
The aliases are update once a day around 0300 and require root privileges to 
make any changes.

We use mailman 
(maintained in Charlottesville) to handle the mailing lists although messages are
sent to local aips++ logs in /export/aips++/Mail on tarzan.
