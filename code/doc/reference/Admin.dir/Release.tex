\chapter{Preparing the Release}
\section{General timeline}
\section{Doing the uprev}
\begin{itemize}
\item Send a early morning/late night [to give Oz a chance to respond] notice
   to aips2-workers telling them that the uprev and check-in freeze will happen
      later today/tomorrow. 

\item Notify aips2-workers so that check-ins will be disabled in 30 minutes to
         allow any in progress check-ins to be completed.

\item (adm) Freeze check-ins on the development master repository by running:
\begin{verbatim}
 /export/aips++/scripts/do_disable develop
\end{verbatim}

\item (adm) Stop the regular exhales in the (aips2adm) crontab; if the misery
	drags on, the regular exhale can gum up the works

\item (adm) Do an initial exhale to ensure that everything is up to date:
\begin{verbatim}
 ksh
 . $HOME/.profile
 exhale 2>&1 | mailx -s "AIPS++ (initial) uprev exhale" aips2-inhale
\end{verbatim}

\item (mgr) Update the data repository in release installation, e.g.
\begin{verbatim}
  cvsup /usr/local/aips++/data/supfile
\end{verbatim}

\item (mgr) Run cumulative inhale (after exhale completes) on local installation
\item (user) run "stable" tests

\item (adm) Back up release master (/export/aips++/repository/release), if desired.

\item (adm) Delete the release master and copy development to release:
\begin{verbatim}
 cd /export/aips++/repository
 rm -rf release
 mkdir release
 cd release
 (cd ../develop; tar cf - .) | tar xf -
\end{verbatim}

\item (adm) In /export/aips++/repository/release, edit:
\begin{verbatim}										
 .cshrc  .login  .profile  aipsinit.csh
 aipsinit.es  aipsinit.rc  aipsinit.sh
 replacing each occurrence of /develop with /release remove editor backups
\end{verbatim}
												
\item (adm) Update ftp link, VERSION, and base release tar file (where <REL> is the (last) base
release upon which the release master is based):
\begin{verbatim}
 cd /export/aips++/repository/release
 rm pub
 ln -s /export/aips++/pub/versions/release pub
 rm -f pub/master/VERSION
 cp master/VERSION pub/master
 cp ../develop/pub/master/master-<REL>.000.tar.gz pub/master
 /opt/local/gnu/bin/touch --file ../develop/pub/master/master-<REL>.000.tar.gz pub/master/master-<REL>.000.tar.gz
\end{verbatim}
\item (adm) Run exhale on the release master:
\begin{verbatim}
 ksh
 . /export/aips++/repository/release/.profile
 exhale 2>&1 | tee $HOME/master/etc/rexhale.log | mailx -s "AIPS++ release exhale" aips2-inhale &
\end{verbatim}

\item(adm) Do uprev (note, backup of VERSION is kept in VERSION.<REL>):
\begin{verbatim}										
 /export/aips++/scripts/do_uprev
\end{verbatim}
\item (mgr) Run test inhales for both the release and develop masters to ensure
that they work properly.
\item (adm) Enable regular exhales in the (aips2adm) crontab
\item (adm) Unfreeze check-ins on the development master repository by running:
\begin{verbatim}
  /export/aips++/scripts/do_enable develop
\end{verbatim}
\item (adm) Unlock any locked files in the release RCS tree
\begin{verbatim}
/export/aips++/scripts/do_unlock.sh
\end{verbatim}
\end{itemize}

\subsection{The freezes}
\subsection{Update the Docs}
\section{Testing the build}
\section{Patching the release candidate}
Once the release master has been updated, patches for defect resolution and documentation will be
sent.  These patches are suppose to be sent a binary shar files (shar -B dir/thefile > myfile.shar).
In general only defects fixes are allowed for .cc, .h and .g files. Documentation is always welcomed.
Steps for patching always done as a user:
\begin{itemize}
\item Make a seperate code tree
\item Save the shar files into a directory, i.e. /home/tarzan5/wyoung/patches/jun21.
\item Edit the files to remove non-shar file text
\item Unpack the shar files into a clean code tree,
\begin{verbatim}
cd /home/tarzan5/wyoung/patches
./unpack.sh jun21 2&>1 | tee patch.log
\end{verbatim}
Check the log to make sure everything unpacked cleanly.  If there are problems notify the sender.
\item Patch the release master
\begin{itemize}
\item (adm) Enable checkins in the release master; /export/aips++/scripts/do\_enable release
\item (user) Apply the patches
\begin{verbatim}
cd /home/tarzan5/wyoung/aips++
../patches/do_patch.sh
\end{verbatim}
\item (adm) Disable checkins in the release master; /export/aips++/scripts/do\_disable release
\item (adm) Exhale the release
\item (mgr) inhale -R release -c
\item (mgr) check the build log for problems
\item (user) Notify aips2-workers of accepted patches
\begin{verbatim}
 cd /home/tarzan5/wyoung/patches
 awk -f okpatch.awk jun21/* > patch.mailing
 mailx -s "Accepted patches for release" aips2-workers@aoc.nrao.edu < patch.mailing
\end{verbatim}
Note the okpatch.awk script will need to be modified with each new version.
\end{itemize}
\end{itemize}
\section{Preparing the binary}
\subsection{The pnultimate build}
\subsection{Preproduction}
\subsubsection{Make the ISO image}
\subsubsection{Cutting the CD candidate}
\subsubsection{Release Candidate Testing}
