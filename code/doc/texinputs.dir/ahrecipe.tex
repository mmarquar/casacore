
\usepackage{longtable}
\usepackage{array}
\pagestyle{plain}

% The following marcos are necessary for the recipe system

% Define a macro for the overall recipe. We'd like the argument
% to go into a database for searching
\newenvironment{ahrecipe}[1]{{\Large\begin{flushleft}{\bf Recipe: #1}\end{flushleft}}}

%What is the type of this recipe (General, Synthesis, Single Dish)?
\newcommand{\artype}[1]
   {{\begin{flushleft}{\Large Type:} #1\end{flushleft}}}

%What are the goals of this recipe?
\newcommand{\argoals}[1]
   {{\begin{flushleft}{\Large Goals:} #1\end{flushleft}}}


% What AIPS++ facilities does the recipe use?
\newcommand{\arusing}[1]
  {{\begin{flushleft}{{\Large Using:} #1}\end{flushleft}}}

% What are the results of the recipe?
\newcommand{\arresults}[1]
  {{\begin{flushleft}{{\Large Results:} #1}\end{flushleft}}}

% What conditions are assumed for use of this recipe?
\newcommand{\arassume}[1]
  {{\begin{flushleft}{{\Large Assume:} #1}\end{flushleft}}}

%What kind of script executes this recipe?
\newcommand{\arscript}[1]
  {{\begin{flushleft}{{\Large Script:} #1}\end{flushleft}}}

%Who submitted this recipe?
\newcommand{\arsubm}[1]
   {{\begin{flushleft}{\large Submitted by:} #1\end{flushleft}}}

%What is the submitter's email address?
\newcommand{\arsubmemail}[1]
   {{\begin{flushleft}{\large Email:} #1\end{flushleft}}}

%When was this recipe submitted?
\newcommand{\arsubmdate}[1]
   {{\begin{flushleft}{\large Date:} #1\end{flushleft}}}

% Verbatim example:
\newcommand{\arverbatim}[1]
   {{\begin{verbatim} #1 \end{verbatim}}}



% Macro for the example (consisting of \arlines's and \aroutputs)
\newenvironment{arexample}
  { \setlongtables
   \setlength{\LTleft}{0pt}
   \setlength{\LTright}{0pt}
   \setlength{\parindent}{0pt}
   \begin{longtable}[c]{p{2.5in}@{\hspace{0.3in}}p{3.2in}}
   \hline \textsl{\bfseries AIPS++/Glish commands and results} & \textsl{\bfseries Purpose and Background}\\
   \\
   }{\end{longtable}}

% Concluding remarks?
\newcommand{\arconclusion}[1]
  {{\begin{flushleft}{{\Large Conclusion:} #1}\end{flushleft}}}

% Macro for lines of input (first argument) and comments (second
% argument
\newcommand{\arline}[2]
  { \begin{tabular}{>{\ttfamily}p{2in}}
    #1
    \end{tabular}  & \begin{tabular}{>{\normalfont}p{2.5in}}
                     #2 
                     \end{tabular}
   \\
   \\ }

% Macro for output from a command
\newcommand{\aroutput}[1]
  {\\
   \multicolumn{2}{p{4.5in}}{\emph{#1}} \\ 
   \\}

