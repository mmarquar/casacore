\newcommand{\aips}{\textit{AIPS++}}
\section{Overview}
The \aips documentation is a combination of standards and tools.
The standards are HTML, LaTeX, and \htmladdnormallink{\aips guidelines}{}. 
The tools are cxx2html, help2tex, and
latex2html, with a dash of shell programming thrown in.
All \aips documentation is available via a web interface.  
Most of the user and system
documentation and \aips Notes and Memos are written using the LaTeX
mark up language. Some older documents may be in texinfo which was abandoned
in favor of LaTeX. C++ classes are documented in header files using 
special cxx2html tags.

Cxx2html, help2tex, and latex2html are all perl scripts. 
A working knowledge of perl is helpful to understand them and essential for 
debugging.

All documentation source is found in the "code" tree.  The make system's docsys
target generates and moves postscript/html files into the "docs" tree.

\section{LaTeX based documents}
\subsection{General Documentation}
\subsubsection{General Notes}
Documents in the \aips Notes and Memo series are mostly written using LaTeX.
All new documents should be written in LaTeX.
There are a few notes that are plain ASCII (.txt) files.  The make system
generates postscript
files from files with .latex and .tex extensions.  Makefile.doc located in
code/install contains all the rules for generating postscript and html files
and moving them into the "docs" tree.

The \aips Notes and Memos series each contain a latex file that has the titles
and authors for each \htmladdnormallink{Note}{../notes/notes.html} and 
\htmladdnormallink{Memo}{../../memos/memos/memos.html}.  Each time a note or memo
is checked in, the index.tex file in code/doc/memos/memos.dir and
code/doc/notes/notes.dir needs to be edited to include the new note or memo.

\subsubsection{Generating HTML}
HTML documents are not automatically generated from LaTeX source.  The author
needs to add a line to the directory's makefile similar to:
\begin{verbatim}
documentroot := -split +2
\end{verbatim}

where documentroot is the file's name less the .latex extension.  After
specifying this line 
latex2html will be run on that document during a sneeze.  Most \aips documents
should use
\begin{verbatim}
-split +1 or +2.
\end{verbatim}


The default behavior of latex2html may altered using at .latex2html-init
file.  Typically the default split level, and appearance of the HTML pager
header and footer will be customized.

Latex2html uses absolute path names to its icons, the shell script
redoicons.sh (code/install/docutils) turns the absolute paths into relative
ones.  Using relative icons avoids have latex2html generated HTML pages
use NRAO as an icon server. To get relative icons sites should specify two
variables in thier makedefs file ICONSERVER and ICONS2LOCAL.  ICONSERVER is
the ICONSERVER variable specified in the latex2html script.  At the AOC,
we have the following two lines in our makedefs to generate local icons:
\begin{verbatim}
ICONSERVER := http://www.nrao.edu/icons/latex2html
ICONS2LOCAL := $(AIPSARCH)/bin/redoicons.sh
\end{verbatim}

For more details about the html commands available from latex2html please 
consult the
 \htmladdnormallink{latex2html documentation}{ 
http://www-dsed.llnl.gov/files/programs/unix/latex2html/manual/manual.html}
found at\\ 
\htmladdnormallink{http://www-dsed.llnl.gov/files/programs/unix/latex2html/manual/manual.html}{http://www-dsed.llnl.gov/files/programs/unix/latex2html/manual/manual.html}.

Latex2html should be found in the system area (/opt/local/bin/latex2html at the
AOC).  \aips  uses version 98.2beta5 (1998-Jul-28).  Latex2html requires several utilities
\begin{itemize}
\item Perl (5.003 or later).
\item dvips (5.516 or later, preferably 5.62 or later)
\item gs (4.03 or later).
\item netpbm (1 March 1994 version)
\end{itemize}

Please consult the README distributed with latex2html for more details.

\subsubsection{HTML Links to Other Documents}
There are several methods for producing links to other documents.  To
reference an HTML document, use:\\
\verb!\htmladdnormallink{text}{link to html page}!.\\
Provided an author uses
\verb!\label! commands you may reference other 
LaTeX based document using 
\verb!\externallabels! command and \verb!\htmlref!.

To make a link to the \htmlref{copy}{catalog:catalog.copy} command of the
catalog object in the catalog module use \verb!\htmlref!
\begin{verbatim}
\htmlref{copy}{catalog:catalog.copy}
\end{verbatim}

The \verb!\externallabels{URL to document}{path to labels file}! command
needs to appear at the top of the document before the \verb!\begin{document}!. 
The URL and path to labels file need to be relative paths.  So for Note 215
\begin{verbatim}
\externallabels{../../user/Refman}{../../user/Refman/labels.pl}
\end{verbatim}

Help files auto-generate some labels based on the following scheme:
\begin{itemize}
\item module
\item module:function
\item module:object
\item module:object.method
\end{itemize}
Additional labels may be put in the .help file by the author using the 
\verb!\label! command.

The order of when documents are run through latex2html is important
for external cross referencing. All released 
documentation needs two passes through latex2html to resolve all 
external references (insures the use of the proper labels.pl file).  

Funny things can happen if labels are multiply defined.


Cross referencing into HTML files created by cxx2html is possible using
\verb!\htmladdnormallink!.  To make a reference to the array class
\htmladdnormallink{array class}{../../aips/implement/Arrays/Array.html#Array}
from a note or memo use:
\begin{verbatim}
\end{verbatim}

Please consult the \htmladdnormallink{cxx2html documentation}
{../../RELEASED/cxx2html.html} to see how anchors are created.

Cross referencing from an HTML document to something in the User's
Reference Manual or another LaTeX based document is possible.
The link must be identified 
in the generated HTML.  This is not recommended since latex2html generated
file names and anchors will likely change.

A word of caution, relative links in the docs tree may be different than those
found in a programmer's code tree.  Within the code/doc tree relative links
should be OK (i.e. builds of LaTeX documents should cross reference properly
amongst themselves for programmer and system builds).  Relative links to html
pages generated by cxx2html will not be the same for programmer and system
builds!  The implement directory is missing in the docs tree.

\subsection{Adding Notes and Memos}\label{215:addnotes}
\textit{Contributed by Wim Brouw, ATNF}\newline
It took me a while to insert a Note in the system and considering the
difficulty others have had, I make the following points:

\begin{enumerate}
\item Select a note number (I just take the next number available, but maybe
   there is somebody to ask?) - say nnn (e.g. 225)
\item Create (check in) directory code/doc/notes/nnn.dir
\textit{(If it's there then others will know the number is taken.)}
\item Put the body of your latex Note (without embracing document information
   like documentstyle, title, \verb+\begin{document}+ etc) in nnn.dir (e.g.
   ai -l mynote.latex)
\item Create, (in code/doc/notes itself) a file called nnn.latex containing the
   document information, and which includes mynote.latex (without a
   directory!)
   Example:
\begin{verbatim}
\documentclass[11pt]{article}
\usepackage{html, epsf}
%%-----------------------------------------------------------------------------

\begin{document}

\title{NOTE 224 -- AIPS++ Least Squares background}
  

\author{Wim Brouw}

\date{22 January 1999}

\maketitle
%%---------------------------------------------------------------------------
\begin{htmlonly}
\htmladdnormallink{A postscript version of this note is available 
(124kB).}{../224.ps}
\end{htmlonly}

\tableofcontents
      
\input{lsq.latex}
\end{document}
\end{verbatim}

\item Add your note number to the code/doc/notes/makefile (just do like the
   other notes, e.g.:
\begin{verbatim}
224 := -split 0
\end{verbatim}

\item Test if all ok, by making a Postscript version:
   gmake nnn.ps\newline
   And an html version:\newline
   gmake nnn\newline

\item If all is ok, then add your Note to the Notes index:
\begin{verbatim}
   code/doc/notes/notes.dir/nindex.tex
   E.g:
\item[224]
\htmladdnormallink{\textit{AIPS++ Least Squares background}}{../224/224.html}
\linebreak  1999/01/25 Brouw
\end{verbatim}

\end{enumerate}


\textit{Note: Other notes, AIPS++ documents can include your note or parts of your note
by adding the following line to the makefile}
\begin{verbatim}
TIROOT := $(word 1, $(AIPSPATH))/code/doc
EXTRA_TEXINPUTS := $(TIROOT)/memos/111.dir:$(TIROOT)/notes/156.dir
/notes/196.dir
\end{verbatim}
\textit{you then use the} \verb+\input{file}+
\textit{ to include it the other document.
The HowTos (code/doc/reference/HowTos.latex) is a good example.}


\subsection{Help System}
The help system is a special type of LaTeX document.  User documentation for
packages, modules, objects, and functions are written in .help files.
Authors use the \htmlref{aips2help.sty}{197:aips2help} package to format their
help text.  The perl script \htmlref{help2tex}{system:help2tex} is run on the
.help file to produce "standard" LaTeX so no special modules are needed by
latex2html.
Help2tex provides a generate consistent looking documentation if the
aips2help.sty commands and environments are used.

The .help files are usually found in same directory as the "code".
The docsys target for code/doc/Refman has been enhanced to do the following:
\begin{enumerate}
\item Collect all the .help files into a temporary directory.
\item run help2tex (a perl script, code/install/docutils) on each package in
the system. Help2tex is run run twice on each package first to to generate
a .htex file (found in docs/user/helpfiles) and then to generate an atoms.g
file used for glish command-line help.
\item run latex to generate the postscript file in docs/user/Refman.ps.
\item run latex2html to generate the html files in docs/user/Refman.
\end{enumerate}


\section{HTML based documents}
\subsection{Top-level WWW pages}
Several WWW pages are found in code/doc/html.  These pages are "straight" HTML
and copied directly into the docs tree.  The \aips home page uses frames because
of the rather complex nature of the documentation tree.  Frames provide a rather
convenient way to navigate the rather large amount of \aips documentation.

Here is a list of some \aips web pages found in code/doc/html:

\begin{description}
\item[aips++.html] Frame based \aips home page.
\item[frameonline.html] Content based html page (left side frame).
\item[homepage.html] Body of \aips home page (right side frame).
\item[online.html] \aips documentation matrix.
\item[betasearch.html and aips2search.html] Search pages for beta and daily documentation.
\item[homepages.html] Page containing links to \aips worker's home pages.
\item[reportbug.html] Problem report page.
\item[whatsnew.html] What's new with \aips
\item[ftp.html] The ftp page (not used currently).
\item[cxx2html.html] How to run cxx2html.
\item[design.html] Alan Bridle's, "\aips Information System Design" document. 
\end{description}


\subsection{Cxx2html generated documents}
C++ class documentation is contained in the C++ header files. \aips uses 
the perl script
\htmladdnormallink{cxx2html}{../../html/cxx2html} (code/install/docutils) to
extract and assemble the class documentation.

\section{Searching \aips Documentation}
\aips uses the htdig system \htmladdnormallink{(http://htdig.sdsu.edu)}{http://htdig.sdsu.edu} to perform searches.
Htdig traverses the web from a specified starting point (\aips home
page) and creates a database of words and documents.  A daily cron job
(/aips++/local/htdig/bin/rundig) is run
to recreate the databases.  The configuration file
(/aips++/local/htdig/conf/htdig.conf)
offers many ways to tune the database.  The search page supplied with
htdig has been customized for \aips and checked into the system
(code/doc/html/aips2search.html). 

Root privileges are needed to install/modify htsearch in the httpd/cgi-bin
directory and add the htdig icons (/tarzan/httpd/icons).

All document searching is currently done on tarzan.aoc.nrao.edu.  Distributing
the search database with the system is not contemplated until there is a
demand for searching local documentation.

\section{\aips Problem Reporting System}
The \aips problem reporting system uses GNATS 3.100-beta as the underlying
system for managing and tracking problems. There are four ways to submit
"a gnat":
\begin{enumerate}
\item send-pr,
\item inside emacs with send-pr.el, 
\item using the modified CERN wwwgnats, or
\item bug() (from glish).
\end{enumerate} 
Most \aips users will use bug or the cern-wwwgnats interface to submit problem
reports.
\subsection{GNATS}
\aips uses GNATS 3.100-Beta.  The current version of
GNATS is 3.104-beta and is available from
\htmladdnormallink{ftp://ftp.cygnus.com/pub/gnats/gnats-3.104-beta.tar.gz}{ftp://ftp.cygnus.com/pub/gnats/}.
A \htmladdnormallink{postscript file}{http://aips2.nrao.edu/wyoung/gnats.ps} of
the gnats documentation is available (at the AOC http://tarzan/wyoung/gnats.ps).

For those not wanting to wade through the documentation a brief
description follows.
\begin{itemize}
\item An email is sent to aips2bugs@nrao.edu,
\item a cron job puts the email into the system (run by gnats on tarzan every
10 minutes),
\item The "appropriate" folks are notified (initial reports are sent to Tim
Cornwell, Brian Glendenning,
and Wes Young). Tim then reassigns the the bug to the appropriate category and
individual.
\item All responses to a bug report are logged with that report.
\item Unrecognized responses are sent to the gnats administrator (Wes Young)
who figures out what to do
with them.
\end{itemize}
Currently, the gnats, executables, and configuration files are found in 
/aips++/local/lib/gnats.  The configuration files are found in
gnats-db/gnats-adm.
The directories found gnats-db will be the category directories which hold the
actual bug reports.
\subsection{WWWGNATS}
The standard distribution of GNATS comes with two web front ends to the GNATS
system, wwwgnats and cern-wwwgnats. \aips has chosen to use the cern-wwwgnats
because its interface was deemed "a better fit" to \aips needs than the standard
wwwgnats.
Cern-wwwgnats is found in the gnats source tree at
gnats/contrib/cern-wwwgnats.  The files in
3.100-B\_src-mods are just copied into the gnats src tree and the gnats is
built.  Alan Lovell (\htmladdnormallink{Alan.Lovell@cern.ch}
{mailto:Alan.Lovell@cern.ch}) at CERN reports he'll eventually update his
modifications to gnats but he's not sure when.

Cern-wwwgnats imposes a domain directory on top of the standard gnats category
files. The domains chosen for \aips are aips, allbugs, atnf, bima, glish,
nfra, nrao, synthesis, and unassigned.  Except for allbugs and unassigned these
correspond to \aips packages.  Inside domains there are categories, i.e.
the aips domain has the following categories arrays, graphics, logging, 
images, lattices, mathematics, measures, tables, utilities, and design.
A category may belong to more than one domain, i.e. all categories belong to
the allbugs domain.

Gnats uses three master
files, categories, editors, and responsible.  Cern-wwwgnats has a categories, 
editors and responsible file for each domain.
These files are found in gnats/gnats-db/gnats-adm.  A brief summary of the
files follows:
\begin{itemize}
\item The domains file contains a list of \aips domains (packages for the
most part).

\item The categories file
contains the category name, a short description, the primary responsible
person, and a list of others to be notified about problems for that category.
Each domain has a categories-"domain" file.

\item The editors file contains a list of who can edit
particular categories. Each domain has an editors-"domain" file.

\item The responsible file contains who is responsible
for addressing problems for that category and who should be notified.
Each domain has an responsible-"domain" file.

\item The config file contains default configuration settings for submitting a "pr".

\item The current file contains the number of the most current bug report.

\item The index file contains a list of all bug reports.
Gnatsd.conf configuration file for a gnats daemon (not used).
\end{itemize}

Several files need to be in the cgi-bin directory (need root access) , libgnats.pl,
mailto.sh, sendmailto.pl, and wwwgnats.pl.  Wwwgnats.pl is the main program.
Some small changes need to be made to these files, customize email addresses
and document titles for \aips use..

\subsection{W3C HTTPD}
The perl scripts used by cern-wwwgnats need the
\htmladdnormallink{W3C httpd}{http://www.w3.org/Daemon/}.  Even
though the httpd.conf files says it's NCSA it isn't. Several commands
(MAP, EXEC, and PASS) at the end of the configuration file are necessary for
cern-wwwgnats.
Testing is best done using a different port number
on another machine before installation on the \aips server.  
See \htmladdnormallink
{http://www.w3.org/Daemon/User/Installation/Installation.html}
{http://www.w3.org/Daemon/User/Installation/Installation.html} for more details
on installing and testing the W3C httpd daemon.
\section{Mailing list setup}
From time-to-time new mailing lists are need.  The process is as follows
\begin{enumerate}

\item Check-in the new mailing list in code/admin/system.
\item Supply the mail alias to NRAO-CV (\htmladdnormallink{Jeff Uphoff}{mailto:juphoff@nrao.edu}), they put the alias into the master
alias table.  \aips
mailing lists have the form aips2-mailinglist, where mailinglist is the name
of the mailing list.
\item On tarzan as aips2mgr,
edit the /etc/mail/tarzan.aliases-tail file and put in three lines like
\begin{verbatim}
#A comment about the new mailinglist
aips2-newlist: :include:/export/aips++/master/etc/aips2-newlist.list
aips2-newlist-log: "| /export/aips++/master/etc/aipsmail aips2-newlist"
\end{verbatim}
\item A daily cron job run by root appends the tarzan.aliases-tail file to
tarzan's alias list.
\item Send a test message.  Put a symbolic link to the list recipients.
On tarzan, cd /export/aips++/Mail/aips2-newlist, ln -s ../../master/etc/aips2-newlist list
\end{enumerate}
More details on \aips mailing lists may be found in the
\htmlref{\aips System Manual} {email exploders}.
% LocalWords:  init dsed llnl unix longtables Refman htex helpfiles glish htdig
% LocalWords:  sdsu cron rundig conf htsearch httpd tarzan nrao wwwgnats cern
% LocalWords:  cygnus wyoung Cornwell Glendenning contrib Lovell allbugs atnf
% LocalWords:  bima nfra config Gnatsd libgnats mailto sendmailto NCSA admin
% LocalWords:  Uphoff juphoff mailinglist newlist aipsmail frameonline homepage
% LocalWords:  betasearch homepages reportbug whatsnew Bridle's HTML html perl
% LocalWords:  texinfo docsys docs
