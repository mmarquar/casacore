\newcommand{\refman}{{\it Reference Manual}}

\begin{abstract}
This note examines some of the current problems and deficiencies of
the AIPS++ documentation and considers how an XML-based approach might
be used to address them.  I describe an over-all framework based on
two key features: the use of the DocBook markup for expository
documents and the use of in-line documentation for glish-based tools
and functions.  This framework will continue to supply users with both
HTML and high-quality printable versions of all documentation.  I
describe the tools needed to support this framework and outline a plan
for migrating from our present system.  An important goal of this
framework is to make it easier for developers to create and maintain
high quality documentation that will improve the user experience.  It
is expected that this note will serve as the basis of one or more
formal change proposals aimed at adopting this framework.
\end{abstract}

\tableofcontents

\section{Introduction:  The Current State of \aipspp\ Documentation}

\subsection{Current Principles and Framework}

\htmladdnormallink{Note 215}{../../notes/215/215.html} describes the
current framework of documentation in \aipspp.  In general, there are
three types of documentation in \aipspp:

\begin{itemize}
\item Expository documents (e.g. {\it 
\htmladdnormallink{Getting Started}{../../user/gettingstarted/gettingstarted.html},
\htmladdnormallink{Getting Results}{../../gettingresults/gettingresults/gettingresults.html}},
\htmladdnormallink{Notes}{../../notes/notes.html}, 
\htmladdnormallink{Memos}{../../memos/memos.html}), usually in \LaTeX\ 
format;

\item Glish tool documentation (i.e. the {\it
\htmladdnormallink{User Reference Manual}{../../user/Refman/Refman.html}});

\item C++ Class documentation, using SGML markup in-lined into C++
header files and extracted with 
\htmladdnormallink{cxx2html}{../../html/cxx2html.html}.

\end{itemize}

\noindent The first two types are the most important to users.  The
{\it Getting *} documents are meant to instruct users
in the general skills and approaches used to process data in
\aipspp.  It's expected that users will read large parts of it in
whole.  The {\it Reference Manual}, on the other hand, is meant
primarily as a definitive guide to specific tools and functions; users
consult this document to locate specific bits of information.  

The current documentation framework is based (in part) on the
following principles:

\begin{enumerate}
\item Users must have access to documentation on-line.
\item Users must be able to print out documentation in a high-quality
form.
\item It must be possible to build the documentation anywhere AIPS++
is installed.
\item The above features are be enabled via three ``cornerstone''
formats: \LaTeX, HTML, and PostScript.  
\end{enumerate}

\noindent \LaTeX\ is the source format for the documents aimed at
end users.  In particular, developers use specialized macros to
markup components of the documentation that makes up the {\it
Reference Manual}.  

\subsection{Room for Improvement}

This section lists some of the problems with the current system which
this study hopes to address.

\subsubsection{General}

\paragraph{Documentation build system is fragile and difficult to
maintain.}  Because the system depends on complex, external packages
which must be installed seperately (\LaTeX\ and \LaTeX2HTML), setting
up the system up to build documentation automatically is not easy.
UIUC does not build documentation locally because of the problems
we've had trying to configure the system.  Instead, we import the
built documentation from Socorro.  

\LaTeX2HTML\ is certainly the weakest link in the documentation
system.   Since importing docs from Socorro, we have found that
the HTML version of the \refman\ regularly does not format
argument lists properly.  Problems formatting equations and images are
common but often go undetected by the make system.  

\paragraph{\LaTeX\ is sufficiently complex as to make creating
documentation non-trivial.}  Many of the formatting problems that
currently exist can be traced to mark-up errors on the part of the
author.  Common errors like the ``missing brace'' can have drastic
effects on the formatting.  It's also important to check the
translation to HTML; a document that processes fine with \LaTeX\ can
be mangled in a variety of ways when passed through \LaTeX2HTML.  The
\aipspp\ system does provide developers with tools checking their
documentation ({\tt make {\it file}.dvi} and, more recently {\tt make
{\it file}.html}).  Nevertheless, formatting errors are still quite
common.  This may be largely due to developer laziness; however, it
may reflect in part on the difficulty in getting the tools to work
properly.

This author's own difficulties are primarily due to the fact that the
UIUC installation is not properly configured to build the
documentation.  However, I was also unable to process my documentation
on {\tt tarzan} at the AOC.  If this is a common experience among
developers, then the supporting tools are not sufficiently helpful in
checking our documents.  

\paragraph{It is difficult to adjust the presentation of
documentation.}  This is due to the inherent complexities of \LaTeX\
and \LaTeX2HTML.  Significant skill and experience is needed to
fiddle with \LaTeX\ style files to adjust the document layout.  It is
arguably even more difficult to adapt \LaTeX2HTML\ to variations in
presentation.  This barrier of complexity not only discourages
experimenting with presentation but also the resolution of formatting
problems.  

As an example of experimenting with presentation, some developers have
expressed a preference for layouts that maximize the amount of content
that can be viewed at once.  This might call for smaller margins in
the hardcopy version.  It might also call for a re-arrangement of
information presented in a \refman\ page.  Small changes as these
could have big effects on the documentation's user--friendliness;
thus, it would helpful if it were easier to modify the layout and try
new presentations.

\subsubsection{Reference Manual}

The overall organization of the \refman, with its hierarchy of
packages, modules, and tools, is good.  A significant portion (if not
the majority) of the content is of good quality.  Nevertheless, it can
still be difficult to find specific information on how to use specific
tools and functions.  This is not always due to a paucity of words,
but often due to structure.  

\paragraph{Process for creating module documentation is laborious}
After creating a tool implementation, the developer must create two
separate files ({\tt {\it tool}.help} and {\tt {\it tool}\_meta.g}),
each to some extent containing redundant information both between them
and the Glish source code.  As separate files and with different
formats, they are not conducive to being developed in parallel with
the software.  Thus, they are usually done as a separate chore, after
the implementation.  This can tend to discourage completeness,
particularly in describing allowed values to functions or caveats to use
special circumstances; this is because, since the developer is
somewhat more removed from the inner workings of the code, including
this information is more difficult.  

\paragraph{Structure of function description makes it hard to find
information related to specific parameters.}  It is common for a user
to want to discover the meaning and use of a specific parameter.  The
obvious place to look is the table of function parameters; however,
this contains limited information (see next item).  For additional
details, the user must sift through the general function description.
(New users may not know to expect additional information to be hidden
there.)  It would be better if all the information related to a
parameter could be found in one, easy-to-find place.  Repetition of
this information in the general description is fine and often useful
when describing the role of the most important parameters in the
overall use of the function.

\paragraph{The presentation of the parameter description discourages
complete descriptions.}  Function parameters are listed and described in a
structured table.  In the HTML version of the manual, this table is
quite narrow (perhaps one half the width of a typical browser
window), so very little information can be contained in this table
before it becomes difficult to read.  As a result, developers must go
to the effort of including more complete information in the general
description of the function, which sometimes does not happen.
Thus, parameter descriptions are limited to a single clause or
sentence, and default and allowed values sometimes go undescribed.
And because the content of the general description section is
unstructured, the fact that this information is missing goes unnoticed
(except by those specifically looking for it).  

\subsection{The Case for a New Approach} 

It is worth noting that a \LaTeX--based system can probably be made to
work and our overall system improved for users with additional effort,
perhaps even comparable to the effort of converting to a new
framework.\footnote{The author notes his own experience and the
experience of others in our community supporting \LaTeX--based
proposal submission systems and conference proceedings created from
\LaTeX\ documents contributed by participants.}  However, the inherent
complexities of \LaTeX\ and \LaTeX2HTML\ discourages us from improving
and enriching the system.  In contrast, an XML-based framework offers
new possibilities for developing documentation features
(e.g. two-way links, automated indexing, context-based help, ASCII
displays for Glish, integration with Tool Manager metadata).  

More important than new capabilities, a new approach to documentation
offers the opportunity to make authoring and delivering
documentation easier.  This note posits that if this process can be
made less laborious, authors will create higher quality
documentation--that is, more comprehensive and with the necessary
information where users can easily find it.  Indeed, substantial
improvement in documentation quality for the end user must be a
requirement for any substantial change in the documentation system.  

\section{An XML-based Framework}

This proposed framework addresses the creation, maintenance, and
delivery of the two types of documentation aimed at end users:
expository documents and the \refman.  The author feels that the
current framework for documenting C++ code in-line is pretty good.
It's backed by a formal quality assurance process that is reasonably
effective.  Furthermore, the large amount of C++ code documentation
already in place makes substantial changes to it hard to justify.
Documentation for the end user is more important.  In particular, this
author feels that changes to how the \refman\ is constructed will
result in the greatest benefits for users.

This new framework is based on the following principles:

\begin{enumerate}
\item Users must have access to documentation on-line.
\item Users must be able to print out documentation in a high-quality
form.
\item It is not necessary that documentation built locally from source
files as part of a standard AIPS++; installing pre-built documentation
should be sufficient.
\item Authors must be able to {\it easily} build and validate their
documents for the different versions supported by the system.
\item HTML is the best format for the on-line version as it offers the 
greatest flexibility to the user for presentation (e.g. font style and
size, window size, foreground and background colors).  
\item It is easier to enable automated machine-processing of
metadata and documentation using XML than \LaTeX.
\end{enumerate}

The proposed framework is characterized by two key features:

\begin{itemize} 
\item \htmladdnormallink{DocBook}{http://www.oasis-open.org/docbook/},
a system for structuring documents in SGML or XML, is used to create
and maintain expository documents. 
\item The \refman\ as well as the Tool Manager metadata files
(i.e. {\tt {\it tool}\_meta.g}) are generated from in-lined comments
using a markup similar to JavaDoc but which is converted to XML upon
extraction.  
\end{itemize}

\noindent XML will serve then as the base format for conversion into
other formats.  HTML is used for browser-based access.  Plain text can
be used for access within an interactive Glish session.  Hardcopy can
be provided in either PostScript or PDF format.  

\subsection{Use of DocBook} 

DocBook is perhaps best known as the most widely used SGML markup
language available as open source and freely available.  It is
maintained by the DocBook Technical Committee of the
\htmladdnormallinkfoot{Organization for the Advancement of Structured
Information Standards (OASIS)}{http://www.oasis-open.org/}.  DocBook
began its life in 1991 as an SGML DTD; however, with Version 4.1, an
XML version is now available and supported by the most common
DocBook-compliant tools.  The XML version includes MathML as an
extension for marking up mathematical equations.  The DocBook DTD is
modular in design and has an explicit framework for extending it to
add new elements or further control attribute values.

\subsubsection{Advantages of using DocBook}

\paragraph{DocBook is well supported}

\begin{itemize}
\item {\bf O'Reilly publishes a book on DocBook}\footnote{Walsh, N \&
Muellner, L. 1999. {\it DocBook: The Definitive Guide}, (O'Reilly:
Cambridge)}. 
\item {\bf LyX, a free WYSIWYG editor,%
%
\footnote{This is not strictly true
according to the LyX web site, \htmladdnormallink{http://www.lyx.org/}%
{http://www.lyx.org/}, which calls it WYSIWY{\it M}, as in ``what you
see is what you mean.''  This is more appropriate for our purposes,
since presentation is separated from content; nevertheless,
the on-screen rendering gives the author a visual indicator about how
the document is organized.}
%
can create and export DocBook-format documents.}  Successful use of
this application could minimize the amount authors need to learn about
DocBook as a markup language.  It is possible to add new Layout
classes to LyX that make it easier to create documents that must
conform to particular styles.  (In fact, in addition to its DocBook
layout classes and templates, LyX also ships with support for other
styles, including one for the AAS journal manuscripts.)  An
easy--to--use word processor of this sort may make documentation
authoring more attractive to non--developers.

\item {\bf Several packages are freely available for converting
DocBook to other formats}.  Jade provides extensible DSSSL style
sheets for converting to \LaTeX and RTF. (Jade can also be used for
validation.)  nwalsh.com supplies free,
customizable XSL stylesheets for converting to HTML.  Novell has 
produced a UNIX script for converting DocBook to HTML.  

\item A few commercial products such as {\bf Adobe's FrameMaker+SGML}
and {\bf WordPerfect 9.0 for MS Windows} support DocBook ``out of the
box.'' 

\item The wide variety of generic SGML/XML tools can be used with
DocBook documents.
\end{itemize}

\paragraph{DocBook is widely used.}

\begin{itemize}
\item The \htmladdnormallinkfoot{{\bf Linux Documentation Project}}%
{http://www.linuxdoc.org/} uses DocBook for authoring the various
HOWTOs, FAQs, and Project Guides it maintains.  In fact, there are a
few aspects of the LDP model that we can borrow:
\begin{itemize}
\item availability of a \htmladdnormallinkfoot{HOWTO-HOWTO document}%
{http://www.linux.org/docs/ldp/howto/HOWTO-HOWTO/} for new authors,
\item style guides,
\item standard style sheets,
\item recommended authoring tools.
\end{itemize}
\item The \htmladdnormallinkfoot{{\bf K Desktop Environment (KDE)}}%
{http://www.kde.org/} uses DocBook for its documentation.  They
provide a very nice ``crash course'' document entitled
\htmladdnormallinkfoot{Writing Documenation using DocBook}%
{http://www.caldera.de/~eric/crash-course/HTML/}.
\end{itemize}

\paragraph{DocBook is a cross-platform format.}

\paragraph{DocBook documents can be edited with any editor.}

\paragraph{As with all SGML/XML formats, content and presentation are
seperated.}  This has two important ramifications.  First, the
presentation can be evolved without having to alter the source
documents.  Second, it is not necessary that all authors have the
specific skills for manipulating presentation.

\paragraph{XML offers greater flexibility for controling
presentation.}  It's worth noting that in principle, \LaTeX\ separates
content and presentation as well; however, in can be argued that this
does not happen in effect.  The expertise necessary to alter a style
file and subsequently support it with \LaTeX2HTML\ discourages
experimentation with presentation.  In this author's opinion, {\it
manipulating XML style sheets is much easier than \LaTeX\ style
files.}  

\subsubsection{Potential Pitfalls}

Very few actual experiments using DocBook for AIPS++ have been done to
date, and many of the details concerning how we might use it have not
been worked out, yet.  Here are some ways that DocBook may not live up
to its promise: 

\begin{itemize}
\item The effort for adapting style sheets for use with AIPS++ may be
more extensive than expected.

\item Extending DocBook for handling AIPS++--specific structures
(analogue to {\tt aips2defs.tex}) may be more difficult than
expected, or doing so makes generic DocBook tools less helpful.  (It
appears at this time, however, that we can make effective use of
DocBook with little or no extensions to the DocBook DTD.)  

\item It may be necessary but difficult to adapt LyX for creating
AIPS++ documentation.  Without LyX or similar product, authors will
have to learn the details of the DocBook DTD which defines about 100
different elements.

\item It is too difficult to convert existing documents into DocBook
when necessary.  This is expected to apply only to documents like {\it
Getting Results} that will continue to evolve and expand.

\end{itemize}

\subsection{In-line Documentation for Glish}

\subsubsection{The Case for In-line Documenation}

Given the proper supporting tools, in-line documentation of software
can provide tremendous advantages:

\begin{itemize}

\item Much of the information that needs to appear in the
documentation can be extracted directly from the code itself.  This
includes: 

\begin{itemize}
\item tool and function names and indices
\item function signatures
\item parameter names
\item default values
\end{itemize}

\noindent As a result, quite a bit of information can be extracted
with little or no markup included.  Furthermore, this reduces the
amount of redundant information the developer has to type in.  

\item In-line documentation aids fellow programmers in understanding
what the code does when the code is examined directly.

\item It enables a convenient regimen for developing documentation
simultaneously with the development of the code itself.  Otherwise,
documenting the code can seems more like a separate chore, usually
started after the development work is done.  

\item It is easier to keep documentation up-to-date with code changes,
since the relevent information can be kept close to the actual
implementation.  

\item In-line documentation favors a format that is well suited
for a reference manual.  Since in-line documentation appears close to
the thing it is describing (e.g. tool or function definition), the
information tends to aggregate to predictable locations in the
documentation.  This is important to users when they are looking for
specific information.  

\item Well-structured in-line documentation can encourage
completeness.  Tags that mark up such things as default and allowed
values make it more convenient to include full descriptions of these
items.  Furthermore, because some information can be extracted
automatically, such as default values, it can be made more obvious
to the developer when these descriptions are not present.  

\end{itemize}

\subsubsection{Features of the In-line Documentation Framework for
Glish}
\label{gdxml}

For the purposes of discussion, this note will refer to the framework
for in-lining documentation into Glish code as GlishDoc.  An
implementation of this framework as described here may result in a
software tool of the same name.

This section gives an incomplete description of how the proposed
framework for in-lining documentation will work.  An extended example
of documented code is given in Appendix \ref{app-sample}.

\begin{itemize}

\item Developers markup Glish source code files using specialized
markup tags (see below) within Glish comments.  It should be possible
to filter out these comments as part of the \aipspp build system if
they pose a performance problem.  

\item To process the documentation, GlishDoc comments are extracted
from the Glish source code and converted into XML using a custom
extraction tool (implemented in Glish).  

\item \label{pg-gdxml} The XML DTD used for GlishDoc documentation may
either be an extended version of DocBook or a custom DTD for Glish
documentation.  If the former is chosen, it will be necessary to
define a number of tags to describe Glish tool interface components
and characteristics.  In the latter case, we may want to enable a
further conversion to the DocBook DTD in order better to integrate the
reference manual with the rest of the documentation.  A hybrid
solution (e.g. akin to the way DocBook supports MathML) may also be
possible.

\item The GlishDoc extraction tool constructs the basic XML document
primarily from the Glish code itself.  Markup tags added by the
developer further guide and augment the output document's structure and
content.  

\item A GlishDoc comment block has a special format that identifies it
as such.  The GlishDoc comment block starts with a line that contains
only a comment starting with the characters \verb|#@|.  Subsequent
comment lines are included in the block.  The block ends prior to the
first line that contains either Glish code or the start of another
GlishDoc comment block.  For example:

\goodbreak

\begin{verbatim}
#@
# Writes a summary of the properties of the imager to the
# default logger.
public.summary:=function() {
\end{verbatim}

\item Markup tags use the JavaDoc style syntax of the form
\verb|@|{\it tag\_name}.  The marked text appears after the tag and
continues either until the next tag or the end of the block;
interpretation depends on the tag.  This style of tag is quicker to
type and less prone to errors than XML or \LaTeX--style tags because
it requires a minimum of special characters and does not require a
closing element or brace.  For example,

\begin{verbatim}
#@
# Summarize the measurement set.
#
# This function will print a summary of the measurement set to the
# system logger. The verbose argument provides some control on how
# much information is displayed.
# 
# @outparam header  Selected header information returned as a Glish 
#                   record.
# @inparam  verbose If true, produce a verbose summary.
public.summary:=function(ref header=[=], verbose=F) {
\end{verbatim}

\item XML tags can also be used for special types of markup within
descriptions.  This will be useful for including links to other
documents, inserting tables or figures, or including mathematical
formulas.  The GlishDoc extraction tool would pass this markup
unchanged. 

\item Many markup tags can be assumed based on the context of the
GlishDoc comment block.  For instance, the first example above could
also be written as: 

\begin{verbatim}
#@tfunction
# Writes a summary of the properties of the imager to the
# default logger.
public.summary:=function() {
\end{verbatim}

However, the fact that the text describes a tool function can be
assumed because it appears within a tool definition and just before a
function definition.  

\item Among the available GlishDoc tags will be an include directive
(e.g. \verb|@include| {\it filename}).  This will allow one to easily
include longer descriptive sections that may be more easily authored
externally using DocBook markup and LyX.

\item All the functionality provided by the current \LaTeX\ markup
(defined in {\tt aips2help.sty}) would be duplicated in the GlishDoc
framework.  

\item Developers can also include information needed by the Tool
Manager to build GUIs automatically.  Initially, this information
could be transformed from XML into Glish code of the form currently
supported via the {\tt {\it tool}\_meta.g} files.  In the future, this
information could be accessed directly from Glish via a Glish XML parser
client.  

\item Prior to checking in documented code, developers can run it
through a validation tool that will alert them of syntax errors and
warn them about missing markup that is required by the style guide.  

\end{itemize}

\section{Supporting Tools and Documents}

This section lists the tools (in the more generic sense) and documents
that will be needed to support the proposed framework.

\begin{itemize}

\item GlishDoc extraction tool.  It is expected that this will be the
only tool that will have to be developed ``from scratch.''

\item Style sheets for validating XML documents and converting them to
other formats.  It is expected that these would be adapted from
existing style sheets that come with Jade or some similar existing
package.  

\item Style sheet application tools.  These would be trivial scripts
that call an existing program for applying our customized style
sheets.  In particular, we would have:

\begin{itemize}

\item a GlishDoc validation tool.  This would be used by Glish
developers to validate the syntax and completeness of their in-lined
documentation.  

\item a GlishDoc format conversion tool.  This would be used by Glish
developers to preview their Glish tool documentation in various
formats.  It would also be used by the \aipspp\ make system to convert
the documentation into supported formats.  

\item DocBook format conversion tool.  This would be used by both
authors and the \aipspp\ make system to convert the documents into the
various supported formats.  Preferably, this tool would be the same
tool used for converting GlishDoc documentation, differing only in the
style sheets it uses.

\end{itemize}

\item Lyx, a WYSIWYG editor for creating DocBook documents, adapted
for \aipspp\ via a pluggable layout class.  

\item A Note describing how to use GlishDoc tags and tools to document
Glish code.

\item A Note describing how to create expository documents
(e.g. Notes, {\it Gettting Results} chapters, etc.) using DocBook and
LyX.  

\item A Style Guide for GlishDoc documentation.

\end{itemize}

\section{Implementation Plan}

This section describes a phased approach for adopting and implementing
the proposed change to the documentation system.  Because XML is
easily converted into other formats, this framework is well-suited for
gradual adoption.  This is helpful as there are many details that have
to yet to be worked out.  As each phase is completed we will have the
opportunity to re-evaluate the framework in light of the
implementation and tools available to date.  It is expected that the
steps listed below would be turned into development targets.

\paragraph{Phase I: Initial GlishDoc implementation}

\begin{enumerate}

\item Adopt and/or create a DTD for encoding tool descriptions that will 
be extracted from the source code.  Define necessary GlishDoc-style
tags for documenting tool source code.  

\item Implement an initial version of the GlishDoc document extractor
in Glish.  (It is expected that a Glish implementation will be easier
for a typical \aipspp\ developer to maintain than one based on, say,
Perl.  It also allows the possibility for further integration of the
documentation extraction into the \aipspp\ system in the future.)  

\item Develop (or adapt) style sheets to convert XML tool
descriptions into currently supported {\tt .help} and {\tt \_meta.g}
files.  Install wrapper script that applies style sheet
transformations to XML descriptions.  (Augment the EMACS Glish mode to
add keyboard macros for inserting commonly used GlishDoc markup.)

\item Create an \aipspp\ note describing how to use GlishDoc tags to
mark up in-lined documentation and how to use associated tools to
extract the documentation and convert it to currently support forms. 

\item Encourage developers to try in-lining documenation.  Evaluate
current state of framework for ease-of-use, robustness, adaptability
to the \aipspp\ system, and effects on the quality of documentation and
software process.  Adjust plan for later phases accordingly.

\end{enumerate}

\paragraph{Phase II: Initial Use of DocBook}

\begin{enumerate} 

\item Decide on the specific relationship between DocBook and the XML
used by GlishDoc as described in \S\ref{gdxml}, p. \pageref{pg-gdxml} (if
not already determined in Phase I).  

\item Define any extensions to the DocBook DTD needed for supporting
\aipspp\ documents.  Ease of use by authors should be an important
factor in defining new tags.

\item Adapt the LyX DocBook layout class as necessary and helpful for
creating \aipspp\ expository documents.

\item Create (or adapt) style sheets converting DocBook documents into
\LaTeX\ in the traditional \aipspp\ style.  

\item Create a DocBook-formatted template for creating \aipspp\ notes.  

\item Create an \aipspp\ note using DocBook that describes how to use
DocBook, LyX, and associated tools for authoring expository documents
for \aipspp.  

\item Encourage developers to try DocBook and LyX for creating
documenation.  Evaluate current state of tools for ease-of-use,
robustness, adaptability to the \aipspp\ system, and effects on the
quality of documentation.  Adjust plan for last phases accordingly.

\end{enumerate}

\paragraph{Phase III: Final Integration}  

Assuming the first two phases are successful and the framework is
determined to be useful, this phase completes the adoption of the new
framework.  

\begin{enumerate}

\item Adapt existing validation tools to check the syntax and
completeness of GlishDoc documentation.  

\item Create (or adapt) style sheets for used to convert
XML-based documentation into other needed formats, including HTML,
Post\-Script/PDF, and (if necessary) plain text.  (Unless a pre-existing
tool for direct conversion to PostScript or PDF is available, it is
expected that XML will be converted first to \LaTeX\ for subsequent
processing into the hardcopy format.)  

\item Fully integrate tools that can be used by authors for validating
and formatting DocBook and GlishDoc documents into the \aipspp
system.  

\item Adapt the \aipspp\ make system as appropriate to build new
XML-based documentation.

\item Determine how much of the documentation should continue to be
built locally to the \aipspp\ and how much should be preformatted and
delivered to an \aipspp\ site on demand.

\item Convert limited number of existing \aipspp\ documents to DocBook
format.  This conversion should be limited to those documents that
will be edited further into the future.  

\item Convert all existing \LaTeX-formatted Reference Manual documents
to XML (DocBook) documents.  

\item Create a style guide for creating complete, high-quality,
inlined Glish tool documentation.  

\end{enumerate}

\section{Conclusions}

The most important aim of the framework described here is to make it
easier for developers to create and deliver documentation that is
high-quality not only in presentation but in content.  Employing the
widely-used DocBook XML markup has the potential for providing greater
flexibility and reliability in formatting and presenting our
expository documents than \LaTeX.  The use of the LyX editor can
greatly reduce the overhead of learning to use DocBook markup and will
hopefully encourage contributions from people outside the core
development team.  

The proposed system for in-lined documentation of Glish code will make
it much easier to create complete, well-organized descriptions of
tools and functions.  Specifically, the simple markup syntax and the
ability to extract some information automatically from the glish code
itself reduces the amount the developer has to type.  Furthermore,
in-lined documentation encourages the practice of developing
documentation simultaneously with implementation: simple descriptions
can be entered when the tool is first prototyped; details can be added
as the implementation crystalizes; once the interface is set, the
function parameters can be described and the necessary metadata for
the Tool Manager, provided; examples can be added during testing; and
finally, as the tool is tweaked and debugged, the changes can be
reflected right away into the documentation.  The structure can
encourage completeness, and the use of a validation tool can help
ensure that all necessary infomation has been encoded.

Despite our best efforts toward software solutions, there is no cure
for the lazy programmer in all of us.  Thus, the key to making any
documentation framework successful is in the conscientious efforts of
developers.  A style guide is important for advising us on how to make
the best of the system.  At the same time, the documentation system
can make the best of our efforts by providing tools that make it
easier pass our expertise onto users.  

\appendix
\section{Example of In-line Documentation Markup}
\label{app-sample}

The following example is provided to give the flavor of in-line
documentation proposed through this note.  It does not contain all the
tags expected to be supported nor all the documentation that should be
included as a matter of style.  However, it should illustrate the
basic structure of the in-line documentation and how certain
information can be assume by the comments context.  

\begin{verbatim}
pragma include once;

include 'unset.g';
include 'xmlstorage.g';

#@tool 
# Buffer for building an element using "easy" storage.  
# 
# This tool is used for loading XML data into a metadata storage
# object which can be returned with a call to 
# <tmlink>getelementrep</tmlink>.  The storage model supported by this
# buffer is optimized for access to XML data from the application/user 
# level.  XML Elements are stored as Glish records, and their attributes,
# as Glish attributes.  
# 
# @constructor
# create an "easy" element buffer.
# @inparam element  previously built element to edit
#                   @type record
#                   @default unset  create a new element
xmleasyelementbuf := function(ref element=unset) {

    #@toolrec public
    public := _xmlstorage();
    private := [=];

    #@
    # set the name of the element
    # @inparam elname  the element name.  This must begin with a letter
    #                  and should should not end in a number.
    #                  @type string
    #                  @default none
    # @inparam clear   if true, clear all previously added data
    # 
    public.setname := function(elname, clear=T) {
        # ...
    }

    #@ 
    # get the name of the top element in this buffer
    public.getname := function() {
        # ...
    }

    #@
    # set an attribute value
    # @inparam attname   the attribute name.  The name must begin with 
    #                   a letter.
    #                   @type string
    # @inparam value    the value of the attribute
    # @inparam default  if true, this value should be considered a default.
    #                   This can used to affect conversion to XML--e.g. one
    #                   may not want to have attributes with default values
    #                   printed out.
    public.setattribute := function(attname, value='', default=F) {
        # ...
    }

    #@ 
    # add some text to the value of the current element
    # @param  text     the text to add
    #                  @type any
    #                  @default an empty string
    # @param  which    replace the which-th text value, if it exists;
    #                  otherwise, just add it.
    #                  @default 0, which means append a new processing 
    #                           instruction
    # @fail if input text is not a string
    public.addText := function(text='', which=0) {
        # ...
    }

    #@
    # return a the buffer of a child element for editing
    # @param elname   the child element name.  If the element has not
    #                 yet been added, it will be.
    # @param  which    replace the which-th element, if it exists;
    #                  otherwise, just add it.
    # @return tool of type xmleasystorage 
    public.getChildElementBuf := function(elname, which=0) {
        # ...
    }

    #@
    # return a copy of the constructed element record 
    # @param options  a record containing options for formatting the
    #                 element representation.  Currently supported
    #		      options include:
    # <pre>                
    #        cleanup  if true, element content will be adjusted to aid in 
    #                 user access.  All non-element children of the same type 
    #                 (text, comments, or processing instructions) adjacent
    #                 to each other will be gathered into a single array.
    #                 Furthermore, if an element contains only text child 
    #                 nodes, the element's children will be "pulled up"--that 
    #                 id, replaced with a single array containing all the text
    #                 node values.  The default is T.
    #        nopullup a list of elements whose child text nodes should not be
    #                 "pulled up" as described above.  This might contain a 
    #                 list of elements that can but not always contain mixed
    #                 (elements and text) content.  
    # </pre>
    # @return the storage object in the form of a Glish record
    public.getElementRep := function(val options) {
        # ... 
    }


    #@
    # shut down this tool
    public.done := function() {
        # ... 
    }

    # ...

    return ref public;
}
\end{verbatim}
	
    
	




\section{Draft Change Proposals}

