
\externallabels{../../user/Refman}{../../user/Refman/labels.pl}

\section{Introduction}
The Table Query Language (TaQL, rhymes with bagel) makes
it possible to select
rows from an arbitrary table based on the contents of its
columns and keywords. It supports arbitrary complex
expressions including regular expressions and many functions.
TaQL also makes sorting and column selection possible.
\\
The first section explain the syntax and show the options.
The last sections show the interface to TaQL using Glish or C++.
The Glish interface makes it possible to embed Glish
variables and expressions in a TaQL command.

\section{Syntax}
The TaQL syntax is based on SQL. The full command looks like:
\begin{verbatim}
SELECT [column_list] FROM table_list [WHERE expression]
  [ORDERBY [NODUPLICATES] sort_list] [GIVING table|set]
\end{verbatim}
The square brackets shown are not part of the syntax, but indicate
that the clauses enclosed in them are optional. The command is
case-insensitive, but of course case is important in string values and
in names of columns and keywords. Whitespace (blanks and tabs) can
be used at will.
The various clauses have the following meaning:
\begin{description}
  \item[ SELECT ]
       indicates the beginning of the command and is a mandatory verb.
  \item[ column\_list]
       is a comma-separated list of column names which have to be selected
       from the primary table in the table\_list (see below).
       If no column\_list is given, all columns will be selected.
       \\In the future it may be possible to use expressions to create
       a new column based on the contents of other columns.
       Note that for subqueries the GIVING clause offers an
       alternative way of specifying the result making it
       possible to use expressions.
  \item[ \label{TAQL:TABLE_LIST}FROM table\_list ]
       is a comma-separated list and indicates which tables are used in
       the SELECT command. The first table in the list is the primary
       table and is used for all columns in the other clauses.
       Usually only one table is used, so the list consists of only one
       table name. E.g.
       \begin{verbatim}
       SELECT col1,col2 FROM mytable
           WHERE col1>col2
           ORDERBY col1
       \end{verbatim}
       In this example columns \texttt{col1} and \texttt{col2}
       are taken from \texttt{mytable}.
       \\However, it is possible to specify more tables in the table\_list.
       In the WHERE clause these secondary tables can be used to take
       keywords from. E.g.
       \begin{verbatim}
       SELECT FROM mytable,othertable
           WHERE col1>othertable.key
       \end{verbatim}
       As shown in the example above a qualifying name
       can be used in the WHERE
       clause to qualify from which table a field has to be taken.
       When no qualifying name is given, the keyword (or column) is taken
       from the primary table (i.e., the first table in the table\_list).
       This means that qualifying names are only needed in special cases.
       The qualifying name can not contain special characters like a slash.
       Therefore a table\_name needs an explicit shorthand when it contains
       special characters.
       \\The full table\_list syntax is:
       \\
       \texttt{table\_name1 [shorthand1], table\_name2 [shorthand2], etc.}
       \\The shorthand defaults to the table\_name.
       In the following example shorthand \texttt{my} is only used
       for demonstration purposes; it is not really needed.
       \begin{verbatim}
       SELECT FROM mytable my, ~user/othertable other
           WHERE my.col1>other.key
       \end{verbatim}
       Similar to SQL and OQL the shorthand can also be given using
       \texttt{AS} or \texttt{IN}. E.g.
       \begin{verbatim}
       SELECT FROM mytable AS my, other IN ~user/othertable
       \end{verbatim}
       Note that when using \texttt{IN}, the shorthand has to preceed
       the table name. It can be seen as an iterator variable.

       There are two special ways to define a table name:
       \begin{enumerate}
       
       \item
       A table name can also be taken from a keyword in a previously
       specified table. This can be useful in a
       \htmlref{subquery}{TAQL:SUBQUERIES}. The syntax for this is
       the same as that for specifying \htmlref{keywords}{TAQL:KEYWORDS}
       in an expression. E.g.
       \begin{verbatim}
       SELECT FROM mytable tab
           WHERE col1 IN [SELECT subcol FROM tab.col2::key]
       \end{verbatim}
       In this example \texttt{key} is a table keyword of column
       \texttt{col2} in table \texttt{mytable} (note that \texttt{tab}
       is the shorthand for \texttt{mytable} and could be left out).
       \\It can also be used for another table in the main query. E.g.
       \begin{verbatim}
       SELECT FROM mytable, ::key subtab
           WHERE col1 > subtab.key1
       \end{verbatim}
       In this example the keyword \texttt{key1} is taken from the
       subtable given by the table keyword \texttt{key} in the main
       table.
       \\When a keyword is used as the table name, the keyword is
       searched
       in one of the tables previously given. The search starts at
       the current query level and proceeds outwards (i.e., up to the
       main query level). When a shorthand is given, only tables with
       that shorthand are taken into account. When no shorthand is
       given, only primary tables are taken into account.

       \item
       Normally only persistent tables (i.e. tables on disk) can
       be used. However, it is also possible to use transient tables
       in a TaQL command given in \htmlref{Glish or C++}{TAQL:GLISHC}.
       This done by passing one or more  table objects to the
       function executing the TaQL command. In the TaQL command a
       \$-sign followed by a sequence number has to be given to
       indicate the correct object containing the transient table.
       E.g. when two
       table objects are passed \$1 indicates the first table, while \$2
       indicates the second one.
     \end{enumerate}

  \item[WHERE expression]
       defines the selection expression which must have a boolean
       scalar result. A row in the primary table
       is selected when the expression is true for the values in that row.
       The syntax of the expression is explained
       \htmlref{below}{TAQL:EXPRESSIONS}.
  \item[ORDERBY sort\_list]
       defines the order in which the result of the WHERE clause
       has to be sorted. The sort\_list is a comma separated list of
       expressions, where each expression is
       optionally followed by \texttt{ASC} or \texttt{DESC}
       (indicating ascending or descending). The default for each
       expression is ascending.
       \\To be compliant with SQL whitespace can be used between the
       words ORDER and BY.
       \\The word ORDERBY can optionally be followed by NODUPLICATES
       which means that only the first row of multiple rows with
       equal sort keys is kept in the result.
       \\An expression can be simply a scalar column or an element from
       an array column. In these cases some optimization is performed
       by reading the entire column directly.
       \\It can also be an arbitrarily complex expression
       with exactly the same syntax rules as the expressions in the
       \htmlref{WHERE}{TAQL:EXPRESSIONS} clause.
       The resulting data type of the expression must
       be a standard scalar one, thus it cannot be a Regex or
       DateTime (see \htmlref{below}{TAQL:DATATYPES} for a discussion
       of the available data types).
       E.g.
       \begin{verbatim}
       ORDERBY col1, col2, col3
       ORDERBY NODUPLICATES uvw[1] DESC
       ORDERBY square(uvw[1]) + square(uvw[2])
       ORDERBY datetime(col)             incorrect data type
       ORDERBY mjd(datetime(col))        is correct
       \end{verbatim}
  \item[ \label{TAQL:GIVING}GIVING table ]
       indicates that the ultimate result of the SELECT command can be
       written to a table (with the given name). This table is a
       so-called reference table.
       It does not contain data of its own, but only references the
       proper rows in the original table. It can be used
       as any other table, thus it can also be used as an input table in
       another SELECT command. A reference table resulting from
       a selection (or sort) on another reference table references the
       original table again. Thus no (expensive) cascade of reference
       tables is created.
       \\It is also possible to specify a set instead of a table name.
       This can be useful when the result of a
       \htmlref{subquery}{TAQL:SUBQUERIES} is used in the main query.
       Such a \htmlref{set}{TAQL:SETS} can contain only one element.
       The parts of this element can be any expression resulting in a scalar.
\end{description}
Although the clauses column\_list, WHERE, and ORDERBY are optional,
at least one of them has to be used. Otherwise no operation is
performed on the primary table (which makes no sense).


\section{\label{TAQL:EXPRESSIONS}Expressions}
An arbitrary expression can be used in the WHERE clause, as a sort
key in the ORDERBY clause, or in the set in the GIVING clause.
Note that the expression result must be
a boolean scalar when used in the WHERE clause. When used in ORDERBY
the result can also be a numeric or string scalar. When used in 
GIVING it can also be a DateTime scalar.
\\The expression in the clause can be as complex as one likes
using the standard
\htmlref{arithmetic, comparison, and logical operators}{TAQL:OPERATORS}.
Parentheses can be used to group subexpressions.
\\The operands in an expression can be
\htmlref{table columns}{TAQL:COLUMNS},
\htmlref{table keywords}{TAQL:KEYWORDS},
\htmlref{constants}{TAQL:CONSTANTS},
\htmlref{functions}{TAQL:FUNCTIONS},
\htmlref{sets}{TAQL:SETS}, and
\htmlref{subqueries}{TAQL:SUBQUERIES}.
\\The \htmlref{index operator}{TAQL:INDEXING} can be used to take a
single element or a subsection from an array expression.
E.g.
\begin{verbatim}
  column1 > 10
  column1 + arraycolumn[index] >= min (column2, column3)
  column1 IN [expr1 =:< expr2]
\end{verbatim}
The last example shows a \htmlref{set}{TAQL:SETS} with a continuous interval.

\label{TAQL:DATATYPES}
TaQL knows the following data types:
\begin{description}
  \item[ Bool]
  \item[ Double] which includes integers and times/positions
  \item[ Complex ] which includes single and double precision complex
  \item[ String ] Operator + can be used (concatenation).
  \item[ Regex ] which is formed by the functions \texttt{regex}
    and \texttt{pattern} (see below).
  \item[ DateTime ] which represents a date/time. There are several functions
       acting on a date/time. Also operator + and - can be used.
\end{description}
All these data types can be used for scalars and, except for Regex,
they can also be used for arrays.
\\When an operand or function argument with a non-matching data type
is used TaQL can do the following automatic conversions:
\\- from Double to Complex
\\- from String to DateTime


\subsection{\label{TAQL:CONSTANTS}Constants}
Scalar constants of the various data types can be formed as follows
(which is very similar to Glish):
\begin{itemize}
  \item A Bool constant is not needed and therefore does not exist.
  \item A Double constant can be any integer or floating-point number.
       \\Another way to define a Double constant is by means of
       a Time or Position. Such a constant is always converted to radians.
       It can be given in several ways:
       \begin{itemize}
         \item An integer or floating-point number immediately
              followed by a unit
              (thus without whitespace). E.g. \texttt{12deg}
              \\Some valid units are deg, arcmin, arcsec (or as), rad.
              The units can be scaled by preceeding them with a letter
              (e.g. mrad is millirad).
         \item A time/position in HMS format.
              E.g. \texttt{12h34m34.5}
         \item A position in DMS format.
              E.g. \texttt{12d34m34.5}
         \item A dot-separated position.
              E.g. \texttt{12.34.34.5}
       \end{itemize}
  \item The imaginary part of a DComplex constant is formed by a Double
       constant immediately followed by a lowercase \textbf{i}. A full DComplex
       constant is formed by adding another Double constant as the
       real part. E.g.
       \begin{verbatim}
       1.5 + 2i
       2i+1.5            is identical
       \end{verbatim}
       Note that a full Complex constant has to be enclosed
       in parentheses when, say, a multiplication is performed on it. E.g.
       \begin{verbatim}
       2 * (1.5+2i)
       \end{verbatim}
  \item A String constant has to be enclosed in " or ' and can be
       concatenated (as in C++). E.g.
       \begin{verbatim}
         "this is a string constant"
         'this is a string constant containing a "'
         "ab'cd"'ef"gh'
             which results in constant  ab'cdef"gh
       \end{verbatim}
  \item A Regex constant as such does not exist. A Regex is formed from a String
       by the function \texttt{regex} or \texttt{pattern} (see below).
       In this way one can form a Regex constant. E.g.
       \begin{verbatim}
         regex ("[a-zA-Z][a-zA-Z0-9]*")
       \end{verbatim}
  \item A DateTime constant can be formed in 2 ways:
       \begin{enumerate}
         \item From a String constant using the \texttt{datetime} function.
              In this way all possible formats as explained in class
              \texttt{MVTime} are supported. E.g.
              \begin{verbatim}
              datetime ("11-Dec-1972")
              \end{verbatim}
         \item A more convenient way is to specify it directly. Since this
              makes use of the delimiters - or /, it conflicts with the
              expression grammar as such. However, such conflicts can be
              solved by using whitespace in a expression and it is believed
              that in practice the convenience surpasses the possible
              conflicts.
              \\A large subset of the MVTime formats is supported.
              A DateTime has to be specified as \texttt{date/time}
              or \texttt{date-time}, where the time part (including
              the / or - delimiter) is optional.
              The possible date formats are:
              \\- YYYY/MM/DD or DD-MM-YYYY
              \\- DD-MMMMMMMM-YY where the - is optional and MMMMMMM is the
              case-insensitive name of the month (at least 3 letters).
              \\- YYYY//DDD or DDD--YYYY where DDD is the day number in
              the year.
              \\In the DMY format, 2000 is added when year$<$50 and
              1900 is added when 50$<=$year$<$100.
              \\When MM$>$12, YYYY will be incremented accordingly.
              \\The general time format in a DateTime constant is:
              \\- hh:mm:ss.s
              \\where the delimiter \textbf{h} or \textbf{H} can be used
              for the first colon and \textbf{m} or \textbf{M} for the second.
              Trailing parts can be omitted. E.g.
              \begin{verbatim}
              10-2-97
              10-02-1997
              10-February-97
              10feb97
              1997/2/10          are all identical

              1May96/3:          : (or h) is mandatory
              1May96/3:0
              1May96/3:0:0
              1May96/3h          h (or :) is mandatory
              1May96/3H0
              1May96/3h0M
              1May96/3hm0.0
              \end{verbatim}
              A DateTime constant with the current date/time can be made
              by using the function \texttt{datetime} without arguments.
       \end{enumerate}
\end{itemize}
Constant arrays cannot be formed, but it is possible to form a
constant set. Such a set is transformed to a 1D array and can be used
as such.
\\Whenever needed it is easy to implement a Glish-like array function
to construct a constant array.

\subsection{\label{TAQL:OPERATORS}Operators}
The following operators can be used (with their normal meaning and
precedence) for scalars:
\begin{description}
  \item[] Unary \textbf{+} and \textbf{-}
       \\Can only be used with double and complex operands.
       They have a higher precedence than the binary operators.
       E.g. \texttt{-3}\verb+^+\texttt{2} results in \texttt{9}.
  \item[] Unary \textbf{!} (or \textbf{NOT})
       \\Logical NOT operator.
       Can only be used with Bool operands.
%  \item[ Binary \verb+^+, *, /, \%, +, and -] 
  \item[] Binary \textbf{$\wedge$, *, /, \%, +}, and \textbf{-}
       \\\% is the modulo operator.
       E.g. \texttt{3\%1.4} results in \texttt{0.2} and
       \texttt{-10\%3} results in \texttt{-1}.
       \\\verb+^+ is the power operator.
       Note that because of the precedence rules
       \texttt{-3}\verb+^+\texttt{2} results in \texttt{9}.
       \\All operators are left-associative, except \verb+^+ which is
       right-associative; thus \texttt{2}\verb+^+\texttt{1}\verb+^+\texttt{2} results in \texttt{2}.
       \\Operator \% can only be used for double operands, while the others
       can be used for double and complex operands.
       Operator + can also be used for:
       \\- Addition (i.e., concatenation) of 2 strings.
       \\- Addition of a double (unit days) and a DateTime
       (resulting in a DateTime).
       \\Operator - can also be used for:
       \\- Subtraction of a double (unit days) from a DateTime
       (resulting in a DateTime).
       \\- Subtraction of 2 DateTimes (resulting in a double with unit days).
  \item[] \textbf{==, $!=$, $>$, $>=$, $<$}, and \textbf{$<=$}
       \\Can be used with any operand as long as their data types conform.
       Operator $>$, $>=$, $<$, and $<=$ cannot be used for Regex.
       They use the norm for Complex values.
       \\To be compliant with SQL $=$ can be used for
       $==$ and $<>$ can be used for $!=$.
  \item[] \textbf{\&\&} and \textbf{$\mid\mid$}
       \\Logical AND and OR operator. 
       \\To be compliant with SQL the
       words \texttt{AND} and \texttt{OR} can be used instead of the
       symbols. These words are case insensitive.
       \\Also \textbf{\&} and \textbf{$\mid$} can be used.
       \\These operators can only be used with Bool operands.
\end{description}
\begin{description}
  \item The precedence order is:
       \\unary \texttt{+, -, !}
       \\\verb+^+
       \\\texttt{*, /, \%}
       \\\texttt{+, -}
       \\\texttt{$==, !=, >, >=, <, <=$}
       \\\texttt{\&\&}
       \\\texttt{$\mid\mid$}
\end{description}
All these operators can be used in the same way for arrays
(also a mix of scalar and array).

\subsection{\label{TAQL:FUNCTIONS}Functions}
Some functions have 2 names. One name is the AIPS++/Glish name, while the
other is the name as used in SQL.
In the following tables the function names are shown in uppercase,
while the result and argument types are shown in lowercase.
Note, however, that function names are case-insensitive.

\htmlref{Sets}{TAQL:SETS}, and in particular
\htmlref{subqueries}{TAQL:SUBQUERIES}, can result in an array.
This means that the functions accepting an array argument can also
be used on a set or the result of a subquery. E.g.
\begin{verbatim}
  WHERE datecol IN date([11-Nov-97,16-Dec-97,14-Jan-98])
\end{verbatim}
\subsubsection{String functions}
These functions can be used on a scalar or an array argument.
\begin{description}
  \item[ \texttt{double STRLENGTH(string),  double LEN(string)}]
       Returns the number of characters in a string
       (trailing whitespace is significant).
  \item[ \texttt{string UPCASE(string), string UPPER(string) }]
        Convert to uppercase.
  \item[ \texttt{string DOWNCASE(string),  string LOWER(string)}]
        Convert to lowercase.
  \item[ \texttt{string TRIM(string)}]
       Removes trailing whitespace.
\end{description}

\subsubsection{Regex functions}
The syntax for forming a regex and pattern are explained in class
\htmladdnormallink{Regex}{../../aips/implement/Utilities/Regex.html}.
These functions can only be used on a scalar arguments.

\begin{description}
  \item[ \texttt{regex REGEX(string)}]
       Handle the given string as a regular expression.
  \item[ \texttt{regex PATTERN(string)}]
       Handle the given string as a filename-like pattern and
       convert it to a regular expression.
\end{description}
A regex can only be used in a comparison == or !=. E.g.
\\\texttt{upcase(object) == pattern('3C*')}
\\to find all 3C objects in a catalogue.

\subsubsection{Date/time functions}
These functions can be used on a scalar or an array argument.
\begin{description}
  \item[ \texttt{DateTime DATETIME(string),  DateTime CTOD(string)} ]
       Parse the string and convert it to a date/time. The syntax of
       date/time is explained in class
       \htmladdnormallink{MVTime}{../../aips/implement/Measures/VelocityMachine.html}.
  \item[ \texttt{DateTime MJDTODATE(double)} ]
       The double value, which has to be a MJD (ModifiedJulianDate), is
       converted to a DateTime.
  \item[ \texttt{DateTime DATE(DateTime)}]
        Get the date (i.e., remove the time part). This function is needed in
       \\\texttt{date(column) == 12Feb1997}
       \\when the column contains date/times with times$>$0.
  \item[ \texttt{double MJD(DateTime)}]
        Get the DateTime as a MJD (ModifiedJulianDate).
  \item[ \texttt{double YEAR(DateTime)}]
        Get the year (which includes the century).
  \item[ \texttt{double MONTH(DateTime)}]
        Get the month number (1-12).
  \item[ \texttt{double DAY(DateTime)}]
        Get the day number (1-31).
  \item[ \texttt{double WEEKDAY(DateTime),  double DOW(DateTime)}]
        Get the weekday number (0=Sunday, ..., 6=Saturday).
  \item[ \texttt{string CMONTH(DateTime)}]
        Get the name of the month (Jan ... Dec).
  \item[ \texttt{string CWEEKDAY(DateTime),  string CDOW(DateTime)}]
        Get the name of the weekday (Sun .. Sat).
  \item[ \texttt{double WEEK(DateTime)}]
        Get the week number in the year (0 ... 52).
  \item[ \texttt{double TIME(DateTime)}]
       Get the time part of the day. It is converted to radians to
       be compatible with the internal representation of times/positions.
       In that way the function can easily be used as in:
       \\\texttt{TIME(date) $>$ 12h}
\end{description}
All functions can be used without an argument in which case the current
date/time is used. E.g. \texttt{DATE()} results in the current date.
\\It is possible to give a string argument instead of a date. In this
case the string is parsed and converted to a date (i.e., in fact the
function DATETIME is used implicitly).

\subsubsection{Comparison functions}
Two functions make it possible to compare 2 values within a range.
They can only be used on scalar arguments.
\begin{description}
  \item[ \texttt{bool NEAR(numeric val1, numeric val2, double tol)}]
    Tests in a relative way if a value is near another. Relative
    means that the
    magnitude of the numbers is taken into account.
    Returns \texttt{abs(val2 - val1)/max(abs(val1),abs(val2)) < tol}.
    If \texttt{tol<=0}, returns \texttt{val1==val2}.
    If either val is 0.0, takes
    care of area around the minimum number that can be represented.
    The default tolerance is 1.0e-13.
  \item[ \texttt{bool NEARABS(numeric val1, numeric val2, double tol)}]
    Tests in an absolute way if a value is near another. Absolute
    means that the
    magnitude of the numbers is not taken into account.
    Returns \texttt{abs(val2 - val1) < tol}.
    The default tolerance is 1.0e-13.
\end{description}

\subsubsection{Mathematical functions}
Several functions can operate on double or complex arguments.
The data types of such functions is given as 'numeric'.
They can only be used on scalar arguments.
\begin{description}
  \item[ \texttt{double PI()}] Returns the value of \textbf{pi}.
  \item[ \texttt{double E()}] Returns the value of \textbf{e} (is equal to \texttt{EXP(1)}).
  \item[ \texttt{numeric SIN(numeric)}]
  \item[ \texttt{numeric SINH(numeric)}]
  \item[ \texttt{double ASIN(double)}]
  \item[ \texttt{numeric COS(numeric)}]
  \item[ \texttt{numeric COSH(numeric)}]
  \item[ \texttt{double ACOS(double)}]
  \item[ \texttt{double TAN(double)}]
  \item[ \texttt{double TANH(double)}]
  \item[ \texttt{double ATAN(double)}]
  \item[ \texttt{double ATAN2(double y, double x)}]
       Returns \texttt{ATAN(y/x)} in correct quadrant.
  \item[ \texttt{numeric EXP(numeric)}]
  \item[ \texttt{numeric LOG(numeric)}] Natural logarithm.
  \item[ \texttt{numeric LOG10(numeric)}]
  \item[ \texttt{numeric POW(numeric, numeric)}] The same as operator \verb+^+.
  \item[ \texttt{numeric SQUARE(numeric)}] The same as \verb+^+2, but much faster.
  \item[ \texttt{numeric SQRT(numeric)}]
  \item[ \texttt{complex COMPLEX(double, double)}]
  \item[ \texttt{numeric CONJ(numeric)}]
  \item[ \texttt{double REAL(numeric)}] Real part of a complex number.
  \item[ \texttt{double IMAG(numeric)}] Imaginary part of a complex number.
  \item[ \texttt{double NORM(numeric)}]
  \item[ \texttt{double ABS(numeric),  double AMPLITUDE(numeric)}]
  \item[ \texttt{double ARG(numeric),  double PHASE(numeric)}]
  \item[ \texttt{numeric MIN(numeric, numeric)}]
  \item[ \texttt{numeric MAX(numeric, numeric)}]
  \item[ \texttt{double SIGN(double)}]
       Returns -1 for a negative value, 0 for zero, 1 for a positive value.
  \item[ \texttt{double ROUND(double)}]
       Rounds the absolute value of the number.
       E.g. \texttt{ROUND(-1.6) = -2}.
  \item[ \texttt{double FLOOR(double)}]
       Works towards negative infinity.
       E.g. \texttt{FLOOR(-1.2) = -2}
  \item[ \texttt{double CEIL(double)}] Works towards positive infinity.
  \item[ \texttt{double FMOD(double, double)}] The same as operator \%.
\end{description}
Note that the trigonometric functions need their arguments in radians.

\subsubsection{Array functions}
These functions can only be used on arrays.
\begin{description}
  \item[ \texttt{bool ANY(boolarray)}] Is any element in array true?
  \item[ \texttt{bool ALL(boolarray)}] Are all elements in array true?
  \item[ \texttt{double NTRUE(boolarray)}] Number of true elements in array.
  \item[ \texttt{double NFALSE(boolarray)}] Number of false elements in array.
  \item[ \texttt{numeric SUM(numericarray)}] Return sum of all array elements.
  \item[ \texttt{numeric PRODUCT(numericarray)}] Return product
    of all array elements.
  \item[ \texttt{double MIN(doublearray)}] Return minimum
    of all array elements.
  \item[ \texttt{double MAX(doublearray)}] Return maximum
    of all array elements.
  \item[ \texttt{double MEAN(doublearray), double AVG(doublearray)}]
    Return mean of all array elements.
  \item[ \texttt{double VARIANCE(doublearray)}] Return variance
    (the sum of \texttt{(a(i) - mean(a))**2/(nelements(a) - 1)}.
  \item[ \texttt{double STDDEV(doublearray)}] Return standard
    deviation (the square root of the variance).
  \item[ \texttt{double AVDEV(doublearray)}] Return average deviation.
    (the sum of \texttt{abs(a(i) - mean(a))/nelements(a)}.
  \item[ \texttt{double MEDIAN(doublearray)}] Return median (the
    middle element).
    When the array has an even number of elements, the mean of
    the two middle elements is returned.
\end{description}

\subsubsection{Miscellaneous functions}
\begin{description}
  \item[ \texttt{bool ISDEFINED(anytype)}]
    Return false if the value in the current row is undefined. Is
    useful to test if a cell in a column with variable shaped arrays
    contains an array.
  \item[ \texttt{double NELEMENTS(anytype), double COUNT(anytype)}]
    Return number of elements in an array (1 for a scalar).
  \item[ \texttt{double NDIM(anytype)}]
    Return dimensionality of an array (0 for a scalar).
  \item[ \texttt{doublearray SHAPE(anytype)}]
    Return shape of an array (returns an empty array for a scalar).
  \item[ \texttt{double ROWNUMBER()}]
       Return the row number being tested (first row is row number 1).
       \\This can, for instance, be used to select the first N rows
       of a sorted table to get the rows with the highest values).
  \item[ \texttt{double RAND()}]
       Return (per table row) a uniformly distributed random number
       between 0 and 1 using a Multiplicative Linear Congruential Generator.
       The seeds for the generator are deduced from the current date and
       time, so the results are different from run to run.
       \\The function can, for instance, be used to select a random
       subset from a table.
\end{description}

\subsection{\label{TAQL:COLUMNS}Table Columns}
A column can contain a scalar or an array value.
Note that only columns in the primary table can be handled directly.
A column in another table can be used via a subquery. E.g.
\begin{verbatim}
  SELECT FROM tab WHERE col >
        mean([SELECT othercol FROM othertab])
\end{verbatim}
An expression has to contain at least one column, since columns
are the only variable part in it. That is, a row can only be selected
or sorted by means of the column values in each row.

The name of a column can contain alphanumeric characters and underscores.
It should start with an alphabetic character or underscore.
A column name is case-sensitive.
\\It is possible to use other characters in the name by
escaping them with a backslash. E.g. \texttt{DATE}\verb+\_+\texttt{OBS}.
\\In the same way a numeric character can be used as the first
character of the column name. E.g. \verb+\+\texttt{1stDay}.
\\
Because several words are used in the language, they cannot
be used directly as column names. The reserved words are:
\begin{verbatim}
   AND AS ASC DESC FROM GIVING IN
   NODUPLICATES NOT OR ORDERBY
   SELECT WHERE
\end{verbatim}
They can, however, be used as a column name by escaping
them with a backslash. E.g. \verb+\+\texttt{IN}.
\\Note that in C++ and Glish a backslash itself has to be escaped
by another backslash. E.g. in Glish:
\texttt{tab.query('DATE}\verb+\\_+\texttt{OBS$>$10MAR1996')}.

When a column contains a record, one has to specify a field in it
using the dot operator; e.g. \texttt{col.fld} means use field
\texttt{fld} in the column. It is fully recursive, so
\texttt{col.fld.subfld} can be used if field \texttt{fld} is a record
in itself.
\\Alas records in columns are not really supported yet. One can specify
fields, but thereafter an error message will be given.

\subsection{\label{TAQL:KEYWORDS}Table Keywords}
It is possible to use table or column keywords, which can have
a scalar or an array value. A table keyword has to be specified
as \texttt{::key}. In an expression the \texttt{::} part can be omitted
when there is no column with such a name.
A column keyword has to be specified as \texttt{column::key}.
\\Note that \texttt{::} is similar to the scope operator in C++.
\\
As explained in the \htmlref{FROM clause}{TAQL:TABLE_LIST} of the syntax
section, keywords from the primary table and from secondary tables
can be used. When used from a secondary table, it has to be qualified
with the (shorthand) name of the table. E.g.
\\\texttt{sh.key} or \texttt{sh.::key}
\\takes table keyword \texttt{key} from the table with the shorthand name
\texttt{sh}.

When a keyword is a record in itself, it is possible to use
a field in it using the dot operator. E.g. \texttt{::key.fld}
to use field \texttt{fld}. It is fully recursive, so when the
field is a record in itself, a subfield can be used like
\texttt{col::key.fld.subfld}

A keyword can be used in any expression. It is evaluated immediately
and transformed to a constant value.

\subsection{\label{TAQL:INDEXING}Array Index Operator}
It is possible to take a subsection or a
single element from an array column, keyword or expression
using the index operator
\texttt{[index1,index2,...]}. The rules for this
are similar to those used in Glish.
Taking a single element can be done as:
\begin{verbatim}
  array[1]
  array[1, some_expression]
\end{verbatim}
Taking a subsection can be done as:
\begin{verbatim}
  array[start1:end1:incr1, start2:end2:incr2, ...]
\end{verbatim}
When a start value is left out it defaults to the beginning of
that axis. An end value defaults to the end of the axis and
an increment defaults to one. When an entire axis is left out,
it defaults to the entire axis.
\\E.g. an array with shape [10,15,20] can be subsectioned as:
\begin{verbatim}
  [,,3]              resulting in an array [10,15,1]
  [2:4, ::3, 2:15:2  resulting in an array [3,5,7]
\end{verbatim}
The examples show that an index can be a simple constant (as it will
usually be), but it can also be an expression which can be as complex
as one likes. The expression has to result in a double value.
\textbf{Note} that as in Glish, array indices start at 1.
\\For fixed shaped arrays checking if array bounds are exceeded
is done at parse time.
For variable shaped arrays
it can only be done per row. If array bounds are exceeded,
an exception is thrown. In the future a special undefined value
will be assigned when bounds of variable shaped arrays are exceeded
to prevent the selection process from aborting due to the exception.

Note that the index operator will be applied directly
to a column. This results in reading only that part of the
array from the table column on disk.
It is, however, also possible to apply it to a
subexpression (enclosed in parentheses) resulting in an array.
E.g.
\begin{verbatim}
  arraycolumn[2,3,4] + 1
  (arraycolumn + 1)[2,3,4]
\end{verbatim}
can both be used and have the same result. However, the first
form is much faster, because only a single element is read
(resulting in a scalar) and 1 is added to it.
The second form results in reading the entire array.
1 is added to all elements and only then the requested element is taken.
\\From this example it should be clear that indexing an array
expression has to be done with care.

\subsection{\label{TAQL:SETS}Sets}
As in SQL the operator \texttt{IN} can be used to do a selection
based on a set or array. E.g.
\begin{verbatim}
  SELECT FROM table WHERE column in [expr1, expr2, expr3]
\end{verbatim}
This example shows that (in its simplest form) a set
consists of one or more values (which
can be arbitrary expressions) separated by commas and enclosed in
square brackets. The elements in a set have to be scalars and their
data types have to be the same.
\\An element in a set can, however, be more complicated and can
define multiple values or an interval. The possible forms of
a set element are:
\begin{enumerate}
\item A single value as shown in the example above.
\item \texttt{start:end:incr}. This is similar to the
way an array index is specified. Incr defaults to 1.
End defaults to an open end (i.e., no upper bound) and results
in an unbounded set. Start and end can be a double or a datetime.
Incr has to be a double. Some examples:
\begin{verbatim}
  1:10     means 1,2,...,10
  1:10:2   means 1,3,5,7,9
  1::2     means all odd numbers
  1:       means all positive integer numbers
  date('18Aug97')::2   means every other day from 18Aug97 on
\end{verbatim}
These examples show constants only, but start, end, and incr can
be any expression.
\\Note that :: used here can conflict with the :: in the
\htmlref{keywords}{TAQL:KEYWORDS}. E.g. \texttt{a::b} is scanned as
a keyword specification. When the intension is \texttt{start::incr}
it should be specified as \texttt{a: :b}. In practice this conflict
will hardly ever occur.
\item Continuous intervals can be specified for double, string, and datetime.
The specification of an interval resembles the mathematical notation
\texttt{1<x<5}, where x is replaced by :. An open interval side
is indicated by \texttt{<}, while a closed interval side is indicated
by \texttt{=}.
\\Another way to specify intervals is using curly and/or angle brackets.
A curly bracket is a closed side, the angle bracket is an open side.
The following examples show how bounded and half-bounded,
(half-)open and closed intervals can be specified.
\begin{verbatim}
  1=:=5   {1,5}     means 1<=x<=5   bounded closed
  1<:<5   <1,5>     means 1<x<5     bounded open
  1=:<5   {1,5>     means 1<=x<5    bounded right-open
  1<:=5   <1,5}     means 1<x<=5    bounded left-open
  1=:  {1,}  {1,>   means 1<=x      left-bounded closed
  1<:  <1,}  <1,>   means 1<x       left-bounded open
  :=5  {,5}  <,5}   means x<=5      right-bounded closed
  :<5  {,5>  <,5>   means x<5       right-bounded open
\end{verbatim}
\end{enumerate}
Each element in a set can have its own form, i.e., one element can
be a single value while another can be an interval.
When a set consists of single or bounded
\texttt{start:end:incr} values only, the set will be expanded to an
array.
This makes it possible for array operators and functions
(like \texttt{mean}) to be applied to such sets. E.g.
\begin{verbatim}
  WHERE column > mean([10,30:100:5])
\end{verbatim}
It is very important to note that the 2nd form results in
discrete values, while the 3rd form results in a continuous interval.

Another form of constructing a set is the
\htmlref{subquery}{TAQL:SUBQUERIES} shown below.

\subsection{\label{TAQL:SUBQUERIES}Subqueries}
As in SQL it is possible to create a set from a subquery. A
subquery has the same syntax as a main query, but has to be
enclosed in square brackets. Basically it looks like:
\begin{verbatim}
  SELECT FROM maintable WHERE time IN
      [SELECT time FROM othertable WHERE windspeed < 5]
\end{verbatim}
The subquery on the other table results in a constant set containing the times
for which the windspeed is okay. Subsequently the main query
is executed and selects all rows from the main table with times in
that set.
Note that like other bounded sets this set is transformed to a
constant array, so it is possible to apply functions to it (e.g. min, mean).
\begin{verbatim}
  SELECT FROM maintable WHERE time IN
      [SELECT time FROM othertable WHERE windspeed <
           mean([SELECT windspeed FROM othertable])]
\end{verbatim}
This contains another subquery to get all windspeeds and
to take the mean of them. So the first subquery selects all times
where the windspeed is less than the average windspeed.
\\A subquery result should contain only one column, otherwise
an exception is thrown.

It may happen that a subquery has to be executed twice because
2 columns from the other table are needed. E.g.
\begin{verbatim}
  SELECT FROM maintable WHERE any(time >=
      [SELECT starttime FROM othertable WHERE windspeed < 5]
                               && time <=
      [SELECT endtime FROM othertable WHERE windspeed < 5])
\end{verbatim}
In this case the other table contains the time range for each windspeed.
For big tables it is expensive to execute the subquery twice.
A better solution
is to store the result of the subquery in a temporary table and reuse it.
\begin{verbatim}
  SELECT FROM othertable WHERE windspeed < 5 GIVING tmptab
  SELECT FROM maintable WHERE any(time >=
      [SELECT starttime FROM tmptab]
                               && time <=
      [SELECT endtime FROM tmptab]
\end{verbatim}
However, this has the disadvantage that the table \texttt{tmptab}
still exists after the query and has to be deleted explictly by the
user. Below a better solution for this problem is shown.

TaQL has two extensions to support tables better,
in particular the AIPS++ measurement sets.
\begin{enumerate}
\item
The time range problem above can be solved elegantly by using
a set as the result of the subquery. Instead of a table name
it is possible to define a set in the GIVING clause (as mentioned
in the \htmlref{syntax section}{TAQL:GIVING}). E.g.
\begin{verbatim}
  SELECT FROM maintable WHERE time IN
      [SELECT starttime FROM othertable WHERE windspeed < 5
                                   GIVING [starttime=:=endtime]]
\end{verbatim}
The set expression in the GIVING clause is filled with the
results from the subquery and used in the main query. So when
the subquery results in 5 rows, the resulting set contains 5
intervals.
\item
In AIPS++ the other table will often be the name of a subtable,
which is stored in a table or column keyword of the main table.
The standard \htmlref{keyword syntax}{TAQL:KEYWORDS} can be used
to indicate that the other table is the table in the given keyword.
Note that for a table keyword the \texttt{::} part has to be given,
otherwise the name is scanned as an ordinary table name. E.g.
\begin{verbatim}
  SELECT FROM my.ms WHERE time IN
      [SELECT time FROM weather::key WHERE windspeed < 5]
\end{verbatim}
In this example the other table is a subtable of table \texttt{my.ms}.
Its name is stored in keyword \texttt{key} of column
\texttt{weather}.
\end{enumerate}


\section{Some further remarks}
\subsection{Time/position considerations}
A position constant (e.g. \texttt{3h4m}) is converted to
radians, so it can be used easily in functions like \texttt{SIN}.
Since a time constant has exactly the same format, it is
also converted to radians, while the user may expect it to be
expressed in seconds. The user has to take this into account
when times are used in a comparison. For example, \texttt{timecol $>$ 3h4m}
is only correct when \texttt{timecol} has unit radians.
\\To make life easier the function \texttt{TIME} results in a value
in radians, so it can be used directly in a comparison. For example,
\texttt{TIME(datecolumn) $>$ 3h4m}.
\\In the future this may change when units are handled in their full glory.

\subsection{Optimization}
There remains a lot of work to be done on improving query optimization.
At this stage the only optimization done is the precalculation of constant
subexpressions.
\\
However, the user can optimize a query by specifying the expression
carefully. Especially when using operator $\mid\mid$ or \&\&,
attention should be
paid to the contents of the left and right branches. Both operators
evaluate the right branch only when needed, so the left branch
should be the shortest one, i.e., the fastest to evaluate.
\\In the future the selection process may do automatic optimization
by swapping the branches. This means that one should not count on
the order of evaluation (as you can do in the C(++) language).

The user should also use functions and operators in a careful way.
\begin{itemize}

\item
\texttt{SQUARE(COL)} is (much) faster than \texttt{COL}\verb+^+\texttt{2}
or \texttt{POW(COL,2)}, because SQUARE is faster.
It is also faster than \texttt{COL*COL}, because it accesses column
\texttt{COL} only once.
\\Similarly \texttt{SQRT(COL)} is faster than \texttt{COL}\verb+^+\texttt{0.5}
or \texttt{POW(COL,0.5)}

\item
\texttt{SQUARE(U) + SQUARE(V) $<$ 1000}\verb+^+\texttt{2} is considerably faster
than
\\\texttt{SQRT(SQUARE(U) + SQUARE(V)) $<$ 1000}, because it avoids the
\texttt{SQRT} function.

\item
\texttt{TIME IN [$0<:<4$]} is faster than
\texttt{TIME$>$0 \&\& TIME$<$4}, because in the first way the column is
accessed only once.

\item
Returning a column from a subquery can be done directly or as a
set. E.g.
\begin{verbatim}
  SELECT FROM maintable WHERE time IN
      [SELECT time FROM othertable WHERE windspeed < 5]
\end{verbatim}
could also be expressed as
\begin{verbatim}
  SELECT FROM maintable WHERE time IN
      [SELECT FROM othertable WHERE windspeed < 5 GIVING [time]]
\end{verbatim}
The latter (as a set) is slower. So, if possible, the column should
be returned directly. This is also easier to write.

\item
Sometimes operator \texttt{IN} and function \texttt{ANY} can be used to test
if an element in an array matches a value. E.g.
\begin{verbatim}
  WHERE any(arraycolumn == value)
and
  WHERE value IN arraycolumn
\end{verbatim}
give the same result.
Operator \texttt{IN} is faster because it stops when it finds a
match. When using \texttt{ANY} all elements are compared first and thereafter
\texttt{ANY} tests the resulting bool array.

\item
It was already shown in the \htmlref{indexing section}{TAQL:INDEXING}
that indexing arrays should be done with care.
\end{itemize}


\section{Examples}
\begin{description}
  \item[] \texttt{SELECT FROM mytable WHERE column1 $>$ 0 }
    \\selects the rows in which the value of column1 $>$ 0
  \item[] \texttt{ SELECT column0,column1 FROM mytable }
    \\selects 2 columns from the table.
  \item[] \texttt{ SELECT column0,column1 FROM mytable WHERE column1$>$0 }
    \\is a combination of the previous selections.
  \item[] \texttt{ SELECT FROM mytable WHERE column0 IN}
       \\\texttt{[SELECT FROM outtable ORDERBY column0 DESC][1:10] }
       \\selects the 10 highest values of \texttt{column0}
  \item[] \texttt{ SELECT FROM mytable ORDERBY column0 DESC GIVING outtable}
       \\\texttt{SELECT FROM outtable WHERE rownumber()$<=$10 }
       \\is the less elegant SQL solution for the previous problem.
  \item[] \texttt{ SELECT FROM resource.table WHERE}
       \\\texttt{any(PValues == pattern('synth*'))}
       \\selects the rows in which an element in array
       \texttt{PValues} matches the given regular expression.
  \item[] \texttt{ SELECT FROM table WHERE ntrue(flags) >= 3}
    \\selects rows where at least 3 elements of array \texttt{flags}
    are set.
  \item[] \texttt{ SELECT FROM book.table WHERE nelements(author) > 1}
    \\selects books with more than 1 author.
  \item[] \texttt{ SELECT FROM mytable WHERE}
       \\\texttt{ cos(0d1m) $<$
         sin(52deg) * sin(DEC) + cos(52deg) * cos(DEC) *
         cos(3h30m - RA) }
       \\selects observations with an equatorial position (in say J2000)
       inside a circle with a radius of 1 arcmin around (3h30m, 52deg).
       To find them the condition DISTANCE$<=$RADIUS must be fulfilled,
       which is equivalent to COS(RADIUS)$<=$COS(DISTANCE).
       \\Note that in the future this useful astronomical test could
       be implemented as a function of its own (with better optimization
       possibilities).
  \item[] \texttt{ SELECT FROM mytable WHERE object == pattern("3C*") \&\&}
       \\\texttt{cos(0d1m) $<$ sin(52deg) * sin(DEC) + cos(52deg) *
        cos(DEC) * cos(3h30m - RA) }
       \\finds all 3C objects inside that circle.
\end{description}

\section{\label{TAQL:GLISHC}Interface to TaQL}
There exists a user and a programmer interface to TaQL.
Glish functions form the user interface, while C++ functions
and classes form the programmer interface.
\begin{itemize}
\item
  The TaQL interface in Glish is formed by the
  \htmlref{\texttt{query}}{table:query} function in module
  \htmlref{\texttt{table}}{table} (in script \texttt{table.g}).
  The function can be used
  to compose and execute a TaQL command using the various (optional)
  arguments given to the \texttt{query} function. E.g.
  \begin{verbatim}
    tab := table('mytable')
    seltab1 := tab.query ('column1 > 0')
    seltab2 := seltab1.query (query='column2>5',
                              sortlist='time',
                              columns='column1,column2',
                              name='result.tab')
  \end{verbatim}
  The first command opens the table \texttt{mytable}.
  The second command does a simple query resulting in a temporary
  table. That temporary table is used in the next command resulting in
  a persistent table. The latter function call is transformed to
  the TaQL command:
  \\\texttt{SELECT column1,column2 FROM \$1 WHERE column2>5}
  \\\texttt{ORDERBY time GIVING result.tab}
  \\During execution \$1 is replaced by table \texttt{seltab1}.
  \\Note that the \texttt{name} argument
  generates the \texttt{GIVING} part to make the result persistent.

  It is possible to embed glish variables and expressions in a TaQL
  command using the syntax \texttt{\$variable} and
  \texttt{\$(expression)}. A variable can be a standard numeric or
  string scalar or vector. It can also be a table object.
  An expression has to result in a numeric or string scalar or vector.
  E.g
  \begin{verbatim}
    tab := table('mytable')
    coldata := tab.getcol ('col');
    colmean := sum(coldata) / len(coldata);
    seltab1 := tab.query ('col > $colmean')
    seltab2 := tab.query ('col > $(sum(coldata)/len(coldata))')
    seltab3 := tab.query ('col > mean([SELECT col from $tab])')
  \end{verbatim}
  These three queries give the same result.
  \\The substitution mechanism is described in more detail in the
  \htmlref{\texttt{substitute}}{utility:substitute}
  functions of module
  \htmlref{\texttt{utility}}{utility} of the
  \htmladdnormallink{User Reference
  Manual}{../../user/Refman/Refman.html}.
  \\The substitution mechanism uses the eval function in glish.
  As of 8-Jul-1998 eval only looks at global variables. This means
  that in a function one needs to create a global variable (with
  a unique name) if the variable is to be used in a TaQL command.
  The global variable should be deleted at the end of the function.
  The name can be made unique by using the function name as a suffix.
  E.g.:
  \begin{verbatim}
    myfunc := function() {
      tab := table('mytable')
      global coldata_in_myfunc;
      coldata_in_myfunc := tab.getcol ('col');
      seltab := tab.query ('col > $(sum(coldata)/len(coldata))')
      symbol_delete ('coldata_in_myfunc');
    }
  \end{verbatim}
  
\item
  The C++ interface consists of 2 parts.
  \begin{enumerate}
  \item
    The function \texttt{tableCommand} in
    \htmladdnormallink{TableParse.h}
    {../../aips/implement/Tables/TableParse.html}
    can be used to execute a TaQL command. The result is a
    \htmladdnormallink{Table}{../../aips/implement/Tables/Table.html}
    object. E.g.
\begin{verbatim}
  Table seltab1 = tableCommand
         ("select from mytable where column1>0");
  Table seltab2 = tableCommand
         ("select column1,column2 from $1 where column2>5"
          " orderby time giving result.tab", seltab1);
\end{verbatim}
    These examples do the same as the Glish ones shown above.
    \\Note that in the second function call the table name
    \texttt{\$1} is replaced by the object \texttt{seltab1}
    passed to the function.

  \item
    The other interface is a true C++ interface having the
    advantage that C++ variables can be used. Class
    \htmladdnormallink{Table}{../../aips/implement/Tables/Table.html}
    contains functions to sort a table or to select columns or rows.
    When selecting rows class \htmladdnormallink{TableExprNode}
    {../../aips/implement/Tables/ExprNode.html} (in ExprNode.h)
    has to be used to
    build a where expression which can be executed by the overloaded
    function operator in class \texttt{Table}. E.g.
\begin{verbatim}
  Int limit = 0;
  Table tab ("mytable");
  Table seltab = tab(tab.col("column1") > limit);
\end{verbatim}
    does the same as the first example shown above.
    See classes \htmladdnormallink{Table}
    {../../aips/implement/Tables/Table.html},
    \htmladdnormallink{TableExprNode}
    {../../aips/implement/Tables/ExprNode.html}, and
    \htmladdnormallink{TableExprNodeSet}
    {../../aips/implement/Tables/ExprNodeSet.html} for more
    information on how to construct a where expression.
  \end{enumerate}
  
\end{itemize}

\section{Future developments}
In the near or far future TaQL will be enhanced by adding new
features and by doing optimizations.
\begin{itemize}
  \item When a branch in an \texttt{OR} or \texttt{AND} node is a
    constant scalar, it might be possible to reduce the node to
    one of its branches.
  \item Optimize by removing an \texttt{IN} node when the righthand
    operand is empty. This cannot be done when the lefthand operand.
    is variable shaped.
  \item Handle invalid subexpressions (e.g. exceeding array bounds)
    as undefined values
    which can be tested with the function ISDEFINED.
  \item Develop versions of the array functions that can project
    an array to an array with a smaller dimensionality. The projection
    is an operation collapsing slices into single values.
    The second argument has to be a set giving the
    dimension(s) forming the slice. E.g.
    \\\texttt{ntrueproject (array, [2,3])}
    \\means that slices consisting of the 2nd and 3rd dimensions are
    taken. When \texttt{array} has dimensionality (n1,n2,n3,n4), the result is
    an array of dimensionality (n1,n4). Each element in the result is
    the number of true flags in the corresponding input slice.
  \item Interpret column units defined in a column keyword UNIT.
  \item Add a function (say POSITIONINCIRCLE or PINC) to search in a circle
       around a position.
  \item Maybe functions to convert positions from one equinox or system
       to another (using the Measures system).
  \item Use of fields in records by means of operator \texttt{.}.
  \item Use a temporary table as the input of a selection.
  \item Optimize using a B+ tree index.
  \item Expressions in the selected column-list to calculate new columns. E.g.
       \\\texttt{SELECT SQUARE(UVW[1])+SQUARE(UVW[2]) UVWNORM FROM ...}
       \\where \texttt{UVWNORM} is the name for that new column.
\end{itemize}
