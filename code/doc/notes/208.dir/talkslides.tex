\documentclass[11pt]{article}
\special{landscape}
\setlength{\textheight}{7in}
\setlength{\textwidth}{8in}
\setlength{\topmargin}{-0.5in}
\begin{document}


%%%%%%%%%%%%%%%%%%%%%%%%%%%%%%%%%%%%%%%%%%%%%%%%%%%%%%%%%%%%%%%%%%%%%%%%%
%%% TeX definitions of mathematical symbols and names used for the %%%%%% 
%%% Measuerement Equation of Generic Interferometer (MEGI)  %%%%%%%%%%%%%
%%%%%%%%%%%%%%%%%%%%%%%%%%%%%%%%%%%%%%%%%%%%%%%%%%%%%%%%%%%%%%%%%%%%%%%%%
%\def\thisfile{~/aips++/nfra/megi-symbols.tex}

% The following list is available as a LaTeX list of definitions. It
% could be the start of an official AIPS++ list of symbol names, to
% encourage (enforce?) consistency in all AIPS++ documents and code.

%%%%%%%%%%%%%%%%%%%%%%%%%%%%%%%%%%%%%%%%%%%%%%%%%%%%%%%%%%%%%%%%%%%%%%%%%
%   Some user-defined TeX commands:

\newcommand{\tbt}[4]
  {\left(\begin{array}{cc}#1 & #2\\ #3 & #4 \end{array}\right)}
\newcommand{\ddp}[2]{\frac {\partial #1}{\partial #2}}
\newcommand{\ddpn}[3]{\frac {\partial^{#3}{#1}}{\partial{#2}^{#3}}}
\newcommand{\dirprod}[2]{{#1}\otimes{#2}^\ast}


%%%%%%%%%%%%%%%%%%%%%%%%%%%%%%%%%%%%%%%%%%%%%%%%%%%%%%%%%%%%%%%%%%%%%%%%%
%   Naming system (method in the madness):

% \mmName              % Matrix 
% \mjName              % Jones matrix (2x2)
% \vvName              % Vector 

% \ssName              % subscript or superscript (e.g. \ssI: \Antenna I)
% \ccName              % Coordinate (e.g. \ccL: Sky L-coordinate)
% \aaName              % Projected angle (e.g. \aaXY: from X-axis to Y-axis)

% \ppName              % Parameter  

% I,J                  % \Antenna labels (e.g. \AntI)
% X,Y,R,L              % Polarisation labels
% A,B                  % \Receptor labels (e.g. \RcpB)
% P,Q                  % \IFout labels (e.g. \IFP)
%%%%%%%%%%%%%%%%%%%%%%%%%%%%%%%%%%%%%%%%%%%%%%%%%%%%%%%%%%%%%%%%%%%%%%%%%

%  \gloshead{Names of things for which there is a precise definition}

\def\Receptor{{\sf receptor}}   % Converts field to voltage (e.g. dipole) 
\def\Feed{{\sf feed}}           % Group of 1-2 receptors (usually 2) 
\def\Antenna{{\sf antenna}}     % Combination of a \Feed with a mirror.
\def\IFout{{\sf IF-channel}}    % Signal outputs from an \Antenna 
\def\Telescope{{\sf telescope}} % Entire instrument (e.g. WSRT, GBT)
\def\Interferometer{{\sf interferometer}} % Two \IFout's or two \Antenna's
\def\Visibility{{\sf visibility}} % 1-4 complex spectra, 
                                  % output of an \Interferometer. 
\def\Projected{{\sf projected}} % Projected on sky plane (or on the
                                % plane perpendicular to propagation)

% NB: Use a closing backslash (\Antenna\ ) for interword space!                      

%  \gloshead{Labels:}

\def\AntI{{\sf i}}        % The `first' \Antenna in an interferometer
\def\AntJ{{\sf j}}        % The `second' \Antenna in an interferometer

\def\RcpA{{\sf a}}        % The `first' dipole in an \Antenna
\def\RcpB{{\sf b}}        % The `second' dipole in an \Antenna

\def\IFP{{\sf p}}         % The `first' IF-channel from an \Antanna
\def\IFQ{{\sf q}}         % The `second' IF-channel from an \Antenna

\def\RPol{{\sf r}}        % r-pol 
\def\LPol{{\sf l}}        % l-pol 

%  \gloshead{Coordinate frames:}

\def\ccT{{\sf t}}                   % time-coord 
\def\ccF{{\sf f}}                   % frequency-coord 

%  Polarisation-frame (for electric polarisation vector):

\def\ccX{{\sf x}}                   % x-coord 
\def\ccY{{\sf y}}                   % y-coord
\def\ccZ{{\sf z}}                   % y-coord

%  Array-frame (for baseline vector):

\def\ccU{{\sf u}}                   % u-coord 
\def\ccV{{\sf v}}                   % v-coord 
\def\ccW{{\sf w}}                   % w-coord  

\def\vvUVW{{\vec{u}}}          % Projected baseline vector
\def\vvAntPos{{\vec{r}}}       % Projected `Antenna' position vector 
                               %   (Receptor phase centre?)

%  Sky coord. frame (source positions) w.r.t. Delay Centre:  

\def\ccL{{\sf l}}                   % l-coord 
\def\ccM{{\sf m}}                   % m-coord 
\def\ccN{{\sf n}}                   % n-coord (=sqrt[1-l^2-m^2]) 

\def\vvLMN{{\vec{\rho}}}             % Vector (l,m) in Sky frame 

%  Antenna coord. frame (l',m'), projected on the Sky:

\def\ccLI{{\ccL^{'}_{\AntI}}}       % l'-coord of \Ant-frame  
\def\ccMI{{\ccM^{'}_{\AntI}}}       % m'-coord 
\def\ccLIO{{\ccL_{\AntI{0}}}}       % l-coord of origin of \Ant-frame i
\def\ccMIO{{\ccM_{\AntI{0}}}}       % m-coord 

%  Receptor coord. frames (l'',m''), projected on the Sky:

\def\ccLIA{{\ccL^{''}_{{\AntI\RcpA}}}}  % l''-coord for \RcpA-frame 
\def\ccMIA{{\ccM^{''}_{{\AntI\RcpA}}}}  % m''-coord 
\def\ccLIAO{{\ccLI_{\RcpA{0}}}}         % l'-coord of origin of \RcpA-frame 
\def\ccMIAO{{\ccMI_{\RcpA{0}}}}         % m'-coord 

\def\ccLIB{{\ccL^{''}_{{\AntI\RcpB}}}}  % l''-coord for \RcpB-frame 
\def\ccMIB{{\ccM^{''}_{{\AntI\RcpB}}}}  % m''-coord 
\def\ccLIBO{{\ccLI_{\RcpB{0}}}}         % l'-coord of origin of \RcpB-frame 
\def\ccMIBO{{\ccMI_{\RcpB{0}}}}         % m'-coord 

%  Projected angles between the various coord frames: 

\def\aaProj{{\gamma}}                % Projected angle 

\def\aaXY{{\aaProj_{\ccX\ccY}}}      % from Pol x-axis to Pol y-axis 
\def\aaLM{{\aaProj_{\ccL\ccM}}}      % from Sky l-axis to Sky m-axis 
\def\aaLX{{\aaProj_{\ccL\ccX}}}      % from Sky l-axis to Pol x-axis 
\def\aaLI{{\aaProj_{\ccL\AntI}}}     % from Sky l-axis to \Ant l'-axis 
\def\aaIA{{\aaProj_{\AntI\RcpA}}}    % from \Ant l'-axis to \RcpA l''-axis 
\def\aaIB{{\aaProj_{\AntI\RcpB}}}    % from \Ant l'-axis to \RcpB l''-axis 

\def\aaXI{{\aaProj_{\ccX\AntI}}}     % from x-axis to \Ant l'-axis 
\def\aaXA{{\aaProj_{\ccX\RcpA}}}     % from x-axis to \RcpA l''-axis 
\def\aaYB{{\aaProj_{\ccY\RcpB}}}     % from y-axis (!) to \RcpB l''-axis  

\def\aaXYexp{{\pi/2}}                % from x-axis to y-axis 
\def\aaLMexp{{\pi/2}}                % from l-axis to m-axis
\def\aaLXexp{{\pi/2}}                % from l-axis to x-axis 
\def\aaXIexp{{-\aaLX+\aaLI}}                % x to l' 
\def\aaXAexp{{-\aaLX+\aaLI+\aaIA}}          % x to l''(A)  
\def\aaYBexp{{-\aaXY-\aaLX+\aaLI+\aaIB}}    % x to l''(B) 


%  \gloshead{Miscellaneous parameters:}

\def\ppHA{{HA}}                  % Hour Angle
\def\ppDEC{{DEC}}                % Declination
\def\ppRA{{RA}}                  % Right Ascension
\def\ppLAT{{LAT}}                % Latitude (on the Earth)
\def\ppParall{{\beta}}           % parallactic angle (between great
			         % circles through North Pole and Zenith)

\def\ppFarad{{\chi}}             % Faraday rotation angle (rad)
\def\ppAmpl{{a}}                 % Amplitude
\def\ppPhase{{\psi}}             % Phase
\def\ppPhaseZero{{\zeta}}        % Phase-zero
\def\ppDiposErr{{\theta}}	       % dipole position angle error  
\def\ppEllipt{{\phi}}            % \Receptor ellipticity

\def\ppBellipt{{\epsilon}}       % beam ellipticity factor      
\def\ppBsigma{{\sigma}}          % sigma of gaussian beam       

%  \gloshead{Jones Matrices (2x2):}

\def\mjJones{{\sf J}}          % Jones instrumental response matrix
\def\mjJonesEl{{\sf j}}        %   matrix element

\def\mjRot{{\cal R}}           % General rotation matrix R(alpha)
\def\mjUnit{{\cal U}}          % Unit matrix
\def\mjMult{{\cal M}}          % Multiplication matrix M(beta)=beta.U
                               % (equivalent to multiplicative factor)
\def\mjLtoC{{\cal H}}          % Conversion from Linear to Circular pol
\def\mjCtoL{{\mjLtoC^{-1}}}    % Conversion from Circular to Linear pol

\def\mjFrot{{\sf F}}           % Faraday rotation matrix
\def\mjFrotEl{{\sf f}}         %   matrix element of \mjFrot

\def\mjGatm{{\sf T}}           % Atmospheric gain matrix
\def\mjGatmEl{{\sf t}}         %   matrix element of \mjGatm

\def\mjProt{{\sf P}}           % Parallactic rotation (dipoles w.r.t. Sky) 
\def\mjProtEl{{\sf p}}         %   matrix element of \mjProt

\def\mjBeam{{\sf E}}           % Antenna Voltage Beam matrix 
\def\mjBeamEl{{\sf e}}         %   matrix element of \mjBeam

\def\mjBatt{{\sf B}}           % Attenuation beam matrix
\def\mjBattEl{{\sf b}}         %   matrix element of \mjBatt

\def\mjObst{{\sf O}}           % Beam obstruction effect matrix
\def\mjObstEl{{\sf o}}         %   matrix element of \mjObst

\def\mjLeak{{\sf D}}           % \Receptor cross-`leakage' matrix
\def\mjLeakEl{{\sf d}}         %   matrix element of \mjLeak

\def\mjHybr{{\sf H}}           % Hybrid network matrix (linear to circular)
\def\mjHybrEl{{\sf h}}         %   matrix element

\def\mjComm{{\sf C}}           % signal-channel commutation matrix
\def\mjCommEl{{\sf c}}         %   matrix element

\def\mjGrec{{\sf G}}           % Receiver Gain matrix (electronics)
\def\mjGrecEl{{\sf g}}         %   matrix element

\def\mjWarr{{\sf W}}           % Fourier Transform phase weight matrix  
\def\mjWarrEl{{\sf w}}         %   matrix element

\def\mjGsum{{\sf Q}}           % Receiver gain of sum-output of tied array 
\def\mjGsumEl{{\sf q}}         %   matrix element

%  \gloshead{Other ME matrices and vectors:} 

\def\vvIQUV{{\vec{I}}}         % Stokes vector (I,Q,U,V)

\def\vvCoh{{\vec{V}}}          % Visibility (`coherency') vector (4)
\def\vvCohEl{{\sf v}}          %   element of \vvCoh

\def\mmStokes{{\sf S}}         % Stokes conversion matrix (4x4) to Coherence

\def\mmMifr{{\sf M}}           % Multiplicative ifr-based gain 
\def\mmMifrEl{{\sf m}}         %   element of matrix \mmMAifr
\def\vvAifr{{\vec{A}}}         % Additive ifr-based vector (4)
\def\vvAifrEl{{\sf a}}         %   element of \vvAifr



%%%%%%%%%%%%%%%%%%%%%%%%%%%%%%%%%%%%%%%%%%%%%%%%%%%%%%%%%%%%%%%%%%%%%%%%
%%%%%%%%%%%%%%%%%%%%%%%%%%%%%%%%%%%%%%%%%%%%%%%%%%%%%%%%%%%%%%%%%%%%%%%%
%  Secondary symbols (much-used combinations of the basic symbols above)
%%%%%%%%%%%%%%%%%%%%%%%%%%%%%%%%%%%%%%%%%%%%%%%%%%%%%%%%%%%%%%%%%%%%%%%%

\def\Antennas{{{\Antenna}s}}        %   
\def\Receptors{{{\Receptor}s}}      %   

\def\ssI{_{\AntI}}                % subscript \Antenna I  
\def\ssJ{_{\AntJ}}                % subscript \Antenna J  
\def\ssP{_{\IFP}}                 % subscript \IFout P  
\def\ssQ{_{\IFQ}}                 % subscript \IFout Q 
\def\ssA{_{\RcpA}}                % subscript \Receptor A 
\def\ssB{_{\RcpB}}                % subscript \Receptor B 

\def\ssL{^{+}}                    % superscript for linear pol  
\def\ssC{^{\odot}}                % superscript for circular pol  
\def\ssLI{^{+}_{\AntI}}           % sub/super  
\def\ssCI{^{\odot}_{\AntI}}       % sub/super  

\def\ssIX{_{\AntI\ccX}}           % subscript   
\def\ssIY{_{\AntI\ccY}}           % subscript   
\def\ssIR{_{\AntI\RPol}}          % subscript   
\def\ssIL{_{\AntI\LPol}}          % subscript   
\def\ssIA{_{\AntI\RcpA}}          % subscript   
\def\ssIB{_{\AntI\RcpB}}          % subscript   
\def\ssIP{_{\AntI\IFP}}           % subscript   
\def\ssIQ{_{\AntI\IFQ}}           % subscript   

\def\ssIJ{_{\AntI\AntJ}}          % subscript 
\def\ssXY{_{\ccX\ccY}}            % subscript   
\def\ssRL{_{\RPol\LPol}}          % subscript   
\def\ssAB{_{\RcpA\RcpB}}          % subscript   
\def\ssPQ{_{\IFP\IFQ}}            % subscript 

\def\ssIXX{_{\AntI\ccX\ccX}}      % subscript   
\def\ssIYX{_{\AntI\ccY\ccX}}      % subscript   
\def\ssIXY{_{\AntI\ccX\ccY}}      % subscript   
\def\ssIYY{_{\AntI\ccY\ccY}}      % subscript   

\def\ssIRR{_{\AntI\RPol\RPol}}    % subscript   
\def\ssILR{_{\AntI\LPol\RPol}}    % subscript   
\def\ssIRL{_{\AntI\RPol\LPol}}    % subscript   
\def\ssILL{_{\AntI\LPol\LPol}}    % subscript   

\def\ssIXA{_{\AntI\ccX\RcpA}}     % subscript   
\def\ssIYA{_{\AntI\ccY\RcpA}}     % subscript   
\def\ssIXB{_{\AntI\ccX\RcpB}}     % subscript   
\def\ssIYB{_{\AntI\ccY\RcpB}}     % subscript   

\def\ssIRA{_{\AntI\RPol\RcpA}}    % subscript   
\def\ssILA{_{\AntI\LPol\RcpA}}    % subscript   
\def\ssIRB{_{\AntI\RPol\RcpB}}    % subscript   
\def\ssILB{_{\AntI\LPol\RcpB}}    % subscript   

\def\ssIAA{_{\AntI\RcpA\RcpA}}    % subscript   
\def\ssIBA{_{\AntI\RcpB\RcpA}}    % subscript   
\def\ssIAB{_{\AntI\RcpA\RcpB}}    % subscript   
\def\ssIBB{_{\AntI\RcpB\RcpB}}    % subscript   

\def\ssIAP{_{\AntI\RcpA\IFP}}     % subscript  
\def\ssIBP{_{\AntI\RcpB\IFP}}     % subscript  
\def\ssIAQ{_{\AntI\RcpA\IFQ}}     % subscript  
\def\ssIBQ{_{\AntI\RcpB\IFQ}}     % subscript  

\def\ssIPP{_{\AntI\IFP\IFP}}      % subscript 
\def\ssIQP{_{\AntI\IFQ\IFP}}      % subscript 
\def\ssIPQ{_{\AntI\IFP\IFQ}}      % subscript 
\def\ssIQQ{_{\AntI\IFQ\IFQ}}      % subscript 
   
\def\ssIPJP{_{\AntI\IFP\,\AntJ\IFP}}     % subscript 
\def\ssIPJQ{_{\AntI\IFP\,\AntJ\IFQ}}     % subscript 
\def\ssIQJP{_{\AntI\IFQ\,\AntJ\IFP}}     % subscript 
\def\ssIQJQ{_{\AntI\IFQ\,\AntJ\IFQ}}     % subscript 


%%%%%%%%%%%%%%%%%%%%%%%%%%%%%%%%%%%%%%%%%%%%%%%%%%%%%%%%%%%%%%%%%%%%%%%%%
%%% End of MEGI symbols definition file megi-symbols.tex %%%%%%%%%%%%%%%%
%%%%%%%%%%%%%%%%%%%%%%%%%%%%%%%%%%%%%%%%%%%%%%%%%%%%%%%%%%%%%%%%%%%%%%%%%

% Added by TJC
\def\mjKarr{{\sf K}}           		% FT matrix
\def\mjWtarr{{\Lambda}}           	% Noise
\def\mjSkyWtarr{{\Lambda^{\sf sky}}}        % Noise
\def\mjConf{{\sf C}}           		% feed configuration matrix
\def\mjConfEl{{\sf c}}         		% matrix element
\def\mjSkyJones{{\sf J^{sky}}}          % Jones instrumental response matrix
\def\mjVisJones{{\sf J^{vis}}}          % Jones instrumental response matrix
\def\vvSkyCoh{{\vec{V}^{\sf sky}}}          % Visibility (`coherency') vector (4)
\def\vvCalCoh{{\vec{V}^{\sf cal}}}          % Visibility (`coherency') vector (4)
\def\xfnCorr{{{X}}}          	% Correlator function
\def\MEGI{{\tt MeasurementEquation}}
\def\SE{{\tt SkyEquation}}
\def\VE{{\tt VisEquation}}
\def\VisSet{{\tt VisSet}}
\def\SkyModel{{\tt SkyModel}}
\def\SkyJones{{\tt SkyJones}}
\def\VisJones{{\tt VisJones}}
\def\MJones{{\tt MJones}}
\def\ACoh{{\tt ACoh}}
\def\XCorr{{\tt XCorr}}
\def\FTCoh{{\tt FTCoh}}
\def\IFTCoh{{\tt IFTCoh}}
\def\WTCoh{{\tt WTCoh}}
\def\solve{{\tt solve}}
\newcommand{\visterm}[1]{{\left[\dirprod{#1\ssI}{#1\ssJ}\right]}}
\newcommand{\skykterm}[1]{{\left[\dirprod{#1\ssI\left(\vvLMN_k\right)}{#1\ssJ\left(\vvLMN_k\right)}\right]}}

\newcommand{\dddp}[3]{\frac {\partial^{2}{#1}}{\partial{#2}\partial{#3}}}



\newenvironment{slide}[1]{\LARGE\begin{center}{\bf #1}\end{center}}{\newpage}
\newenvironment{figlab}[1]{\Large\begin{center}{\bf #1}\end{center}}{\newpage}

\begin{slide}{What is AIPS++ and why are we doing it?}
\begin{itemize}
\item {\em Consortium} of radio observatories, building package
to handle processing for new telescopes, and improved processing
for existing telescopes.
\item All observatories facing same problem of old packages
leading to user dissatisfaction, and limited technical
development:
\begin{itemize}
\item Need better tools for end-user, improved interfaces,
and new packages
\item Astronomers need to be able to program at {\em e.g.}
IDL-type level
\item Programmers need richer set of tools and packages to
work with
\end{itemize}
\end{itemize}
\begin{description}
\item[Data] Uniform access to data
\item[Tools] General operations, array math, FFTs, fitting, plotting,
display, {\em etc.}
\item[Tasks] Applications for doing radio-astronomy {\em e.g.}
processing of data from radio-telescopes, both single dishs and
synthesis arrays
\end{description}
See {\em http://www.nrao.edu/aips++} for more information
\end{slide}

\begin{slide}{What is AIPS++ being used for now?}
\begin{description}
\item[ATNF] At the Parkes telescope for Parkes Multibeam 
observing.
\item[NRAO] At Green Bank, for support of the Green Bank
Telescope engineering.
\item[NFRA] At WSRT, integrated into the Telescope Management
System.
\end{description}
\end{slide}

\begin{slide}{AIPS++ Project timeline}
\begin{description}
\item[First beta release {\em Feb 97}]:
\begin{itemize}
\item Continuum synthesis imaging and self-calibration
\item General Glish-based tools
\end{itemize}
\item[Second beta release {\em September 97}]:
\begin{itemize}
\item Spectral line additions to synthesis processing
\end{itemize}
\item[Third beta release {\em Early 98}]:
\begin{itemize}
\item GUIs for many standard objects
\item More synthesis functionality
\end{itemize}
\item[V1.0 {\em Mid 98}]: Limited Public Release
\item[V1.5 {\em Late 98}]: Code development release
\begin{itemize}
\item Code development system (documentation, examples, templates). 
\end{itemize}
\item[V2.0 {\em Early 99}]: Full release
\begin{itemize}
\item Completed connected element synthesis package
\item Initial VLBI capabilities
\item Mosaicing package
\item Contributed code
\end{itemize}
\end{description}
\end{slide}

\begin{slide}{AIPS++ Personnel}
\begin{description}
\item[ATNF]: Neil Killeen, David Barnes, Wim Brouw, and Mark
Wieringa

\item[BIMA/NCSA]: Dick Crutcher, Dan Briggs, John Pixton,
Harold Ravlin, Doug Roberts, and Peter Teuben

\item[NFRA]:  Jan Noordam, Michael Haller, Friso Olnon, 
Ger van Diepen, Henk Vosmeijer

\item[NRAO]: Tim Cornwell, Bob Garwood, Brian
Glendenning, Athol Kemball, Ralph Marson, Joe McMullin, Pat Murphy,
Darrell Schiebel, Jeff Uphoff, Kate Weatherall and Wes Young.

\end{description}
\end{slide}

\begin{slide}{What's available in First Beta Release (0.8)?}
\begin{itemize}
\item The synthesis capabilities of AIPS++ are as follows:
\begin{itemize}
\item Filling from and writing to a UVFITS file,
\item Filling from WSRT format,
\item Full imaging, deconvolution, and self-calibration,
\item Joint deconvolution of Stokes IQUV,
\item Robust, uniform and natural weighting,
\item Flexible windowing in the deconvolution,
\item Non-Negative Least Squares Deconvolution,
\item Flexible construction of models for self-calibration,
\item A sophisticated multi-component model for gain effects,
\item Interactive editing,
\item Flagging of gain solutions by antenna, spectral window, and
time interval.
\item Writing of final images to FITS files.
\end{itemize}
\item AIPS++ has a powerful command line interpreter called Glish.
\item AIPS++ has a number of general purpose tools that are accessible
from Glish.
\begin{itemize}
\item Access to all data in AIPS++ via the table module.
\item Catalogs of directories, interpreting the contents to show 
different types of files, and catalogs of potentially everything in Glish
({\em i.e.} functions and variables).
\item Plotting of Glish variables.
\item Assorted mathematical capabilities such as 
statistics, random numbers, polynomial fitting, interpolation,
Fast Fourier Transforms and convolutions.
\item Manipulation of measured quantities with units and reference
frames either from the Glish command line or via a gui.
\item Display of AIPS++ or FITS images or Glish arrays using the
aipsview program.
\item Conversion of images to and from FITS image format, 
reading and writing AIPS++ images to and from Glish. Statistics, histograms and 
moments of images may be calculated. Subimaging and padding are both allowed.
\item Logging of messages to a gui window and a table, also printing
to a postscript printer or to ghostview.
\item Miscellaneous useful utilities including: obtaining AIPS++ 
configuration information, the help utility, 
a printer tool (including a gui), execution of shell commands, reading and 
writing to external files using C-like commands, {\em etc.}
\end{itemize}
\item All user capabilities of AIPS++ are documented via the
AIPS++ User Reference Manual
and on-line help is available from the command line.
\end{itemize}
\end{slide}

\begin{slide}{Responses to First Beta release (0.8)}

\begin{itemize}
\item Announced on February 26 1997. 
\item All consortium sites, and Caltech, University of Iowa, Kapteyn
Astronomical Institute, and the National Optical Astronomy
Observatory. 
\item Most feedback on the beta release is conducted either via the AIPS++
Bug Tracking system or or via an e-mail exploder. A fair amount also
via private e-mail.
\item The responses to beta release can be summarized as follows:
\begin{enumerate}
\item Installation seems relatively straightforward apart from some
difficulties with shared libraries.
\item Configuration does not require much work.
\item The initial verification of the system via an assay function
seems to have gone straightforwardly.
\item The documentation seems comprehensive but hard-to-understand.
Some of the OO terminology is either not explained well or has crept
into places where it is not wanted.  Testers have found a
(not-unexpected) number of errors in documentation.
\item The user interface is viewed as overly verbose and unfriendly.
\item The synthesis code is viewed as being powerful but hard to
master.
\item The synthesis code is slow and subject to memory
bloat in some circumstances.
\item There have been a number of bugs, perhaps slightly more in code
written in Glish (as opposed to C++).
\end{enumerate}
\end{itemize}
\end{slide}

\begin{slide}{What's new in Second Beta Release (0.9)?}

\begin{itemize}
\item Announced on September 11 1997. 
\end{itemize}
\begin{description}
\item[Aipsview] Many improvements to facilitate control of the
     program. Can also provide flexible slicing through multidimensional
     images.  
\item[Glish] Now considerably more robust. 
\item[Image DO] Production of moments images, Hanning smoothing and 
numerous small improvements have been made to the Glish/image
interface. FITS history cards are now preserved for
images. Asynchronous execution is possible (as for other Distributed
Objects). No transpose necessary.
\item[imager] Many changes involving spectral line processing.
A demonstration of spectral line processing is available as {\tt
imagerspectraltest()}. There have been improvements to the speed of
all types of processing, and a number of changes to the
interface. Imager now traps errors more carefully, and can be
interrupted (by Control C) and restarted (by {\tt di.restart()}.
Filling of MeasurementSets is now possible from BIMA and WSRT
data sets.
\item[Messages] Some unnecessarily obscure messages have been eliminated
or made less threatening.
\item[Spectral Cubes] The FITS/spectral axis interconversion is more
robust now.  
\end{description}
\end{slide}

\begin{slide}{Responses to Second Beta release (0.9)}

\begin{itemize}
\item Some errors in FITS processing
\item Some worries over channel zero processing
\item That's it!
\end{itemize}
\end{slide}

\begin{slide}{What's planned for the Third Beta Release?}
\begin{description}
\item[GUIs for standard Distributed Objects]: Beta testing showed that
we needed a better interface. There will be a standard format interface,
autogenerated for most Distributed Objects.
\item[A GUI for imager]: The imager interface will be redesigned
and a GUI added.
\item[Visibility visualization tool]: Glish-based Visibility visualization
tool {\em written by an astronomer with no C++ needed!}
\item[More synthesis capabilities]: Need more functionality asap
\item[Single Dish processing]: GUI-based tool for Single Dish processing
\item[Improved tools]: {\em e.g.} much improved table browser with editing,
selection of columns to display, user-controlled format, SQL-like queries,
flexible plotting, {\em etc.}
\item[AIPS/AIPS++ Interoperability]: Can access AIPS Tasks from Glish,
also AIPS++ Data and Functions available via client-server interface
%(mainly for testing/debugging).
\item[Cookbook]: Once the look and feel of the interfaces has stabilized,
we plan to start on a cook-book.
\end{description}
\end{slide}

\begin{slide}{What's in the works?}
\begin{description}
\item[Synthesis]: to include wide-field (3D) imaging, mosaicing, improved
calibration methodology. Also working on parallelization of synthesis code.
Great team effort: Briggs, Brouw, Kemball, Marson, Noordam, Wieringa, and 
Cornwell.
\item[Visualization library]: C++-based visualization library. Available
to application programmers as a service. Some initial applications will
be available soonish. Eventually plan {\em e.g.} better visibility
visualization tool based on this library.
\item[Internals]: Rearrangement of internal directory structure, {\em etc.}
\item[Developer's release]: Sometime late 98 or early 99.
\end{description}
\end{slide}

\begin{slide}{Why would I want to use AIPS++ now?}
\begin{description}
\item[Synthesis] unique capabilities:
\begin{description}
\item[Non-Negative Least Squares deconvolution]
\item[IQUV Simultaneous CLEAN]
\item[Polarization leakage self-cal]
\end{description}
\item[Ad Hoc processing] {\em e.g.} Strange image or visibility
processing.
\item[AIPS GUI Tool] Can run AIPS from a GUI (in next beta, or in daily
version now).
\end{description}
\end{slide}

\begin{slide}{Examples of Ad Hoc processing}
\begin{description}
\item[ASCII Table]: can convert ASCII table to AIPS++ Table then
use AIPS++ and Glish-based math, plotting, browsing tools.
\item[Image]:
\begin{itemize}
\item Convert FITS to/from AIPS++ Image
\item Read/write AIPS++ Image to/from Glish
\item Do strange math in Glish
\end{itemize}
\item[Visibility Data]: For example, removal of Cygnus A from
75 MHz data:
\begin{itemize}
\item Can convert UVFITS to/from AIPS++ MeasurementSet
\item Use {\tt componentmodel} DO to hold model of Cygnus A
\item Use {\tt measures} DO to do angle conversions, {\em etc.}
\item Use {\tt ms} DO to read/write {\bf any} part of visibility
data to/from a MeasurementSet
\item Use Glish to do math
\item Use {\tt mathematics} DO to help math
\end{itemize}
\end{description}
\end{slide}

\begin{slide}{AIPS++ Operations}
\begin{itemize}
\item Starting to think about Operations of AIPS++:
\begin{description}
\item[Observatories] Observatories will use AIPS++ as the vehicle for
providing support for particular telescopes. 
a number of components:
\begin{description}
\item[Software] for observing preparation, monitoring of observations,
and reduction of data.
\item[Distribution] of ancillary data for the telescope {\em e.g.}
source lists, antenna coordinates, instrument history, physical 
parameters (such as ionospheric menasurements), {\em etc.}
\item[Support] for astronomers using the telescope.
\end{description}
\item[Astronomers] Astronomers will use AIPS++ for access to
common telescope reduction packages, and for analysis tools that are
not the specific responsibility of any one observatory.
\item[Programmers] Programmers (who may be active astronomers) will use AIPS++
as a resource for tools to solve problems and as a distribution mechanism for
their products.
\end{description}
\item More details in AIPS++ Note 209.
\item Core functions:
\begin{description}
\item[Maintenance]: Bugs, tracking environment changes
\item[Distribution]: System, Data, Knowledge, Update schedule
\item[Development]: Core library, Applications, System
\end{description}
\item Astronomer and programmer support:
\begin{description}
\item[Help]:
\item[Education and training]: User and developer workshops, tutorials
and demonstrations
\item[Feedback]: email, phone, bug-reports
\item[Library and contributed code]: core, documentation, consortium, 
contributed, ad hoc.
\end{description}
\item Quality Assurance group: Unit testing, application testing, package
testing, benchmarking, code reviews.
\item Advice and Oversight: AIPS++ User Group and Technical Working
Group
\end{itemize}
\end{slide}

%\end{document} 
