\documentstyle[11pt]{article}
\begin{document}


\title{Response to the Second report of the AIPS++ Scientific \& Technical 
Advisory Group}

\author{T.J. Cornwell, NRAO}

\date{March 10, 1998}

\maketitle

\section{Introduction}

This is in response to the second report of the AIPS++ Scientific and
Technical Advisory Group (AIPS++ Note 218). We would like to thank the
group for their generous donation of time, and we appreciate their
advice on numerous issues connected with AIPS++. Overall, we do find
the report to be helpful in directing AIPS++, though we are disturbed
by the complete change in recommendation from the last report in one
key area. We will address this point in the next section.

\section{General remarks}

We are glad that the STAG recognizes the process made since the last
meeting. Progress has indeed occurred in many areas, along the lines
recommended in the last STAG report, such as stressing unique
functionality and improving user interfaces. 

There are three particular areas of concern that the STAG note in the
introduction: first, the need for planning of the delivery of package
functionality, second, the need for an improved user interface, and
third, the question of the optimum strategy to garner support in the
user community. We discuss these in turn.

First, we spend considerable effort on both planning and tracking
development of AIPS++ as a whole and in the various
components. Development plans for AIPS++ and for various components
may be found in the AIPS++ Notes series. In particular, our release
strategy, which follows the overall directions recommended by the STAG
at the last meeting, is detailed in the AIPS++ Notes and has been
regularly described in the AIPS++ Quarterly Reports. Tracking of
existing tasks and allocation of new tasks is performed weekly, and
list of accomplishments are given in the Quarterly Reports.

Second, our beta testing did indeed tell us that the user interface
was too difficult for most users. Following the first beta release, we
embarked upon an initiative to write GUIs for AIPS++ applications. As
as result, GUIS are now being placed into use and our testers at the
AOC and elsewhere are giving useful feedback. We anticipate that GUIs
will be one of the major deliverables in the next, upcoming beta
release. Our plans for an improved command line interface wait on our
experience with the GUIs since we would finalize the GUIs before
proceeding much with the former. Our current expectation is that a
parameter-setting shell along the lines that we presented to the STAG
can be placed in operation without too much additional work beyond
that in the GUI.

Moving on to the third point of the strategy for garnering user
community support, we think that initiating development of a VLA
filler now is an excellent idea and we will start devoting some effort
to the filler in the near future. We hope that this will send a
reassuring message about our commitment to establishing VLA data
reduction capabilities in AIPS++ as soon as possible.  However, in a
change from the last report, the STAG now believes that full
end-to-end data reduction capability for all the consortium telescopes
is a {\em prerequisite} for a first public release. After considerable
discussion inside the Project, we are not convinced that this change
would be wise. Our overall development approach has been to aim
initially for highly targeted and unique applications that will
attract key users who will lead the way to eventual adoption and
acceptance of the package by others. There are a number of examples of
packages that have evolved this way: AIPS, MIRIAD, difmap. In none of
the initial versions of these packages could full end-to-end reduction
for the relevant telescopes be accomodated. Instead, key capabilities
attracted new users, and the packages subsequently expanded in
functionality. The alternative as recommended by the STAG, of
deferring the first release until we can support a full end-to-end
data reduction path would prevent us from getting the support of those
people for whom the package does have interesting capabilities. We
identify the following as being deliverable as part of the first
release: a complete data path starting from data initially calibrated
elsewhere, incorporating sophisticated enhancements to calibration and
novel imaging capabilities, and finishing with publication quality
plots.  Once this data path is in place, works well, allows new
science, and the user interfaces are well-established, then we will
work back towards providing full data reduction capabilities. Our
reading of the North American community, via for example the NRAO
Visiting and Users' Committees, is that a considerable majority of the
interested parties favor this approach. However, in light of the
STAG's change of mind from the previous report, this is an issue that
we will revisit in our upcoming meetings with North American review
bodies.  The implications of the STAG's recommendations have the most
impact for VLA data reduction since the filler and the various
ancillary tools are a significant development cost. The situation with
regard to supporting WSRT data reduction is somewhat different, since
the existing package, NEWSTAR, is now being phased out in favor of
AIPS++, and we have a commitment to provide calibration and imaging
capabilities for WSRT on a short timescale. This is possible because
the filler is largely in place, and since the tools required to
support full calibration are less onerous to develop.

\section{Specific comments}

\begin{description}
\item[Scheduling of releases] We agree that a frequency of 6 months
for $\beta$ releases is realistic and will avoid exhausting
our testers' interest and patience, as well as our ability to
respond.
\item[Programmer's release] The advice on the relative importance
of the programmer's release is particularly welcome.
\item[Limited Public Release] With the exception of the end-to-end
capability addressed above, we agree with the list of components
deemed necessary for the release.
\item[Stability of Glish, Table system and low-level library] We 
agree that these components have stabilized and we do intend
to move people from developments in these areas towards applications
development.
\item[Graphics] It is true that our effort on graphics
has resulted in a number of overlapping and redundant tools. We aim
to consolidate these tools shortly.
\item[Display Library] We agree with the priority of a visibility
visualizer as a first application for the DL, and also with
the expressed priority of an AIPSView replacement on a longer
time-scale.
\item[Documentation] Development of a cook-book
is the highest priority for our AOC testers and is now proceeding.
\item[Single Dish processing] Work on the analysis and
reduction of multi-dimensional datasets will commence once
the dish program is closer to a final product. We see such work
as being closer to the corresponding work in synthesis than
to dish itself, and we anticipate that many of the tools
will be usable in both single dish and synthesis processing.
\item[Platforms and compilers] We agree with the STAGs advice in
these areas.
\item[Parallelization] We will take care to keep the development
of parallelized code in the correct proportion. We do regard
this as a vital investment for the future.
\item[Performance] Much of our performance testing has been
in comparison with MIRIAD, and we intend to continue this
work, expanding to difmap and other packages.
\item[New functionality] We appreciate the advice concerning
the importance of mosaicing, wide-field imaging and
polarimetric imaging.
\end{description}
\end{document}
