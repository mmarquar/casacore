\section{Introduction}
The \htmladdnormallink{AIPS++ Table System}{../255/255.html}
is used to store all data in the AIPS++ environment.
The Table System supports concurrent access by means of table locking.
The user can control how the table locking is used which may have
impact on the performance of the Table System.
This note describes the various ways in which a table can be locked
and the best ways to use them.

\section{Locking and Synchronization}
Before a table can be read or written, an appropriate lock has to be
acquired. As in most systems there are two types of locks:
\begin{itemize}
\item A read lock (also called shared lock) is used to grant a process
read-access to a table. At a given time multiple processes can have
a read lock on the same table (hence shared lock).
\item A write lock (also called exclusive lock) is used to grant a
process write-access to a table. At a given time only one process can
have a write lock on a table (hence exclusive). If a write lock
is acquired, no read lock on that table can exist at the same time.
Thus at a single time there can be one writer or multiple readers.
\end{itemize}
The locking assures that the internal buffers of the Table
System are synchronized and kept consistent.

Note, however, that the Table System also supports the so-called
NoReadLocking mode, which makes it possible to read a table without a
lock. That implies that no synchronization is done before the read,
unless it is explicitly done. It is described in a
\htmlref{later section}{TABLELOCK:NOREADLOCKING}.

The Table System only supports locks on the entire table; there is
no fine-grained page or row locking. Thus if a process is updating a
table, no other process can read or update any part of that table.

\medskip
Basically there are two functions in the Table System to handle
locking.
\begin{itemize}
\item \texttt{Table::lock} can be used to acquire a read or write
lock. Note that a write lock also grants read-access.
Internal data buffers will be refreshed as needed.
\item \texttt{Table::unlock} can be used to release a lock.
It will flush the internal data buffers to disk if they are changed.
\end{itemize}

\subsection{Synchronization}
The Table System uses a special lock file (\texttt{table.lock}) in the table
directory to hold information about the storage managers containing
data changed in a lock/unlock cycle. That information is used to
achieve that a process acquiring a lock only refreshes the buffers
really needed. This saves a lot of needless buffer refreshing.

Apart from changing data, a table can also change by adding or
removing rows, keywords, or columns. These changes are also
synchronized with the exception of added or removed columns. Such
changes mean that the layout of the table is changed. Alas the Table
System cannot cope with such changes yet (it will in a future
version). If it detects that a column is added or removed, the
synchronization function throws an exception.

\medskip
The Table System uses the file locking functions provided by the
operating system
to do the actual locking. For UNIX systems it means that the
\texttt{fcntl} function is used. This also works fine for NFS files
provided that the \texttt{lockd} and \texttt{statd} deamons are
running (which is usually the case).

The first few bytes of the lock file form the data part
that is being used for the locking.
\\Furthermore an extra lock is used to keep track of the tables that are
open in any process. This is used by the Table System to prevent
a table from being deleted while another process is still accessing
it.

\medskip
Note that on a UNIX system file locks are handled per inode. That
means that if the same file is opened twice an unlock on one the the
file descriptors also unlocks the other file descriptor.
The Table System usually opens a table only once. However, if opened
with a different name, the Table System might not recognize that
the file bas been opened already and open it again in the same process.
However, in practice it never happens.


\section{\label{TABLELOCK:LOCKINGMODES}Locking Modes}
Data base systems usually hold locks only for a short period of time. It means
that buffers have to be flushed often because another process needs
up-to-date data. A flush can be expensive as it involves I/O.
\\It was believed that this would be too limiting for the Table
System, because many AIPS++ processes like to lock a table for as
long as possible to avoid too many flushes.
Therefore three locking modes are supported by AIPS++. They can be
given when a \texttt{Table} object is constructed. In order high to low
they are:

\begin{enumerate}
\item \texttt{PermanentLocking} locks the table for
the full session, so no other process can use it. It is always
locked for read. It is locked for write if the table is not opened
as readonly.
\\This mode is rarely used, because it makes concurrent access
impossible (unless another process uses NoReadLocking).
However, in some circumstances it might
be useful. Note that in this mode lock and unlock calls can still be
done, but they are not doing anything.
\item \texttt{AutoLocking} locks automatically when a read or write needs
to be done and the table does not have an appropriate lock yet.
Unlocking is done automatically at an appropriate time.
\\This mode makes concurrent access possible and tries to hold the lock as
long as possible. Furthermore the programmer
does not have to worry about (un)locking, although it is still possible to
do explicit locking and unlocking.
\\The problem with this mode is that automatic unlocking can only
be done when possible. Therefore it
could happen that a table is locked for a longer period than expected.
It is discussed in more detail in a
\htmlref{later section}{TABLELOCK:AUTOLOCKING}.
\item \texttt{AutoNoReadLocking} is the same as AutoLocking, but no
locks are needed when reading the table. NoReadLocking is discussed in
\htmlref{another section}{TABLELOCK:NOREADLOCKING}.
\item \texttt{UserLocking} means that locking and unlocking have
to be done explicitly.
\\This mode gives the finest control, but it it may be hard for the
programmer to decide how fine-grained it should be used. On one hand
not too fine, because it involves a lot of flushing. On the other hand
fine enough to give other processes the opportunity to grab a lock.
An exception is thrown if a
read or write is done before an appropriate lock is acquired.
\item \texttt{UserNoReadLocking} is the same as \texttt{UserLocking}, but no
locks are needed when reading the table. NoReadLocking is discussed in
\htmlref{another section}{TABLELOCK:NOREADLOCKING}.
\item \texttt{DefaultLocking} is the same as \texttt{AutoLocking}. It is lower
in the hierarchy when merging locking modes (see next section).
\end{enumerate}
The default locking mode is \texttt{DefaultLocking}.

\subsection{Locking Mode Merge}
It is possible that in a process multiple \texttt{Table} objects are created
for the same table. These \texttt{Table} objects share the same underlying
\texttt{BaseTable} object doing the actual locking.
Since each \texttt{Table} object can be created with its own locking mode,
those locking modes have to be merged in the \texttt{BaseTable} object.
The merge result is the locking mode with the highest position in the
\htmlref{locking mode table}{TABLELOCK:LOCKINGMODES}.

\subsection{Locking Mode for a Subtable}
\texttt{Table} objects for subtables are created automatically
when the table is retrieved from a keyword set like:
\begin{verbatim}
    Table tab(``test.ms'');
    Table anttab(tab.keywordSet().asTable(``ANTENNA''));
\end{verbatim}
In this way it inherits the locking mode of its parent
\texttt{(Base)Table} object.
\\However, there is a second \texttt{asTable} function accepting
a locking mode. In that way a subtable can be opened with a given
locking mode.

\section{How to lock and unlock}
As explained above locking and unlocking has to
be done explicitly for UserLocking and can optionally be done for
AutoLocking.

Function \texttt{Table::lock} is used to acquire a lock. It can be
specified if a read or write lock is needed.
If the lock cannot be acquired immediately, the process will be added
to a list (in the lock file) to indicate that it needs a lock. The
list is used by AutoLocking (see \htmlref{below}{TABLELOCK:AUTOLOCKING}).
Thereafter it will try again to acquire the lock until
\texttt{nattempts} is reached. It sleeps a little while between each
attempt. After 30 attempts a message is sent to the logger telling
that the process is waiting for a table lock.
If the lock could be acquired, the process is removed from the list.
\\Having acquired a lock means that the
internal table and storage manager buffers are refreshed as needed.

Function \texttt{Table::unlock} releases a lock. If table data have
been changed since the lock was acquired, it will flush the changed
table and storage manager buffers and indicate in the lock file which
storage managers were changed.

\subsection{class TableLocker}
It should be clear that when using lock/unlock explicitly, care should
be taken that the unlock is also done in case of exceptions. This can
be quite cumbersome. Therefore the class \texttt{TableLocker} has
been created. Its constructor acquires a lock, while the destructor
releases it. C++ scoping can be used to invoke the destructor
automatically. E.g.
\begin{verbatim}
  Table tab(``test.ms'', Table::UserLocking);
  {
    TableLocker locker(tab, FileLocker::Read);  // acquire read lock
    ... write data into the table ...
  }  // end of scope, so TableLocker destructor is called
\end{verbatim}
The nicest thing of using \texttt{TableLocker} in this way is that
in case of an exception
its destructor is called, thus the lock is always released.

\section{\label{TABLELOCK:AUTOLOCKING}AutoLocking working}
AutoLocking is a handy mode because it frees the user from having to
do explicit locking and unlocking. Furthermore it has the advantage
that it does not release a lock before another process needs it.
This advantage is at the same time the weakness of
AutoLocking, because it may take a while before a process holding a
lock recognizes that another process needs a lock. For this to
understand it is explained how AutoLocking works.
\begin{enumerate}
\item When table I/O is done, it is checked if the table is
appropriately locked. If not, it is tried to acquire a lock
with \texttt{TableLock::maxWait} as the maximum number of attempts
(default is trying forever).
\item Unlocking is also done automatically. After some I/O-s are done,
the Table System inspects the list in the lock file to see if another
process needs the lock. If so, it releases the lock.
The inspection interval can be defined in the \texttt{TableLock}
constructor and defaults to 5 seconds.
\end{enumerate}
In general this scheme works fine, but the problem is that if no I/O is
done, the Table System does not check if another process needs a lock.
This is especially a problem for glish clients as they can be idle for
some time waiting for a command to be given. To circumvent this problem
the function \texttt{Table::relinquishAutoLocks} can be used to
release locks on tables using AutoLocking. Either all such tables are
unlocked or only the tables needed by another process.

\subsection{Releasing AutoLocks in Glish Clients}
Care has been taken that glish clients do not hold AutoLocks too long.
Glish clients time out after some period and call
\texttt{Table::relinquishAutoLocks} to release AutoLocks at regular
intervals. 
The time out period is controlled by two aipsrc variables. They are:
\begin{itemize}
\item \texttt{table.relinquish.reqautolocks.interval} defines the
number of seconds to wait before relinquishing autolocks requested in
another process. The default is 5 seconds.
\item \texttt{table.relinquish.allautolocks.interval} defines the
number of seconds to wait before relinquishing all autolocks. The
default is 60 seconds.
\end{itemize}
The user can define these variables at will, but usually the defaults
suffice. 

\section{\label{TABLELOCK:NOREADLOCKING}NoReadLocking}
Normally locking should be used to read data from a table because in
that way it is assured that the data is always consistent.
However, in some cases it might be useful to be able to read data from
a table without having to acquire a lock. That could, for instance, be the
case for an online process filling a MeasurementSet. It should not happen
that such a process has to wait for a write lock because some other
process is holding a read lock.
NoReadLocking is possible with AutoLocking and UserLocking modes.

The NoReadLocking option makes it possible to read table data without
acquiring a read lock. It means that the internal buffers are not
automatically synchronized with the data on disk. It is possible
though to do that explicitly using the function
\texttt{Table::resync}.
For this to work well, the writer should flush its data regularly,
otherwise it may take a long while before data appears on disk.

Often a NoReadLocking reader is coupled to the writer by means of
interprocess communication. In such cases it is best that the writer
tells the reader when it should resync and read.

\section{Lock Information}
Sometimes it is not clear which process is holding a lock.
A glish function in \texttt{os.g} has been made to tell the user
if a table has been opened in some process and if and how it has been locked.
\begin{verbatim}
  dos.showtableuse ('test.ms')
\end{verbatim}
prints this information for the given table. The function
\texttt{dos.lockinfo} returns this information in a glish record.

The information contains the PID of a process holding the lock or
having opened the table. Note that in case of a read lock multiple
processes may hold a lock. The PID of only one process is given
in the information though.


\section{Overview of classes/functions related to locking}
A brief overview is given. For detailed information the relevant
documentation should be examined.

\subsection{class \htmladdnormallink{TableLock}
{../../aips/implement/Tables/TableLock.h}}
This class defines the locking mode which can be given to the \texttt{Table}
constructor. For AutoLocking mode a few parameters can be set.

\subsection{class \htmladdnormallink{TableLocker}
{../../aips/implement/Tables/TableLocker.h}}
This class can be used to acquire and release a lock, especially in
UserLocking mode. As described above, it is particularly useful to ensure
that a lock is released in case of an exception.

\subsection{\htmladdnormallink{Table}
{../../aips/implement/Tables/Table.h} functions}
\begin{itemize}
\item \texttt{Table} constructor accepts a \texttt{TableLock} argument.
\item \texttt{lock} tries to acquire a read or write lock and resync-s..
\item \texttt{unlock} releases the lock and flushes the table buffers.
\item \texttt{hasLock} tests if the table holds a read or write lock.
\item \texttt{canLock} tests if the table can acquire a read or write lock.
\item \texttt{flush} flushes the table buffers.
\item \texttt{resync} resync-s the table buffers with the data on disk.
\item \texttt{hasDataChanged} tests if the table has changed since the
last time this function was called.
\item \texttt{lockOptions} gets the current locking mode.
\item \texttt{isMultiUsed} tests if the table is used in another process.
\item \texttt{nAutoLocks} is a static function returning number of tables
using AutoLocking.
\item \texttt{relinquishAutoLocks} is a static function releasing some
or all AutoLocks.
\end{itemize}

\subsection{glish}
In \texttt{table.g} functions are defined similar to the ones in class Table.
\\In \texttt{os.g} the functions \texttt{showtableuse} and
\texttt{lockinfo} give information about table locking.
