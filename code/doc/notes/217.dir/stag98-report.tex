\documentstyle[11pt]{article}
\def\aipspp{AIPS$^{++}$}
\begin{document}


\title{Report of the AIPS++ Scientific \& Technical Advisory Group}

\author{STAG, (ed. R. Braun)}

\date{February 20, 1998}

\maketitle

%\tableofcontents

\section{Introduction \& General Comments}

The AIPS++ Scientific and Technical Advisory Group met for the second
time in Socorro on 9 and 10 February 1998. Substantial progress has
clearly been made in the project during the 15 months since the first
meeting of the STAG. Functionality is being extended and reorganized
into more useful components. Memory and performance problems have been
remedied to some extent. The newly developed ``Display Library'' may
provide a solid base for future developments. The recently appointed
documentation specialist is a valuable addition to the team.

Even so, it is sobering to look back at the recommendations made by
this group in our previous report (Note 198) and realize how relevant
most of them still are. At that time we stressed the need for a
comprehensive and realistic plan (in terms of timetable and manpower)
for the delivery of package functionality. This has not appeared. We
also stressed that the highest possible priority should be given to
providing an improved user interface. Although some examples were
presented to the STAG during demonstrations at this meeting, these
have not yet been available for assessment and feed-back in the field.
In our opinion, the existing user interface is a formidable barrier to
user acceptance of the package and is the main reason for the lack of
tester feed-back in response to the $\beta$ releases. Simply put, it is
just too painful for working astronomers to even exercise the system
at this time.

Our gravest concern is that the project achieve some degree of
community support on the shortest practical time-scale. The strategy
which has been in place during the past year or so, was to provide a
small number of truly unique applications that would demonstrate the
power of AIPS++ to the sophisticated user. While we still support this
goal, we would stress again the absolute necessity of a ``pleasant''
user interface, and in addition, stress the need to provide at least a
few simple tools that allow a large fraction of the community to
exercise AIPS++ in a useful way. To this end, we recommend that very
simple end-to-end cross-calibration and imaging be provided for data
originating at the VLA, WSRT and ATCA within the first ``limited
public release''.  Much of the recommended simple functionality is
already present in the system, including limited archival data access
for both the WSRT and ATCA. We fully understand the practical
difficulties of providing direct archive access for VLA data.
We also understand that providing such a capability may
entail a significant delay in the public release schedule. We consider
such a delay preferable to squandering the very limited patience of
the user community for products that can't do anything useful for
them. One reason we stress improved support for VLA data is our
perception that the North American community in particular is both
rather critical of the project at this time and represents an
enormous untapped resource for constructive feed-back.  

While we applaud the recent formation of a testing group of
astronomers within NRAO Socorro, it seems to underline the very
limited role that NRAO astronomy (in contrast to computing) staff has
taken in the project to date. We consider it vital that this be
remedied. More extensive and continuous astronomer input is critical
at this stage to insure a relevant and successful product.

\section{Content of Releases}

\subsection{$\beta$ Releases}

As significantly enhanced functionality or interfacing becomes
available it should be released for testing and feed-back via this
method. We can envision several more releases of this variety before
going public. A release frequency of once in 6 months might be
realistic, so as not to generate an undue amount of overhead. It is of
particular importance that all of the content of the Limited Public
Release receive thorough $\beta$ testing before distribution. 

\subsection{Limited Public Release}

The components that the STAG deemed essential for a successful limited
public release are listed below. As stated above, all components
should first be subjected to thorough internal and $\beta$ testing
before public distribution.

\begin{itemize}
\item  polished auto- or custom- GUI's for each application 

\item  parameter setting shell for non-GUI processing

\item  cookbook and improved documentation 

\item  improved on-line help, including ``bubble'' help
  
\item simple end-to-end calibration and imaging capability for at
  least VLA, WSRT, ATCA interferometers and GBT single dish data,
  implying a stable updated MeasurementSet format
\end{itemize}


\subsection{Programmer's Release}

It was felt that the programming community is sufficiently small and
(potentially) well-connected to the project that a separate release
aimed at external programmers did not deserve a high priority.


\section{AIPS++ Core}

We stress again the dire need for a user-friendly interface. More
sophisticated GUI interfaces including graphical inputs should be
addressed in the medium term. The use of auto-GUI's, while 
convenient in the short term, do not seem to offer the final solution.

Glish has matured to an extent that little more development is needed
apart from bug fixes and possibly a debugging environment. Similarly,
tables and the low level library are reasonably mature and do not 
appear to require more development. This would suggest that a strong
shift in emphasis is needed from system software to interfaces and
applications. 

In the area of data display and graphics, we commend the development
of the ``display library''. However, this general area seems to have
become splintered, with simultaneous use of several different plotting
libraries in different applications.  We would like to see a
consolidated design of future graphics and display capabilities.
Visibility editing/flagging might be a good application with which to
test both the design and the library.  An application to take over the
functionality of AipsView should probably be developed within the next
18 months.

Hiring a documentation specialist is a commendable step that should
provide ample pay-back.  A well researched and tested cookbook is a
critical component of the documentation. Although not optimal, there is
no urgency to replace the current "latex2html" system. While we
recognize the importance of Web-based documentation, this should not
be pursued to the detriment of high-quality paper documentation.

\section{Single Dish Processing}

Substantial progress has been realized in the evolution of SDCalc to
the GUI-driven DISH environment. This is an example of the very
positive role an interested and active astronomer can have in keeping
a software effort focused on results. Unfortunately, much work still
remains to be done to integrate DISH with AIPS++ MeasurementSets,
while the more general problem outlined in our previous report, of
planning support for multi-dimensional SD data-streams, is being
approached in an add-on fashion rather than proceeding from a global
design.

\section{Specific Technical Issues}

The canonical machine: It remains a worthwhile goal to achieve
satisfactory performance on modest machines with only 64~MB memory,
since this will still be typical of user machines for at least the
next year.

Platforms: We recommend supporting only those OS's which are currently
available on short to medium time-scales.  A Windows port may be
important in the longer term, but should not draw significant
resources at present.

Compilers: It seems prudent to wait for the C++ standard to be widely
available before moving to standard compliance. It also appears very
desirable to retain a public domain compiler if at all possible.

Parallelization: Since the vast majority of the user community will
not have ready access to this processing mode on a short time-scale,
care should be taken that the efforts in this direction do not have a
negative impact on other aspects of system readiness.

Performance: There have been encouraging improvements in this area.
However, AIPS should not be used as the sole standard for performance
comparisons. It is important that available resources in a machine
should be utilized wherever possible to achieve performance improvements
(e.g. big memory machines should use big memory models).

Native FITS support: Providing the capability of direct application
access to FITS format files on disk did not seem to warrant a high
priority. 

AIPS -- AIPS++ interoperability: While the transparent use of some of
the AIPS functionality from within AIPS++ might be an effective method
of testing applications in the short term, this effort should not be
pursued if it results in a significant drain of system resources.

Error images and error propagation: This is potentially a bottomless
pit. The framework for the association of an error image or data
should be designed and implemented, but we do not advocate extensive
work on error propagation within applications at this time.

\section{New Functionality}

As part of an overall plan to implement increased functionality, we
recognize several areas where new applications would be
particularly welcome and competitive: these include mosaicing, wide
field imaging and polarimetric imaging.

\section{STAG Membership and Attendance}
\parindent=0cm

Robert Braun (NFRA) chair

Jayaram Chengalur (NCRA) not present for this meeting

Roger Foster (NRL) not present for this meeting

Dennis Gannon (Univ. Indiana) not present for this meeting

Walter Jaffe (Univ. Leiden)

Lee Mundy (Univ. Maryland) not present for this meeting

Bob Sault (ATNF)

Lister Stavely-Smith (ATNF)

Dave Shone (NRAL)

Doug Tody (NOAO)

Huib Jan van Langevelde (JIVE)

Tony Willis (DRAO)

Al Wootten (NRAO)
\end{document}
