\section{Introduction}

In this document, will the discuss the more basic standards and suggestions
for \emph{C++} development. The conventions of Plum and Saks
will be used for terminology \footnote{Plum, Saks, C++ Programming Guidelines, 
Plum Hall, 1991}; a standard is a rule which is nearly always followed, 
and a guideline is a recommendation. Wherever possible, a coding example will be provided
to clarify the current discussion, and a justification will be given for all
standards and guidelines.

This is intended as a living document, and as such all of its contents are
open to discussion and debate. Additions can be
added if it can be successfully argued that they will help the maintainability,
understandability, or reusability of \textsc{aips++} software, or sections can be
removed it can be argued that their benefits do not 
significantly increase the quality of \textsc{aips++} software. It is very important to
maintain a minimal set of standards to avoid a overwhelming barrage of
rules which could result in the standards not being followed.

There will be two documents which will accompany this document. One will describe
the format for source code documentation and the tool(s) for extracting the 
documentation and creating \emph{texinfo} files or \emph{man} pages. Perhaps, at
some point, the coding standards and documentation standards will be combined, but
for now they will be separate. The second will describe the exception mechanisms 
which will be used for signaling and handling errors.

There are several other sources of information utilized in creating this 
document. Meyers book on effective \emph{C++} programming
proved useful \footnote{Meyers, Effective C++, Addison Wesley, 1992}, but
many of the points which this book makes is also in the ``C++ Programming Guidelines''.
However, the ``C++ Programming Guidelines'' is much more terse. The \textsc{ARM}
\footnote{Stroustrup, Ellis, The Annotated C++ Reference Manual, 
Addison Wesley, 1990} was also used. Since the \textsc{ARM} is the \textsc{ANSI} draft
standard for \emph{C++}, it is the \emph{last word} when questions arise, and
it also contains interesting commentary on language features.

\section{Standards}

\subsection{\texttt{0} and \texttt{NULL}}
The integer constant \texttt{0} should be used instead of the usual \texttt{NULL} from
\texttt{<stdio.h>} or \texttt{<string.h>}. In \emph{C}, this \texttt{\#define} was used to insure
that pointer assignments had the proper size. In \emph{C++}, however, this can 
result in type mismatch problems, e.g. \texttt{NULL} may be defined to be 
\texttt{(void *)0}. Zero,``\texttt{0}'', holds a special place in the \emph{C++} type 
system because even though zero's type is \texttt{int}, it is automatically cast to 
any pointer, char, or float. (See p.30 Plum, p.87 Meyers, p.35 ARM)

\subsection{Header Files}
Every include file must have protecting \texttt{\#if defined()}s 
to prevent multiple inclusion, so for ``aips.h'' the include file would be 
enclosed by:
\begin{verbatim}
\#if !defined(AIPS\_H\_)
\#define AIPS\_H\_

           BODY OF INCLUDE FILE

\#endif
\end{verbatim}

Header files should not initialize values. The user of a given header file should
be assured that including the header file will not increase the object-code
size. This practice forces the data to be owned by a given source file, and
avoids complicated initialization constructs in the header 
files. (See p.154 Plum)


\subsection{\texttt{const}, \texttt{\#define}, and macros}
Constants, \texttt{\#define}s, and macros, should have the smallest scope possible;
restricted to a single source file when possible to avoid name clashes.
The \emph{C++} \emph{const}, \emph{inline}, and \emph{enum} facilities 
should be used as opposed to defines, because these facilities provide a 
typesafe means of defining constants, as opposed to the preprocessor which 
defeats the type system, and can make macro bugs difficult to find. 
Sometimes, however, \texttt{\#define}s must be used, for example include file 
protection against multiple includes.

Another example of preprocessor utility is the \texttt{\#\#} operator for 
concatenation of tokens. This has no equivalent operation in \emph{C++}.
In addition, \texttt{\#ifdef}s also provide the only reasonable means of 
conditional compilation. e.g.
\begin{verbatim}
\#if defined(\_\_cplusplus)
           C++ CODE
\#elif defined(\_\_STDC\_\_)
           ANSI C CODE
\#else
           K\&R C CODE
\#endif
\end{verbatim}

With the exception of include file protection, preprocessor functions 
should be used with caution. The \texttt{\#\#} operator, for example, is not 
available on K\&R \emph{C} compilers. The typical work around is
to define a macro which uses an empty comment, \texttt{/**/}, for
concatenation. Careless \texttt{ifdef}s for conditional
compilation throughout code make it difficult to maintain.

When the preprocessor is utilized macro or defines which begin with a 
double underscore, ``\_\_'', or single underscore, ``\_'', should be 
avoided because are used by \emph{C} and \emph{C++}.

\subsection{Memory Allocation}
\texttt{new} and \texttt{delete} will be used for memory allocation and deletion 
instead of the malloc/free family of functions because, for objects, new and 
delete allow for destructors to clean up storage allocated by the object. 
In addition, the use of new and delete allows redefinition the global 
new and delete, \texttt{::new} and \texttt{::delete}. By redefining these 
global operators, allocation/deallocation inconsistencies can be tracked 
down to the source code line, with the help of the preprocessor. Thus new 
and delete allow for finer control of the memory allocation process for 
all dynamic data. (p.18 Meyers) However, this method is useful primarily for
testing, and no code should be checked into the system with \texttt{::new} or
\texttt{::delete} defined because defining these affects the whole project. 
Modifications to \texttt{::new} or \texttt{::delete} must be approved by the
various member of the consortium.


\subsection{Name Space Management}
To avoid name space conflicts, global variables should be avoided. A better
approach is to use static member variables of a class.
To avoid type clashes we will use the method which was used in InterViews
\footnote{Linton, et. al., \emph{InterViews 3.0} Class Library}
A macro will be defined,
\begin{verbatim}
\#define \_lib\_aips(name) aips\#\#name
\end{verbatim}
\noindent
which will create a token unique to the aips project. This will be generated by 
sed/egrep which will pull out class definitions and generate the defines
for the class name. So a define will be generated for each class name,
\texttt{\#define MyClass \_lib\_aips(MyClass)}. Thus aips scope can be entered
by including a given file, \texttt{aips-enter.h}, and aips scope can be exited by 
including a different file, \texttt{aips-exit.h}.

Thus to enter an \textsc{aips++} code section the programmer would include the file
\texttt{aips-enter.h} and this file would be automatically generated from the 
other \textsc{aips++} header files and might look like:
\begin{verbatim}
// No multiple inclusion protection to allow reentering AIPS++
//    scope in the same file, file contains only \#defines.

\#define String \_lib\_aips(String)
...MORE SIMILAR DEFINES...
\end{verbatim}
\noindent
Thus every time a developer referenced a member of class \texttt{String} the
class name would be replaced with \texttt{\_lib\_aips(String)} which would
expand as described above to \texttt{aips\#\#String} and finally to 
\texttt{aipsString}. Then to exit the scope of aips code the developer would
include file \texttt{aips-exit.h} which is also automatically generated. This
file might look like:
\begin{verbatim}
\#undef String
...MORE SIMILAR UNDEFS...
\end{verbatim}
\noindent
Thus each reference to \texttt{String} would now reference a string defined
possibly in some other class library. Hopefully, this sort of context
switching will be the exception rather that the rule, and developers 
will simply include \texttt{aips\_enter.h} at the top of a source file
and not have to worry about name clashes.

In addition, the use of identifiers which begin with a double underscore,
``\_\_'', or a single-underscore, ``\_'', should be avoided. Identifiers 
which begin with a double underscore are reserved for \emph{C++} and 
those which begin with a single underscore are reserved for \emph{C}.

\subsection{Gotos}
The use of gotos in code should be avoided. This well used standard is 
especially applicable in object oriented design. Gotos in code make the
code much more difficult to read and violate logical scoping.
(See Plum, p.126)

\subsection{Switch Statements}
Switch statements should be used when applicable. However, designs which
rely upon switch statements based on the \emph{runtime type} of objects
should be avoided. Occasionally, this may be the only solution, but
it should be cause of reevaluation of the design.

\section{Guidelines}
\subsection{Encapsulation}
Object operations should be accomplished via member functions. Users of the
object should have no access to the data which the object contains and the
data contained in the object should be as protected as possible. In addition,
member functions should not return pointers or references to any internal
protected data.

\subsection{Comments}
The format of comments is described in another document(work in progress). However,
it is often useful to use the \emph{C++} style of comments, \texttt{//}, as
opposed to the \emph{C} style of comments, \texttt{/* */}. By using
the \emph{C++} style, whole sections of code can be commented out using
\emph{C} style comments while debugging a section of code, without worrying
about nested \emph{C} comments.(See Meyers p.16) Of course another
technique for \emph{commenting out} a section of code is to put an
\texttt{\#if 0} around it. However, here again care must be taken not to 
begin or end the \texttt{\#if 0} inside a \emph{C} comment.

\subsection{Functions}
\subsubsection{Parameters and Return Values}
Pass and return large objects by reference instead of by value, although
sometime objects \emph{must} be returned by value. This allows for much more efficient 
invocation of functions because when objects are passed by value, the constuctors 
for the object are called to create a duplicate copy of the object. This can be an 
expensive operation. A reference to an object should be returned whenever 
it is desirable for the return value to be used as an lvalue. For example,
it might be nice to have an \texttt{operator[]} for an array class. In this case,
the return type of the \texttt{operator[]} should be a reference, so that operations 
like \texttt{A[2] = 9} can be performed. Likewise the \texttt{operator=} should
return a reference so that \texttt{a = b = c} is possible where a, b, and
c are some objects with the \texttt{operator=} defined.

However, sometimes an object must be returned as opposed to a pointer or
reference to a object. For example, when defining an addition operator for
complex numbers:
\begin{verbatim}
class Complex \{
  friend Complex operator+(const Complex \&,const Complex \&);
  public:
    Complex(float x=0, float y=0) : r(x), i(y)\{\};
  private:
    float r,i;
\}

inline Complex operator+(const Complex \&a, const Complex \&b)\{
  return(Complex(a.r + b.r,a.i + b.i));
\}
\end{verbatim}
\noindent
The important thing to note is that the \texttt{Complex} constructed in
\texttt{operator+} is destroyed as a temporary in expressions like 
\texttt{d = a + b + c} where a, b, c, and d are all instances of the 
\texttt{Complex} class. If the storage were dynamically allocated
within the \texttt{operator+}, there would be no way of freeing the 
storage generated in the \texttt{a + b} sub-expression above 
(See Meyers, pp. 78-84). 

\subsubsection{Operators}
If you want conversion to happen on both the left and right hand sides of
an operator make the operator a friend instead of a member. For example,
if we extended the previous example to provide a \texttt{operator-} member
function, we would end up with a class that looks like:
\begin{verbatim}
class Complex \{
  friend Complex operator+(const Complex \&,const Complex \&);
  public:
    Complex operator-(const Complex \&b){ return(Complex(r - b.r, i - b.i));};
    Complex(float x=0, float y=0) : r(x), i(y)\{\};
  private:
    float r,i;
\};

inline Complex operator+(const Complex \&a, const Complex \&b)\{
  return(Complex(a.r + b.r,a.i + b.i));
\}
\end{verbatim}
\noindent
Both of these classes would work with expressions like \texttt{c = a + b} or
\texttt{c = a - b} where a, b, and c are instances of class \texttt{Complex}, but
in the case of, \texttt{c = 100 + a;} and \texttt{c = 100 - a}, only the 
\texttt{operator+} would succeed because the constructor with defaults
would be applied to \texttt{100} to produce a temporary of type \texttt{Complex}.
In the case of \texttt{operator-} the compiler has no way of knowing to which
type the \texttt{100} should be converted (See Meyers, 66-71).

\subsubsection{const and functions}
Use const parameters for functions whenever possible, use const pointers and
references as return values to avoid the cost of object creation, and
use const member functions whenever the function does not modify the
object. Using the \texttt{const} specifier allows for further specification
of the type signatures of functions. By declaring the parameters as 
\texttt{const}, the function accepts a wider number of arguments, i.e. both
const and non-const arguments. By declaring the function as const, the
potential users of the class are told that they can call the function without
worrying about modifications to the object. By returning const pointers and
references, the integrity of the object is maintained without the cost
of copying a portion of the object.  

\subsection{Parameterized Types/Templates}
Parameterized types should be used to express a family of types. The most 
obvious use is for parameterized container classes which will contain
heterogeneous data, e.g. a linked list. "Do Not use polymorphism 
(derived classes with virtual functions) to implement parameterized 
types."\footnote{Plum, p.60} Thus heterogeneous objects should not be
derived from a common class simply for the purpose of making a container
class or whatever from the common class. The primary reason this is 
problematic is that the heterogeneous objects are free to mix in 
the container class, and the only way to maintain a homogeneous collection
is by runtime type checks. Of course, sometimes this is the desired behavior.

\subsection{Object Interface}
The object interface should be a minimal interface, particularly the public
interface. Members should be as protected as possible, and only the
members designed for users of the class should be public. In addition,
if possible the members should be private, and only when they are intended
for use by derived classes and not for object users should they be made 
protected.

\subsection{Casts}
Although some times type casts are necessary, they should, in general, be
avoided. The presence of casts undermines the \emph{C++} type system. Each
cast should be clearly documented. Also, the casting away of const variables
and objects should be avoided. In fact, attempted modifications to
a const object can either cause an addressing exception or modify the object
as specified. The choice is implementation dependent\footnote{ARM p.71}.
Downcasts from a virtual base class are always illegal. Sometimes, casts can
be eliminated by using virtual functions in the base class to perform the
necessary operations. Whenever possible the burden of performing operations
specific to derived classes should be shifted from a case statement on
``object type'' to virtual functions and inheritance.

\subsection{Nested Types}
Advantage can be gained by using nested type to avoid name conflicts in the 
global name space. The type names can be made local to the class which
utilizes them. For example:
\begin{verbatim}
class A \{
 public:
  enum data \{ IN, OUT, UP, DOWN \};
  data getData()\{return x;\}
  void setData(data z)\{ x = z;\}
  A() : x(IN)\{\};
 private:
  data x;
\};

main()\{
  A a;
  A::data t;
    t = a.getData();
    a.setData(A::UP);
\}
\end{verbatim}
\noindent
Here \texttt{data} is a type local to the class \texttt{A}, but since it is
public it can still be used as a type name and the elements of the 
enumeration can be accessed by prefacing them with the class specifier,
\texttt{A::}. The elements of the enumeration, \texttt{IN, OUT,} etc. must
be unique, i.e. there could not be another enumeration with any of the
same elements as the \texttt{data} enumeration within \texttt{A}.

\subsection{Error Handling}
Exceptions will provide the primary means of flaging and handling errors.
An exception class, will be built for general use. It will be based upon
the OpenSys Inc., \emph{Exc C++ exception enabling library} and the 
1988 Usenix paper on exceptions \footnote{Miller, Exception Handling Without
Language Extensions, 1988 Usenix Proceedings, C++ Conference, pp.327-341}.

\subsection{IO Streams}
It is often advantageous to use the \texttt{<iostream.h>} package because it
is extensible. That is the designer of a package can design functions to 
print out his/her classes.

\subsection{Library Names}
To identify members of libraries, e.g. math, system, etc., the classes 
or stand alone public functions belonging to these libraries should be 
prefixed with one ore two letters to denote their membership to the 
library. Thus, for example:
\begin{verbatim}
class mSin \{
  ...etc...
\};
\end{verbatim}
\noindent
would belong to the \textsc{aips++} math library. In general, mixed case will
be used to distinguish words within an identifier, e.g. 
\texttt{MyLongIdentifier}.

\subsection{Format}
Many feel that it is important to have a format guidelines. These are
standards which make the code ``more readable'' by having a indicating
how control structures and code blocks should look. There are three
primary standards \footnote{Plum p.181}. Of these, the Kernighan and
Ritchie format is the control structure format for \textsc{aips++}.
So for example:
\begin{verbatim}
main(int argc, char *argv[])\{
  char *x;
  for (int i=1; i < argc; i++) \{
    int j = i+1;
    while (j < argc) \{
      if (strcmp(argv[j],argv[i]) < 0) \{
        x = argv[i];
        argv[i] = argv[j];
        argv[j] = x;
      \} else \{
//      argv[i] = argv[i];                       // Format of else
      \}
      j += 1;
    \}
  \}

  for (i = 0; i < argc; i++) \{
    cout << argv[i] << "\n";
  \}
\}
\end{verbatim}

\section{Warnings}

\subsection{Machine Size and Ordering Dependencies}
\subsubsection{Size of Builtin Types}
Concerning the sizes of the standard integral types the following are 
guaranteed \footnote{Stroustrup 2ed p50}:
\begin{itemize}
\item
\texttt{1 == sizeof(char) <= sizeof(int) <= sizeof(long)}
\item
\texttt{sizeof(float) <= sizeof(double) <= sizeof(longdouble)}
\item
\texttt{sizeof(\emph{t}) == sizeof(signed \emph{t}) == sizeof(unsigned \emph{t})} where t is one of the basic types
\end{itemize}
\noindent
Additionally, the following are the minimum sizes guaranteed for the integral
types \footnote{Stroustrup 2ed p50} and the floating point 
types \footnote{ARM p.24}:
\begin{itemize}
\item
\texttt{char} --- 8 bits
\item
\texttt{short} --- 16 bits
\item
\texttt{long} --- 32 bits
\item
\texttt{float} --- 32 bits
\item
\texttt{double} --- 64 bits
\item 
\texttt{long double} --- 96 bits
\end{itemize}
\noindent
The size of an \texttt{int} is typically 16 to 32 bits, but this size is not
guaranteed (32 bits in all probability for machines running \textsc{aips++}). The
size of the integral and floating point types are available in the header
files \texttt{<limits.h>} and \texttt{<float.h>} respectively.
(see p.72 Plum)

In addition, dependencies on the byte ordering of numeric types should be 
avoided. For example, the following is a bad idea:
\begin{verbatim}
short i = 25;
char *cptr;

cptr = (char *) \&i;
\end{verbatim}
\noindent
One cannot anticipate the byte to which \texttt{cptr} points, given the possible
difference in architecture byte orderings (p.74 Plum). In addition, when 
passing integral values between machines, the values should be converted 
to/from network byte ordering via the routines in \texttt{<netinet/in.h>}, 
e.g. htonl, etc.

\subsubsection{Structure Size and Member Offset}
Another possible source of alignment errors, is hard-coding offsets into 
structures because the packing of structure members into the storage
is implementation dependent. The proper way to determine the offset
is to use the \texttt{offsetof} macro in \texttt{<stddef.h>}(see p. 84 Plum).
So for a struct like:
\begin{verbatim}
struct mystruct \{
  int x;
  int y;
\};
\end{verbatim}
\noindent
\texttt{offsetof(mystruct, y)} might yield 4.

\subsubsection{Character Constants and String Literals}
Character constants should contain only one character because differences
in machine byte order may lead to values which differ in numeric value of
character sequence, for example:
\begin{verbatim}
short crlf = '\r\n';
\end{verbatim} 
\noindent
is a non-portable use of character constants. A literal string can be used
if appropriate, or a left shift can be used to portably generate a 
integral value out of characters, e.g. 
\begin{verbatim}
\#define CHAR2(a, b) (((a) << CHAR\_BIT) + (b))
\end{verbatim}
\noindent
portably combines two characters \footnote{Plum, p.76}.

String literals could also be a source of problems. They should not be 
modified instead named arrays should be used. In the following example,
the first method is the portable way to obtain a modifiable string; the
second is the wrong way \footnote{Plum, p.86}:

\begin{enumerate}
\item
\texttt{static char fname[] = "/tmp/edXXXXXX"; mktemp(fname);}
\item
\texttt{mktemp("/tmp/edXXXXXX")}
\end{enumerate}

\subsection{Creation and Deletion}

\subsubsection{Initialization}
It is important to remember that global static objects which have a constructor
or are initialized with a constant expression are initialized in the order
in which they are encountered \footnote{Plum, p.138}. There are ways to ensure
a certain initialization order \footnote{Schwarz, "Initializing Static
Variables in C++ Libraries", C++ Report. Vol1 No2, Feb. 89, pp1-4.}.

Another possible source of problems is with the initialization
of member variables in a constructor definition. The member variables are
initialized in the order in which they occur in the class, regardless of their
order in the constructor definition. So for example:
\begin{verbatim}
class A\{
  private:
    int x;
    int y;
    int z;
  public:
    A() : z(12), y(9), x(z+y) \{\};
    A(int t) : x(8), y(t), z(0) \{\};
\};
\end{verbatim}
\noindent
will cause problems, because its order of initialization will always be ---
x, y, then z. This choice was made to have consistent initialization orders. 
Otherwise, if the initialization order was dependent upon the order in the
constructor definition, the two constructors in this example would have
different initialization orders \footnote{Meyers, p.42}.

In general, it is best to avoid designs which depend on the initialization
order of static globals or static members. Also, it is important to remember
the initialization order of member variables to avoid problems.

\subsubsection{Deletion of Arrays}
When deleting an array of objects, it is important to call the delete 
operation with the array specifier. Otherwise, the destructors for each of the
elements in the array will not be called. Only the destructor for the first
one will be called. So for example, if we have a string class which allocates
memory to hold the string:
\begin{verbatim}
class String\{
  private:
    char *rep;
  public:
    String(char *);
    String()\{ rep = 0;\}
    const char *operator*()\{ return rep;\}
    String \&operator=(char *);
    ~String()\{ delete rep;\}
\};

String::String(char *v)\{
  if (v != 0)\{
    rep = new char[strlen(v)+1];
    strcpy(rep,v);
  \} else
    rep = 0;
\}

String \&String::operator=(char *newval)\{
  if (rep != 0)
    delete rep;
  rep = new char[strlen(newval)+1];
  strcpy(rep,newval);
  return(*this);
\}

main() \{
  String *s = new String[3];

    s[0] = "Hello";
    s[1] = "there";
    s[2] = "friend";

    delete s;
\}
\end{verbatim}
\noindent
the call to \texttt{delete s} only causes the destructor for the first
\texttt{s[0]} to be called. The space allocated for \texttt{there} and 
\texttt{friend} is lost forever. The correct way to delete \texttt{s} is 
\texttt{delete [] s} \footnote{Meyers, p.19}.

Another interesting point is that for an array of objects to be allocated
via \texttt{new} a \emph{default constructor} has to be defined, i.e. a
constructor without parameters \footnote{ARM p.61}. Specifying a constructor 
with all default parameters is not sufficient.

\subsubsection{Virtual Destructors}
Sometimes it is important for destructors to be declared as virtual in a
base class. This is important when all of the following hold 
\footnote{Plum, p.208}:
\begin{itemize}
\item
There are classes derived from the base class.
\item
The destructor in the derived class is different form the base class 
destructor.
\item
Derived class objects may be deleted via a pointer or reference to base
class objects.
\end{itemize}
\noindent
It is important for the destructor to be virtual when these conditions hold
because when deleting a derived object via a pointer to the base class,
only the destructor for the base class will be called. The derived class'
destructor will not be called, thus possibly causing memory leaks.

\subsubsection{References}
Another cause of memory problems is references. It is generally a bad idea to
initialize a reference to memory which can be deleted, e.g. free store, 
automatic variables. Typically this problem will result in segmentation
violations or the like, and not in memory leaks. However, a substantial amount
of time could be expended tracking down the source of error (see p.38 Plum).

\subsubsection{New and Delete}
Some problems can arise with classes which define the operators \texttt{new} and
\texttt{delete} with more than one argument because these definitions hide
the global or base class definitions. For example \footnote{Meyers, p.25}:
\begin{verbatim}
typedef void (*PEHF)();

class X \{
  public:
    void *operator new(size\_t, PEHF pehf);
\}

main()\{
  void specialErrorHandler();

  X *px1 = new (specialErrorHandler) X; 
  X *px2 = new X;
\}
\end{verbatim}
\noindent
The first \texttt{new} succeeds with no problems. The second however causes an
error because the global \texttt{new} which takes one parameter, \texttt{size\_t},
is hidden by the local \texttt{new} in class \texttt{X}. This problem can be
corrected by either calling the global new explicitly, \texttt{X *px2 = ::new X},
or by providing a \texttt{new} operator in \texttt{X} which takes only one
parameter. It is also a good idea to supply both \texttt{new} and 
\texttt{delete} if one is needed to avoid future maintenance problems.

\subsection{Copy Operator}
Be aware that if you don't explicitly disallow the copy operator, 
\texttt{\emph{Type}::operator=(\emph{Type}\&)}, one will be supplied for
you, i.e. a bit copy. One way to ensure that there is no copy operator
is to make its declaration private, preventing users from accessing it, and not
providing a definition, preventing \texttt{friends} from using it.


